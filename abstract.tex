% $Log: abstract.tex,v $
% Revision 1.1  93/05/14  14:56:25  starflt
% Initial revision
%
% Revision 1.1  90/05/04  10:41:01  lwvanels
% Initial revision
%
%
%% The text of your abstract and nothing else (other than comments) goes here.
%% It will be single-spaced and the rest of the text that is supposed to go on
%% the abstract page will be generated by the abstractpage environment.  This
%% file should be \input (not \include 'd) from cover.tex.
%%
% Quoting
% http://www.eecs.mit.edu/docs/grad/EECS_Thesis_Proposal_and_Thesis_Guidelines.pdf#page=9
%
% After a thesis has been completed, its further value is largely
% dependent on the extent to which it is read and found useful by
% others.  It is important to supply a well-written abstract, which
% outlines the scope and achievements of the thesis so that
% prospective readers can determine whether or not they should read
% any further. An additional advantage is gained because the abstract
% will in many cases enable the library staff to catalogue the work
% more fully and more accurately.  Accordingly, the Committee on
% Graduate Programs requires that each thesis contain an
% abstract--preferably one typewritten page (single-spaced), but in no
% case more than two such pages--in which is given a description of
% the problem and of the procedure used in the investigation, together
% with a brief statement of the results found or of the conclusions
% reached. Other material may be included in the summar y if you find
% it pertinent. Your objective is to inform another engineer or
% scientist, who is not necessarily a specialist in your field, what
% you worked on, how you did it, and what one may expect to learn
% about the problem by reading further
%
% For submission to MIT library: Abstracts should be no longer than
% 350 words, longer abstracts will be edited by ProQuest
%
% https://libraries.mit.edu/distinctive-collections/thesis-specs/#graduate

%\todo{fix abstract}
%In critical software systems there are opposing pressures to innovate and to let things be as they are.
Formal verification is increasingly valuable as our world comes to rely more on software for critical infrastructure.
A significant and under studied cost of developing mechanized proofs, especially at scale, is the computer performance of proof generation.
This dissertation aims to be a partial guide to identifying and resolving performance bottlenecks in dependently typed tactic-driven proof assistants like Coq.

We present a survey of the landscape of performance issues in Coq, with micro- and macro-benchmarks.
We describe various metrics that allow prediction of performance, such as term size, goal size, and number of binders, and note the occasional surprising lack of a bottleneck for some factors, such as total proof term size.
To our knowledge such a roadmap to performance bottlenecks is a new contribution of this dissertation.

The central new technical contribution presented by this dissertation is a reflective framework for partial evaluation and rewriting, already used to compile a code generator for field-arithmetic cryptographic primitives which generates code currently used in Google Chrome.
We believe this prototype is the first scalably performant realization of an approach for code specialization which does not require adding to the trusted code base.
Our extensible engine, which combines the traditional concepts of tailored term reduction and automatic rewriting from hint databases with on-the-fly generation of inductive codes for constants, is also of interest to replace these ingredients in proof assistants' proof checkers and tactic engines.
Additionally, we use the development of this framework itself as a case study for the various performance issues that can arise when designing large proof libraries.
%
We also present a novel method of simple and fast reification, developed and published during this PhD.

%We identify three main categories of workarounds and partial solutions to performance problems: design of APIs of Gallina libraries; changes to Coq's type theory, implementation, or tooling; and automation design patterns, including proof by reflection.
Finally, we present additional lessons drawn from the case studies of a category-theory library, a proof-producing parser generator, and cryptography code generation.
%verified compiler and code generator for low-level cryptographic primitives.
