\begin{quote}
  The most common mistake in performance engineering is to blindly optimize without profiling; you most often spend your time optimizing parts that aren't actually bottlenecks.
\end{quote}
\begin{flushright}
  --- Charles Leiserson (heavily paraphrased, reconstructed from memory of 6.172)
\end{flushright}

\begin{quote}
  premature optimization is the root of all evil
\end{quote}
\begin{flushright}
  --- Donald Knuth~\cite[p.~671]{KnuthPrematureOptimization}
\end{flushright}

\begin{quote}
  Testing shows the presence, not the absence of bugs
\end{quote}
\begin{flushright}
  --- Edsger Wybe Dijkstra, 1969~\cite{naur1969software}
\end{flushright}


\chapter{Introduction}\label{ch:intro}

\section{Introduction}\label{sec:intro:intro}

%\paragraph{Paragraph 1: Orient the reader to very broad things} Software w/o bugs via formal methods
As our world comes to rely more and more on large and complex software, the impact of bugs in our code grows ever more severe.
Bugs in the software of the Therac-25, a radiation-therapy machine, killed three people and injured at least three more between 1985 and 1987 by delivering potentially leathal doses of radiation.~\cite{Therac}
\todoask{do I need a better citation than wikipedia?}
\todo{should I mention the pricing bug on amazon? https://www.usatoday.com/story/tech/2019/07/19/amazon-prime-day-error-sold-13-000-camera-equipment-94-report/1775314001/, https://www.theguardian.com/money/2014/dec/14/amazon-glitch-prices-penny-repricerexpress, http://www.michaeleisen.org/blog/?p=358}
More recently, unforseen edge-cases in DAO smart-contracts nearly cost investors over 40 million USD.~\cite{DAO2018Guecluetuerk}
\todo{should I mention anything else? the pentium bug costing Intel \$475 million? http://top-100s.blogspot.com/2009/05/10-most-expensive-software-blunders.html, \$500 million Ariane Rocket disaster? , https://www.olenick.com/blog/articles/infamous-software-bugs-fdiv-bug}
\todo{should I pull from https://xavierleroy.org/talks/Milner-award-lecture.pdf? ``Fiat-Chrysler recalled 1.5 M vehicles for software update.''}

While testing of various kinds is widely used to catch bugs, it's quite costly \todoask{should I email Xavier Leroy and ask if he has references on how much the airline industry spends on testing?} and does not guarantee a lack of bugs.
In some adversarial contexts, random testing cannot hope to eliminate all bugs.
For example, some bugs in cryptographic code only occur in as few as 19 out of $2^{255}$ cases.~\cite{curve25519-donna-commit-correct-bounds}
Suppose you wanted to catch this bug in a ``modest'' twenty years of continuous random testing.
You'd need to have over a thousand times as many computers as there are atoms in the solar system!
% https://www.wolframalpha.com/input/?i=%282%5E255%2F19+*+529812+%2F+%285+ghz%29+%2F+%2820+years%29+%2F+%28atoms+in+the+solar+system%29%29+*+1+atom = 533260
% (2^255/19 * 529812 / (5 ghz) / (20 years) / (atoms in the solar system)) * 1 atom

%\paragraph{Paragraph 2: Success of foundational tools}
One appealing solution to this problem is to \emph{prove} critical software correct.
The idea is that if we can write down in formal mathematical language what we'd like our software to do, then we can try to prove that the code that actually runs behaves in the same way that the math specifies.
Ideally, the specification will be relatively simple, and much easier to trust, than the millions of lines of code making up the software we use every day.

While proofs of algorithms tend to be done with pen-and-paper (consider the ubiquitous proofs that various sorting algorithms are correct in introductory algorithms classes), proofs of actual code are much harder.
For reasons of performance, historical accident, interfacing with real users, or otherwise, code in the real-world tends to be much more complicated than code in an algorithms textbook.
Proofs of code-correctness thus tend to be filled with tedious case-analysis with only sparse mathematical insights, and attempts to create and check these proofs by hand are subject to the same issues of human falibility as writing the programs in the first place.

%\todo{transition to foundational tools}
%\todo{explain how foundational tools like Coq are already useful}
Enter machine-checked proofs.
Foundational tools which allow users to write down theorems and proofs to be checked by machine free the proof author from error-prone case analysis; the computer will check that no cases were missed.
Although you can never gain complete confidence in your software---the layer underneath may be untrustworthy in ways you cannot see~\cite{Reflections1984Thompson}, you can never prove that the mathematical system you're working in is correct~\cite{sep-goedel-incompleteness}, and most proof-checkers are not yet themselves proven correct---machine-checked proofs allow a drastic reduction in how much code needs to be trusted.
Rather than trusting the millions of lines of code of the software being verified, you only need to trust the specifications of the software, and the likely-much-smaller codebase of the proof checker.
Furthermore, it's very unlikely that bugs in the proof checker will overlap with bugs in the proof in just the right way to allow bugs only in one particular piece of software to slip through.
Said another way, we no longer need to write tests for every piece of software separately; writing tests for the proof-checker helps eliminate bugs in \emph{all} software proven by that checker.

%\paragraph{Paragraph 3: performance of tools is important and non-trivial, and what we're doing here is explaining / demonstrating how and why it's non-trivial, and how to fix nontrivial performance issues}
We come now to the main thrust of this thesis.
Tools for machine-checked proofs have already had many successes, in both software verification and traditional mathematics proof formalization, from compilers~\cite{Compcert} to microkernels~\cite{seL4SOSP09} to webservers~\cite{Network2015Chlipala} to cryptography code~\cite{FiatCryptoSP19}, from the Four-Color Theorem~\cite{gonthier2008formal} to the Odd-Order Theorem~\cite{gonthier2013machine} to Homotopy Type Theory~\cite{HoTTBook}.
While compiler performance---both the time it takes to compile code and the time it takes to run the generated code---has long been an active field of study~\cite{CC++Performance2017,georges2007statistically,mytkowicz-wrong-data}\todo{find more citiations here}, to our knowledge there is no extant body of work systematically investigating the performance of proof assistants, nor even any work primarily focused on the problem of proof assistant performance writ large.
\todo{integrate stuff from \textcite{Efficiency1994Boulton} such as ``For example, much of the research in the field of automated reasoning is concerned with producing proofs more efficiently''}
\todo{Related to this is work on sound interfaces to external tools, also an enduring topic: \textcite{TrustedSlind}}

\todo{work on this paragraph}
Why is this problem worthy of study?
This thesis argues that the problem of \emph{compile-time performance} or \emph{proof-checking performance} is both significantly different from the problem of performance of more typical programs in industry and that it is nontrivial.
While many papers mention performance, obliquely or otherwise, and some are even driven by performance concerns of a particular algorithm or part of the system,~\cites[p.~1382]{gonthier2008formal}{Efficiency1994Boulton}{Proving2005Benjamin}{Idris2Faster2020Brady}{Recognizing1989Benanav}{mechanical1990Pierce}{CelikETAL17iCoq}{PalmskogETAL18piCoq}{Gregoire-Leroy-02}{thesis-nogin}{Idris2Faster2020Brady}, we have not found any that investigate the performance problems that arise \emph{asymptotically}, when proof assistants are used to verify programs at large scale.
% ``While we tackled this project mainly to explore the capabilities of a modern formal proof system—at first, to benchmark speed—'' - gonthier2008formal
% ``We present a new implementation of a reflexive tactic whichsolves equalities in a ring structure inside the Coq system.The efficiencyis improved to a point that we can now prove equalities that were previ-ously beyond reach.'' - Proving2005Benjamin

Unlike many other performance domains, in our experience, the time it takes to check proofs as we scale the size of the input is almost always super-linear---quadratic at best, commonly cubic or exponential, ocassionally even worse.
Empirially, this might look like a proof script that checks in tens of seconds on the smallest of toy examples; takes about a minute on the smallest real-world example, which might be twice the size of the toy example; takes twenty hours to generate and check the proof of an example only twice that size; and, on a (perhaps not-quite-realistic) example twice the size of the twenty-hour example, the script might not finish within a year, or even a thousand years---see \autoref{sec:fiat-crypto-codegen-numbers} for more details.
In just three doublings of input size, we might go from tens of seconds to thousands of years.
Moreover, this is not an isolated experience in proof assistants; this sort of performance behavior is \emph{typical}.

Drawing experience from case studies in category theory~\cite{category-coq-experience}, parsers~\cite{jgross-masters-thesis}, and generation of low-level cryptographic code~\cite{FiatCryptoSP19}, we investigate and seek to explicate a broad swath of performance issues and bottlenecks.
While we propose some design principles for avoiding these performance issues in \autoref{part:design} and present a research prototype of a tool and methodology for achieving acceptable performance at scale in \autoref{part:rewriting}---and we claim both of these are original contributions of this PhD\todo{how to write PhD in text?  with periods? not?}---the story this thesis aims to tell is that in order for proof assistants to scale to industrial uses, \emph{we must get the basics of asymptotic performance right}.

\subsection{What are proof assistants?}\label{sec:intro:history}
Before diving into the details of performance bottlenecks and solutions, we review the history of formal verification and proof assistants to bring the reader up to speed on the context of our work and investigation.

Formal verification can be traced back to the early 1950's.~\cite{Brief2019Darbari}
The first formally verified proof, achieved in 1954, was of the theorem that the sum of two even numbers is even.~\cite{davis2001early}

\todo{\textcite{luckham1979stanford}}
\todo{wide look at various proof assistants (historical overview of how the field got to where it is today, maybe with weight according to importance) and successes (applications of proof assistants like compcert~\cite{Compcert}, sel4 (kernel)~\cite{seL4SOSP09}, cakeml~\cite{CakeML}, nqthm~\cite{Nqthm,wiki:Nqthm}, ACL2~\cite{ACL2,ACL2Applications}, crypto???, flyspeck~\cite{flyspeck2014Hales}, 4-color-theorem~\cite{gonthier2008formal}, odd-order-theorem~\cite{gonthier2013machine})}
\todo{https://www.eeweb.com/profile/adarbari/articles/a-brief-history-of-formal-verification}
\todo{look at OSDI submission (papers/lightbulb/) and mine related work}
\todo{read and cite Talia's paper \textcite{ringer2020qed}?}

\section{Basic Design Choices}\label{sec:intro:proof-assistant-design-choices}
\todo{introduce section where we talk about what some big design decisions are and why we might make them the way we do}

\subsection{Dependent Types: What? Why? How?}\label{sec:why-how-dependent-types}
%\subsection{Dependent Type Theory}
\todo{note: target audience: Saman (or POPL/PLDI)}
\todo{explain what dependent types are at all}
\todo{explain the idea of reducing proof checking to typechecking (with examples)}
\todo{explain idea of structuring programming so that some properties are true by construction}
\todo{maybe explain extensional vs intensional}
\todo{maybe mention lego as early implementation of dependent types}
\todo{explain dependent type theory, motivate using it}
\todo{read and cite \textcite{Should1999Lamport} and \textcite{Formalising2018Paulson} h/t Karl Palmskog @palmskog on gitter \url{https://gitter.im/coq/coq?at=5e5ec0ae4eefc06dcf31943f}}

\subsection{The De Bruijn Criterion}\label{sec:debruijn-criterion}
\todo{explain the De Bruijn criterion}
\todo{explain why it's worth following this, and what goes wrong if you don't}
\todo{look into \textcite{Sealing2020Selsam} Sealing Pointer-Based Optimizations Behind Pure Functions}
\todo{example: Ltac vs Gallina, kind-of}

\section{Look ahead: Layout and contributions of the thesis}\label{sec:intro:layout}
In the remainder of \autoref{part:introduction}, we will finish laying out the landscape of performance bottlenecks in dependetly-typed proof assistants;
\fullref{ch:perf-failures} gives a more in-depth investigation into what makes performance optimization in dependent type theory hard, different, and unique, followed by describing major axes of super-linear performance bottlenecks in \fullref{sec:perf-axes}.

\fullref{part:design} is devoted, by and large, to the performance bottlenecks that arise from the use of dependent types as the basis of a proof assistant as introduced in \autoref{sec:why-how-dependent-types};
in \fullref{ch:design}, we discuss lessons on engineering libraries at scale drawn from our case study in formalizing category theory %~\cite{category-coq-experience}
and augmented by our other experience.
The category theory library formalized as part of this doctoral work, available at \textcite{HoTT/HoTT-categories}, is described briefly in this chapter; a more thorough description can be found in the paper we published on our experience formalizing this library.~\cite{category-coq-experience}

\fullref{part:rewriting} is devoted, in some sense, to performance bottlenecks that arise from the de Bruijn criterion of \autoref{sec:debruijn-criterion}.
We investigate one particular method for avoiding these performance bottlenecks.
We introduce this method, variously called \emph{proof by reflection} or \emph{reflective automation}, in \fullref{ch:reflection}, with a special emphasis on a particularly common use case---transformation of syntax trees.
\fullref{ch:rewriting} describes our original contribution of a framework for leveraging reflection to perform rewriting and program transformation at scale, driven by our need to synthesize efficient, proven-correct, low-level cryptographic primitives.~\cite{FiatCryptoSP19}\todo{cite also the rewriter paper}
Where \autoref{ch:rewriting} addresses the performance challenges of verified or proof-producing program transformation, \fullref{ch:rewriting-more} is a deep-dive into the performance challenges of engineering the tool itself, and serves as a sort-of microcosm of the performance bottlenecks previously discussed and the solutions we've proposed to them.
Unlike the other chapters of this thesis, \autoref{ch:rewriting-more} at times assumes a great deal of familiarity with the details of the Coq proof assistant.
Finally, \fullref{ch:reification-by-parametricity} presents a way to efficiently, elegantly, and easily perform \emph{reification}, the first step of proof by reflection, which is often a bottleneck in its own right.
We discovered---or invented---this trick in the course of working on our library for synthesis of low-level cryptographic primitives.~\cite{FiatCryptoSP19,reification-by-parametricity}

\fullref{part:conclusion} is in some sense the mirror image of \autoref{part:introduction}:
Where \autoref{ch:perf-failures} is a broad look at what is currently lacking and where performance bottlenecks arise, \fullref{ch:coq-tooling-fixes} takes a historical perspective on what advancements have already been made in the performance of proof assistants, and Coq in particular.
Finally, while the present chapter which we are now concluding has looked back on the present state and history of formal verification, \fullref{ch:conclusion} looks forward to what we believe are the most important next steps in the perhaps-nascent field of proof-assistant performance at scale.

\begin{subappendices}

\section{TODOs from \autoref{sec:intro:intro}}
%\todo{read / look into \textcite{georges2007statistically}}
%\todo{maybe read \textcite{mytkowicz-wrong-data}}
%\todo{Karl Palmskog: @Jason Gross I saw your post on performance optimization/measurements for proof assistants on Coq-Club. We summarize a lot of work on parallelization/selection for proof assistants in our regression proving papers (ASE 17 \& ISSTA 18): \textcite{CelikETAL17iCoq} \textcite{PalmskogETAL18piCoq}}
\todo{read and cite \textcite{lean-tactic-language}}
\todo{read \textcite{Implementing1998Shao}}
``It's not directly related to proof assistants, but the techniques described can be applicable to proof assistants and the experience \emph{may} be applicable to some extent.''
\todo{read \textcite{Inductive2003Brady}}
\todo{read \textcite{thesis-nogin}}
\todo{read Grégoire and Leroy's paper from ICFP 2002 \textcite{Gregoire-Leroy-02}}
\todo{read Dirk Kleeblatt's PhD thesis \textcite{Strongly2011Kleeblatt}.}
``Both of these are about using compiled code in dependent type checkers instead of interpreters.''
\todo{read András Kovács has smalltt at \textcite{smallttKovacs}}``, which I don't think has been written up anywhere but has nonetheless been influential, both on Idris 2 and on Olle Fredriksson's reimplementation of Sixten at \url{https://github.com/ollef/sixty}''
\todo{look at \url{https://github.com/AndrasKovacs/normalization-bench}}
\todo{look at \url{https://github.com/AndrasKovacs/smalltt}}
\todo{``I haven't yet updated the smalltt repo, but there's a simplified (\url{https://gist.github.com/AndrasKovacs/a0e0938113b193d6b9c1c0620d853784}) implementation of its evaluator, which seems to have roughly the same performance but which is much simpler to implement.''}
%\todo{\cite{Idris2Faster2020Brady}}
\todo{\cite{NewCoqTactics2016Pedrot} about why not LCF tactics in dependently typed setting}


András Kovács wrote:
\begin{quotation}
The basic idea is that in elaboration there are two primary computational tasks, one is conversion checking and the other is generating solutions for metavariables. Clearly, we should use NbE/environment machines for evaluation, and implement conversion checking in the semantic domain. However, when we want to generate meta solutions, we need to compute syntactic terms, and vanilla NbE domain only supports quote/readback to normal forms. Normal forms are way too big and terrible for this purpose. Hence, we extend vanilla NbE domain with lazy non-deterministic choice between two or more evaluation strategies. In the simplest case, the point of divergence is whether we unfold some class of definitions or not. Then, the conversion checking algorithm can choose to take the full-unfolding branch, and the quoting operation can choose to take the non-unfolding branch. At the same time, we have a great deal of shared computation between the two branches; we avoid recomputing many things if we choose to look at both branches.

I believe that a feature like this is absolutely necessary for robust performance. Otherwise, we choke on bad asymptotics, which is surprisingly common in dependent settings. In Agda and Coq, even something as trivial as elaborating a length-indexed vector expression has quadratic complexity in the length of the vector.

It is also extremely important to stick to the spirit of Coquand's semantic checking algorithm as much as possible. In summary: core syntax should support *no* expensive computation: no substitution, shifting, renaming, or other ad-hoc term massaging. Core syntax should be viewed as immutable machine code, which supports evaluation into various semantic domains, from which sometimes we can read syntax back; this also leaves it open to swap out the representation of core syntax to efficient alternatives such as bytecode or machine code.

Only after we get  the above two basic points right, can we start to think about more specific and esoteric optimizations. I am skeptical of proposed solutions which do not include these. Hash consing has been brought up many times, but it is very unsatisfying compared to non-deterministic NbE, because of its large constant costs, implementation complexity, and the failure to handle sharing loss from beta-redexes in any meaningful way (which is the most important source of sharing loss!). I am also skeptical of exotic evaluators such as interaction nets and optimal beta reducers; there is a good reason that all modern functional languages run on environment machines instead of interaction nets.

If we want to support type classes, then \href{https://arxiv.org/pdf/2001.04301.pdf}{tabled instance resolution} \textcite{Tabled2020Selsam} is also a must, otherwise we are again smothered by bad asymptotics even in modestly complex class hierarchies. This can be viewed as a specific instance of hash-consing (or rather ``memoization''), so while I think ubiquitous hash-consing is bad, some focused usage can do good.

Injectivity analysis is another thing which I believe has large potential impact. By this I mean checking whether functions are injective up to definitional equality, which is decidable, and can be used to more precisely optimize unfolding in conversion checking.

I'd be very interested in your findings about proof assistant performance. This has been a topic that I've been working on on-and-off for several years. I've recently started to implement a system which I intend to be eventually ``production strength'' and also as fast as possible, and naturally I want to incorporate existing performance know-how.
\end{quotation}
\todo{look into ``So technically, the lost sharing is the second-order sharing that is preserved
    in ``optimal reduction'' of lambda calculi [Levy-1980, Lamping-1990, Asperti-Laneve-1992],
    while hash consing normally is directly usable only for first-order sharing.''}
\todo{look at \url{https://math.stackexchange.com/questions/3466976/online-reference-book-for-implementing-concepts-in-type-theory}}
\todo{look at \url{https://github.com/AndrasKovacs/elaboration-zoo/blob/0c7f8a676c0964cc05c247879393e97729f59e5b/AIMprez/AIMprez.pdf} or \url{https://eutypes.cs.ru.nl/eutypes\_pmwiki/uploads/Meetings/Kovacs\_slides.pdf}}
Konrad Slind wrote:
\todo{read \textcite{Programming2000Barras}}
\todo{\textcite{Efficiency1994Boulton}}

\begin{comment}
\section{Transcript bits from talking with Adam}
And my phone is now recording.

Yeah, so my son's this story is that the there'll be sort of introduction and at some point I'll have to introduce caulk and some amount of detail and I'm not sure where exactly that bit of it goes. But the main thing I want to talk about in the introduction is.

Like what makes performance in caulk and assistance especially dependently type ones different from performance and other programming languages, let me suggest here for the very beginning. I would try to write introduction that doesn't go into a lot of detail about conflict but paints a broad picture.

Say sort of like going through a breath first traversal of the material starting with the higher level motivation, so that's whoever reads it understands what you're trying to accomplish and what major it's a progress you made towards those goals and then chapter two. Presents more the details on call background, that'll be needed to understand the precise rules of the game and what's going to come later okay so then I'm thinking the chapter one in the introduction is something like performance in crew or like proof assistance or thing preface systems are important performance improve assistance is important yeah.

I'm not sure how much it feels like in order to like talk about why performance improve assistance is important. I need to say something about like what makes it different like why it's not already solved. The information at a very high level there so the the like highest level the sketch here is that in most languages the performance looks something like EU write something and it works and maybe you have to optimize it a little and like as you get bigger examples, it slowly gets slower.

Whereas in caulk the experience is that he writes something in that works and you get bigger examples and it gets a bit slower and then you make your examples just a little bit bigger and now it takes a week or a month or like unclear just I'm sure we can find examples like that in traditional software also like you you just pushed your working set beyond the cast size or.

Something like that.

I think that's true put my senses that in preface this then it like this is. This is just how the like. It seems like this is pervasive in preface systems. I think this is the wrong level abstraction for. Section one. I would first try to convey the big message of the detention between flexibility and trust in a free persistent building out alcohol methods tools, we can typically get around a lot of these issues.

I say okay, so. So is this section where I want to talk about the divide between kernel trusted code base and the rest of it you might try starting out with with just introducing the debate on criteria remind me what that does. Roughly what you just said small currently that's in terms of some sort of record of approved that can be appreciated in terms of only a small senators okay, yeah that seems good.

And you might want to give an idea for what dependent types are what's the peel they give people using them and some sort of fuzzy idea for the challenges that might emerge are.

Yeah.

I'm trying to figure out what it feels like. I have like two very different levels that I can pitch dependent types at and not sure if either one is adequate for thesis. One of them is explaining dependent types as in many languages you want to know at compile time if you try to pass an integer when you're function is expecting a string yeah in caulk you like you have this part of these system on steroids where you can do things like oh.

You need to pass element of the empty set if the turning machine does not halt and if the turning machine does halt the need to pass an element of the one element set. There was definitely skipping forward way through an explanation of what depends sorry why are we here?

Well represented about the actual way is so this is sort of the like what? What makes dependent types powerful and challenging. The the what dependent types are. I feel like my technical explanation is something like they let the home. So one version of it is that they return value for your the return type of your function gets the depend on the value of the arguments.

Yes, that does sound like a definition. I mean, it's it's not.

Hit. To only works precisely if you're either fully curried or like it doesn't capture segment types exactly or something.

Sure. Okay. I'm right. And no, and then the the turning machine bit is an explanation of like what? What dependent types let you do and how they let you encode proofs and this part of the system. I don't think it's an explanation of why it pays off to do things that way.

They're really important to have in the first section. I feel like I don't have a good explanation of why why it appears off to be a superficial systems on type theory, rather than doing something like Isabel Hall. It's gonna be a problem, but if you spend a while. Theseus explaining how to handle dependent types properly and you have an explained why that was the right size choice to put them in there in the first place.

Yeah. It feels like currently I like to have the knowledge about to have an argument for them. Okay.

Other than like lots of people do it or something. I think part of it is connected to the small trusted go-based story of how tribulation checking works. Really don't have another option but using types of things sometimes. I feel like if you base everything on the axioms of sub theory, you can have a.

TCP that's much smaller than that of call. Maybe. We get the same performance advantages ignore this issue that there's a design choice that creates a large fraction of the challenges and the thesis and not explain why it's there. If you want to start out by saying there was a time when we were confused and thought this was a good idea and now we're going to trace through the consequences.

That's better than trying to ignore it.

I think I want guidance from you on what to do here. It feels like the best I can say is like this is a choice that lots of people make and we're going to look at like assuming that you've made this choice like what follows? Maybe should try to pull people who seem like on a bashed fans of a family type programming improve assistance even today and get looked there.

There didn't justifications over. I think it's fun. Okay, it's a good start. Which might translate into what happens to match every people of intuition as well. I think that's close correlative fun here. Yeah.

I guess I can like email. Call club or something.

Okay. Okay, I will plan to email golf club. It's good. And figure out something to write. Feels like this will be one of the one of the week links in my thesis. Okay. Okay, so there will be some chunks that's like talking about dependent types and.

You won't you want to make a similar introduction and motivation for every major feature that leaked into interesting and significant challenges in the work you presents.

Yeah, so I guess the it feels like the two main.

Like two main features that I'm going to talk about are motivated by the Deborah and Criterion and using dependent types.

I think there's something about computation for efficiency being built into certain parts of the system, that is. Fundamental here having compiled my thoughts. Yeah. I feel like that that's sort of runs through both of them or something. So we're sorry that like runs through both so I'm with my overall plan is that there'll be this introduction section, there'll be another section that like introduces call can talks about alike.

Survey of what performance issues and calls look like from a like bird's eye view. Okay, and then there'll be a section on.

\section{more transcript from Adam}
I expect. In terms of where I expect to have trouble feels like I expect that. I'll have a lot of trouble making the technical details of the rewriter the targeted right level.

And I'll have trouble sort of weaving all of the different parts together coherently.

Because of what to be feels like it's sort of inherent. Between the category theory part and the the article is very important yeah. I feel like. Like I like the current way that I'm dealing with that of like we're going to look at like two completely different parts of the system that like overlapping little bit and how you solve them or like ways to solve them and like one of them is convergent.

And then the other one is or like one of them is conversion and the root of the issue is dependent types and then the other one is ah. Program transformation and rewriting in the root of the issue is this separation between the trusted code base and the parts that you freely optimize.

Okay. And like, Ah in the conversion that there's like a couple solutions one of them is that is basically all the way on the never call conversion end of things and the other one is on the like shove everything into the type level and this is where the like the amber efficient computation shows up and then in the the program transformation and rewriting section the solution that we're using is the shove everything into the piano.

Okay.

Did you bitch you reification in your little walk-through? I did not but I expect the the. Program transformation section is going to or like the program transformation chapter is going to split and will have. For like we'll have a bunch of sections and one of them will be on like.

Proof by reflection and then another one will be on like reification and some part of that will be on reification by parametricity. It's probably the case of the content from that paper as long enough that it shouldn't just be one section within a chapter. It's the reader. Feel happy about making a progress if you don't have evidence long chapters.

Length of an independent research paper as a good standard for roughly how long a chapter can get I could instead of calling them chapters call them like half parts yeah and I'm beneath that have chapters and have like perk introduction that has the like true introduction and then the like here is called and performance issues in call and then part conversion.

And then part.

I don't know what to call the next word. It's not you're writing in production. I mean, maybe it's rewriting in reduction but it's also like it also covers fruit by reflection in general, but if you're general pattern is apart of corresponds to a problem and then that treatment of each problem you introduce a solution including the background for it then makes sense to me, then you'd have reflection be introduced within there well within the first portion be conversion, it should be API design or something, okay?

Yeah, I could do that. Oh so API design is a part and then in part that's on rewriting and reduction.

With chapters on. The like introduce the problem the talk about proof by reflection talk about reification and reification by parametricity.

And then talk about the rewriter and then your like talk about the performance of the rewriter and the like broad strokes of it and then talk about the technical details and challenges and implementing the rewriter and then there's like a part conclusion.

That talks about the like let's look back on on performance over the past decade and talk and see where we've like made strides and what this can say about future proof assistance and then also look forward and be like here's the like next challenged tackle and performance of previous systems like call.

Yeah. Sounds good to me. Cool was that the level of detail you were looking for here home.

I think I. To the part that I feel most fuzzy on still is the introduction, oh.

I could go into more detail and tell you it feels like I've given a good level of detail of the outline, okay, oh.

Except it feels like I still don't know how to do the introduction or split it up. Okay, so you want to drill down I think I want to drill down more on the introduction unless you think it's better to save the introduction part for later. Do we have anything higher priority by later?

I meant after I read the other parts of the thesis or something, okay? You try to productively use the remaining time in this meeting and if you don't have a better idea of what to do, we should talk about. The only other idea of that I have for what to do is to tell you more bits of the story for from the other parts.

When it sounds like you're much more confident that you have that story inside you. Yeah, it feels like I'm I'm like still a little bit fuzzy, but I like every time I tell that it gets more clear and I feel confident that this will continue to be the pattern.

Whereas for the introduction, I feel like Like every time I tell it it's completely different and. It's not getting any more clear. You know, most people these things don't take shape until they're actually writing. You didn't necessarily expect to reach it fixed point by speaking it over and over again.

Yeah that way I feel like I'm I'm trying out a methodology of this like recording and using transcripts. Yeah and.

Then like taking the transcripts and putting them into tech and polishing them and soon it might be the case that's speaking it. Is closer to writing. Okay the way I'm trying to approach this.

\section{more Adam transcript}

So tell me the story of land called Intro. Okay, so. One cold in true has. Two three chapters.

It feels like I I know how to end the introduction more than I know how to start it. How do you end it? Oh so I ended with a the like final chapter in the introduction is a sort of a painting of where map of like. This is what performance and caulk looks like.
\section{more Adam transcript}
Ah. Okay, so things that need to go in the earlier parts. What are what is proof assistant and what is cock? To proud criterion.

Whatever dependent types. Maybe like what is what is conversion what is or I don't know if I introduce conversion on the here if I save it for later. Should you included what is a proof assistant but you didn't include why do we care about them, okay, maybe you all said mind.

I mean, I do have it in mind but I didn't have it in mind here idea why do we care about purpose systems, what are they already been used for was the basis of confidence that this is a painful tool? So this is like like look at all the prior work things that.

Shouldn't shouldn't less literally yeah.

And then why do we care about performance and purpose systems?

I do worry that there is prior work that I'm not going to find on performance improvement systems. I feel like I'm not currently aware of much other than maybe now there's. I think there's like making. I feel like there's some things that like touch on performance, that's like. Canonical structures for less ad hoc automation and.

The maybe some of burglaries stuff on reflection. If yep stars not raise and so forth native stuff.

Yeah.

I can't explain to you don't exhausted literature search there myself, okay. I shouldn't it's done some time it's returned alert literature yeah. I'm spending a few hours trying to find some other things out there. I remember what I did that with a much smaller time parameter than a few hours right before the poll deadline.

I found this this paper from the the Isabel crowd doing allows him with things and, Wasn't the ideal time to realize that yeah for the the rewriter paper yeah, yeah. Apparently we just never done a web search before for. Something like rewriting normalization by calculation proof assistance came up pretty quickly, yeah.

I feel like we ran into a similar issue with pressures and that I dove into implementing part series without having read any of literature on pursuers.

So what's these? Push down literature search. Early in the writing process this time even if it's too late to be early in the research process, you know. I will in to do that, okay?

Happy set enough about chapter one. I don't think so. I think we've like thrown a bunch of things out but I like and like I know how to say a little bit about each of them, okay, but I don't I don't know how to say enough about or like I don't know what is enough about each of them some of them.

I don't know how to say enough about them and I definitely don't know how to weave them into a story. While you're trying to introduce main performance element aspects of performance metal and aspects of. Particular. Design philosophy using caulk in some related systems, you're trying to explain why they were introduced originally.

And something of the challenges for the user that they introduced.

Were trying we need to keep our focus on not having this come across as here's a system that someone threw at us and we figured out how to use it well. That could be the nature of the experiment we ran to answer larger design questions, but we have we want to keep relating back to the larger questions.

I feel like when I when I try to imagine the intricate trend I keep running into the problem where. I can like. Say a lot of things and eventually I'll get to where I want to go, but. The it feels like I'm going to leave the reader not knowing what we're doing for the chapter or two.

That's good it's good it's it it's good relative to what's possible. What we're doing is so technical that we need to introduce a lot of background before anyone can appreciate it.

I feel like it would be good to give them a sense and maybe even like part of me wants to be like we should give them a sense in the first paragraph that we're like, I don't know dealing with performance issues and verified in life. Making. Systems not have bugs or something okay, yeah if we're just literally putting that phrase into the first paragraphs.

Sounds plausible. So that would evangelize more detail. I've been something that's like roughly it feels like sort of what I'm want to do is I want to orient the reader. I want to be like here here's what here's the broad thing of what we're going to look at and I want that and I don't know the first sentence the first paragraph.

That worries the like first sentence of the second paragraph. And then I want to like introduce some amount of context and then be like okay now that you have this context. I can orient you better like here's a better version of what we're going to be doing and then that paves the way for more context and then I can be like, okay now that you have this more context.

Like here's an even better version of what we're going to be doing and either that will be the last iteration or one more iteration and be like okay now that you have all of the context now I can actually start talking about what we're doing you just need to give yourself permission to write low-polity texts ready to revise it later, ah feels like a skill that I've never learned when I'm very handy one.

It feels like so the skill that I do have is like talk to someone with a low quality explanation, okay, and then as they express confusion revise on that. And it feels like that's a suddenly different skill yeah. It's just really hard to get oh you work your way up to complex information if you're just speaking it.

There's a reason we use written explanations, what do you mean? Working memories not sufficient to. Receive a complex idea just by listening to someone talk with no other visual aids up. So it may be that you're. By forcing yourself to use the conversational medium, you're so eliminating you're the set of what you could possibly convey you for you restricted your attention to such easy things and you feel like you're making progress, but oh.

It feels like the things that I can convey to the conversational medium or enough to get me to the point where I'm comfortable writing details or something, okay? Like it feels like like we have the rewriting paper and like, Even if I throw out the introduction bits of it.

I feel like I should be able to get to the point where I should be able to get up to the meat of it with the conversational medium and then just take the written made of it. I feel like you might need to point to code examples to do that.

I could believe that.

It feels like I'm floundering much earlier in the process.

Like I'm floundering that the orient the reader step. You know.

Tempted to just try again with the. Telling you that to know that I have this picture about what I'm trying to do with the introduction which I did not have before okay attempted to just try again to give you the story of the introduction, okay for minutes give you as much as I can give you in three minutes, let's do it, okay, so.

Story is that Jesus is going to look at. How.

We're looking at verified or at getting systems that don't have bugs in them and how to. Be performant when doing this what's hard about being performant and like how to. Succeed you haven't mentioned a aspect of foundational tools that a small trusted basis. I think that is central to this.

Depends on how you interpret it says it's not having buck you might be worried about bugs in the fall methods tools. Which case perhaps this is the unified. Oh people could use the nudge with a more explicit framing. So, I feel like I want that to be in the.

Non leave like super initial contacts but in the in the like background after the first contact setting that's like, okay the way that we're going to the like tools that we're using for not forgetting systems without bugs are proof assistance and these are foundational tools and here's this large body of work that's about how this has been useful and like why this is a reasonable way to get systems without bugs make sense as a buildup principle for the introduction.

I think throughout most of the paper you want the top-level frame of the problem people's minds to be fun. Damentally about foundational tools. Yeah, I think I'm going to.

Like. I think by paragraph three. I want to stop talking about or will. Yeah, I feel like by something around paragraph three of the introduction. I want to like like we're not. We're not talking about other ways of getting systems without bugs, we're talking about proofs. And purposes those sorts of foundational tools, okay?

And.

I feel like now I need to say something about performance and I don't know what to say.

Can't use the tools. I don't get what's that's kind of the maybe I just say that oh and then.

So I'm tempted here to be like okay and like this is this is what makes performance in pure physicists different but I feel like. You're suggesting from earlier was don't do that here. Like like my story wasn't compelling enough about why performance is bad and purpose. I think talking about this dope the proof system is this organism we found in the jungle we're going to tell you what's why isn't how you deal with it isn't near this convincing as talking about a fundamental trade-off between flexibility and trust.

Okay, so it goes here is where I. Want to introduce the grind criteria.

And I feel like then I want to introduce dependent types and maybe this is very.

Um and that. I'll need to like pull cockclub or something. And then I feel like now I want to reorient it's also three o'clock. Okay, oh and then I want to like reorient where. Like what we're going to be doing or something. Like spiral back to the.

So we're looking at. Performance improvement systems and these two issues you're going to generate. Like broad swaths of the. System where performance issues occur and I'll be talking about. What the performance issues look like and also ways to solve them. And then after that I get more context on.

In cock here is this palette of performance issues and like what they look like okay, so that seemed like a good introduction sketch. That level of abstraction you have my. Cult thanks sure. I'm assuming that you don't want to have any input into this strategy.


\section{Transcript bits from Talking with Rajee}
High level story:

Coq and proof assistance are important. Performance in them is important, especially at scale. Performance engineering in proof assistants has some unique challenges that don't show up in other programming languages.

And I want to paint a picture of what the unique challenges are.  There are two main areas in the existing system that I want to call attention to, in regards to performance bottlenecks.  I will describe them, and describe the performance issues, and propose some reasons about why there might be performance bottlenecks, and describe solutions for them. And maybe also there'll be another section that has some miscellaneous other performance bottlenecks.
\end{comment}

\section{Coq's design}
\todo{where does this description of Coq's design go?}
Coq is split into two parts.

There's the part of the system that is called the kernel or the trusted code base.
Once you get a proof this part will be like ``yup, I believe the proof'' or like ``nope your proof is bad.''
And then there's the other part that is like ``here's magic and it will make proof for you.''
You're like ``I have an arithmetic expression please prove that it's true'' and there's a bit in this other part that's like ``I know how to prove arithmetic expressions'' and it gets the arithmetic expression and then it generates a certificate or proof that this other trusted part checks.
If the part generating the arithmetic proof is wrong then the users come complaining to you that you have a bug.
If the part checking the proofs is wrong, then you don't see the bug and now suddenly your users can prove anything they want.
And the system is no longer trustworthy right and so for that bit of it you need to be very careful with any changes you make.
\end{subappendices}
%\begin{itemize}
%  \item Mention domains: math/CT and program transformation
%  \item Case studies: parsers, fiat-crypto, CT library
%\end{itemize}
\todo{this chapter}
