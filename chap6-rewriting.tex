\chapter{Engineering Challenges in the Rewriter} \label{ch:rewriting-more}

\begin{quote}
  premature optimization is the root of all evil
\end{quote}
\begin{flushright}
  --- Donald Knuth
\end{flushright}

\todo{Better chapter title?}
\section{Introduction} \label{sec:rewriting-more:intro}
\autoref{ch:rewriting} discussed in detail our framework for building verified partial evaluators, going into the context, motivation, and the techniques used to put the framework together.
However, there was a great deal of engineering effort that went into building this tool which we glossed over.
Much of the engineering effort was mundane, and we elide the details entirely.
However, we believe some of the engineering effort serves as a good case-study for the difficulties of building proof-based systems at scale.
This chapter is about exposing the details relevant to understanding how the bottlenecks and principles identified elsewhere in this thesis played out in designing and implementing this tool.

\section{A Brief Survey of the Engineering Challenges} \label{sec:rewriting-more:challenges-overview}

While the core rewriting engine of the framework is about 1\,300 lines of code, and early simplified versions of the core engine were only about 150 lines of code%
\footnote{%
See \href{https://web.archive.org/web/20200716002534/https://github.com/JasonGross/fiat-crypto/blob/3b3e926e4186caa1a4003c81c65dad0a1c04b43d/src/Experiments/RewriteRulesSimpleNat.v}{\texttt{https://github.com/JasonGross/fiat-crypto/blob/3b3e926e4186caa1a4003c81c65dad0a1c04b43d/src/Experiments/RewriteRulesSimpleNat.v}} for the file \texttt{src/Experiments/RewriteRulesSimpleNat.v} from \href{https://github.com/JasonGross/fiat-crypto/tree/experiments-small-rewrite-rule-compilation}{the branch \texttt{experiments-small-rewrite-rule-compilation} on \texttt{JasonGross/fiat-crypto}} on GitHub.%
}%
, the correctness proofs take nearly another 8\,000 lines of code!
% git ls-files "src/Rewriter/Rewriter/*.v" | grep -o 'src/Rewriter/Rewriter/[^/]*\.v' | xargs coqwc  | sort -h | less
% add up totals, subtract off the lines in Rewriter.v
As such, this tool, developed to solve performance scaling issues in verified syntax transformation, itself serves as a good case study of some of the pain that arises when scaling proof-based engineering projects.

Our discussion in this section is organized by the conceptual structure of the normalization and pattern matching compilation engine;
we hope that organizing the discussion in this way will make the examples more understandable, motivated, and incremental.
We note, however, that many of the challenges fall into the same broad categories that we've identified earlier in this thesis:
issues arising from the power and (mis)use of dependent types, as introduced in \fullref{sec:why-how-dependent-types};
and issues arising arising from API mismatches, as described in \fullref{ch:api-design}.

\subsection{NbE vs.~Pattern Matching Compilation: Mismatched Expression APIs}
\begin{itemize}
\item \todo{talk about rawexpr\_types\_ok}
\item \todo{talk about how we can unify types at all (c.f.~\texttt{preunify\_types})}
\end{itemize}
\subsubsection{The Pain of Type-Indexed Swap}
$\left.\right.$

\subsection{Patterns with Type Variables -- The Three Kinds of Identifiers}
$\left.\right.$

\subsection{Pre-evaluation}
$\left.\right.$
\subsubsection{CPS}
$\left.\right.$
\subsubsection{Type Codes}
$\left.\right.$
\subsubsection{What Can We Unfold?}
\begin{itemize}
\item \todo{talk about the ``known'' parameter of rIdent}
\item \todo{talk about multiple aliases like \texttt{option\_bind'}}
\item \todo{talk about lack of help from the compiler / type-checker}
\item \todo{talk about ``Note that here we are jumping through some extra hoops to get the right reduction behavior at rewrite-rule-compilation time.'' for \texttt{eval\_decision\_tree}}
\item \todo{foward-reference pain with casts?}
\item \todo{talk about whether or not to eliminate PositiveMap.t?} % ``In a possibly-gratuitous use of dependent typing to ensure that''
\item \todo{talk about the general tradeoff between runtime checks and static proofs} % ``However, the proofs are much simpler if we simply do a wholesale check at the very end
\item \todo{talk about the cost of inconsistent decisions spreading pain elsewhere} % ``Here we pay the price of an imperfect abstraction barrier (that we have types lying around, and we rely in some places on types lining up, but do not track everywhere that types line up).''
\end{itemize}
\subsubsection{Revealing ``Enough'' Structure}
\begin{itemize}
\item \todo{talk also about tracking both the revealed and unrevealed structure}
\item \todo{talk about $\eta$-expanding identifier matches}
\end{itemize}

\subsection{The Let-In Monad: Missing Abstraction Barriers at the Type Level}
\todo{Talk also about the pain of wf statements for, e.g., \texttt{wf\_normalize\_deep\_rewrite\_rule}, having to go underneath multiple monads}

\subsection{Delayed Rewriting in Variable Nodes}
\todo{rValue vs rExpr}

\subsection{Relating Expressions and Values}
\todo{figure out where this goes, how to explain it, where it arises from:}
``In general, the stored values are only interp-related to the same things that the ``unrevealed structure'' expressions are interp-related to. There is no other relation (that we've found) between the values and the expressions, and this caused a great deal of pain when trying to specify the interpretation correctness properties.''

\subsection{Rewriting Again in the Output of a Rewrite Rule}
$\left.\right.$
\subsection{Which Equivalence Relation?}
\todo{talk about \texttt{rawexpr\_equiv}, 4-place \texttt{wf\_rawexpr}, nuance of revealing structure in CPS'd \texttt{eval\_decision\_tree}, non-obvious \texttt{wf\_value} binding list, \texttt{interp\_related\_gen} unsuccessfully avoiding funext (and also needing to be instantiated with \texttt{value\_interp\_related} sometimes), \texttt{rawexpr\_interp\_related} and separating (or not) goodness from relatedness, \texttt{rawexpr\_types\_ok}?, \texttt{unification\_resultT'\_interp\_related}?, \texttt{interp\_unify\_pattern'}?, interpretation-correctness related to rewriting again}

\subsection{Dependently Typed Pain in Applying Rewrite Rules}
\begin{itemize}
\item \todo{talk about ``There are two steps to rewriting with a rule \ldots''}
``\ldots\space both conceptually simple but in practice complicated by dependent types.
We must unify a pattern with an expression, gathering binding data for the replacement rule as we go; and we must apply the replacement rule to the binding data (which is non-trivial because the rewrite rules are expressed as curried dependently-typed towers indexed over the rewrite rule pattern).
In order to state the correctness conditions for gathering binding data, we must first talk about applying replacement rules to binding data.''
\item \todo{mention a trade-off here: note that we can't eliminate equality tests/casts early if we introduce them in the wrong place, c.f.~\texttt{app\_transport\_with\_unification\_resultT'\_cps}}
\end{itemize}

\subsection{Dependently Typed Pain in Indexing Over Types}
\todo{this subsection}
``We can define a transformation that takes in a \texttt{PositiveMap.t} of pattern type variables to types, together with a \texttt{PositiveSet.t} of type variables that we care about, and re-creates a new \texttt{PositiveMap.t} in accordance with the \texttt{PositiveSet.t}.
This is required to get some theorem types to line up, and is possibly an indication of a leaky abstraction barrier.''

\subsection{What's the Ground Truth: Patterns Or Expressions?}
\todo{default interpretation of a pattern --- complicated, but perhaps needed for phrasing correctness of unification}

\section{Leaky Abstraction Barriers: NbE vs.~Pattern Matching Compilation}

\clearpage

\todo{this chapter}
\todo{mention frowned-upon Perl scripts previously in BoringSSL(?) OpenSSL?; (ask Andres for reference?)} Perl scripts were complicated, a number of steps removed from actual running code, hard to maintain and verify.
\todo{Refer back to representation changes (good abstraction barriers / equivalences) being important in fiat-crypto, and being cheap only because we have a rewriter}
\setlistdepth{20}
\renewlist{itemize}{itemize}{20}%
\setlist[itemize,1]{label=\textbullet}%
\setlist[itemize,2]{label=\normalfont \bfseries \textendash}%
\setlist[itemize,3]{label=\textasteriskcentered}%
\setlist[itemize,4]{label=\textperiodcentered}%
\setlist[itemize,5]{label=\textbullet}%
\setlist[itemize,6]{label=\normalfont \bfseries \textendash}%
\setlist[itemize,7]{label=\textasteriskcentered}%
\setlist[itemize,8]{label=\textperiodcentered}%
\setlist[itemize,9]{label=\textbullet}%
\setlist[itemize,10]{label=\normalfont \bfseries \textendash}%
\setlist[itemize,11]{label=\textasteriskcentered}%
\setlist[itemize,12]{label=\textperiodcentered}%
\setlist[itemize,13]{label=\textbullet}%
\setlist[itemize,14]{label=\normalfont \bfseries \textendash}%
\setlist[itemize,15]{label=\textasteriskcentered}%
\setlist[itemize,16]{label=\textperiodcentered}%
\setlist[itemize,17]{label=\textbullet}%
\setlist[itemize,18]{label=\normalfont \bfseries \textendash}%
\setlist[itemize,19]{label=\textasteriskcentered}%
\setlist[itemize,20]{label=\textperiodcentered}%

\hypertarget{rewriter}{%
\section{Rewriter}\label{rewriter}}

\hypertarget{introduction}{%
\subsection{Introduction}\label{introduction}}

\begin{itemize}
\tightlist
\item
  The goal of the rewriter is to take an abstract syntax tree and
  perform reduction or rewriting.
\item
  There are three things that happen in rewriting: beta reduction,
  let-lifting, and replacement of rewrite patterns with their
  substitutions

  \begin{itemize}
  \tightlist
  \item
    Beta reduction is replacing \texttt{(λ\ x.\ F)\ y} with
    \texttt{F{[}x\ ⇒\ y{]}}. We do this with a
    normalization-by-evaluation strategy.
  \item
    Let-lifting involves replacing \texttt{f\ (let\ x\ :=\ y\ in\ z)}
    with \texttt{let\ x\ :=\ y\ in\ f\ x}. Note that for higher-order
    functions, we push lets under lambads, rather than lifting them; we
    replace \texttt{f\ (let\ x\ :=\ y\ in\ (λ\ z.\ w))} with
    \texttt{f\ (λ\ z.\ let\ x\ :=\ y\ in\ w)}. This is done for the
    convenience of not having to track the let-binding-structure at
    every level of arrow.
  \item
    Replacing rewriting patterns with substitutions involves, for
    example, replacing \texttt{x\ +\ 0} with \texttt{x}.
  \item
    There's actually a fourth thing, which happens during let-lifting:
    some let binders get inlined: In particular, any let-bound value
    which is a combination of variables, literals, and the identifiers
    \texttt{nil}, \texttt{cons}, \texttt{pair}, \texttt{fst},
    \texttt{snd}, \texttt{Z.opp}, \texttt{Z.cast}, and \texttt{Z.cast2}
    gets inlined. If the let-bound variable contains any lambdas, lets,
    or applications of identifiers other than the above, then it is not
    inlined.
  \end{itemize}
\end{itemize}

\hypertarget{beta-reduction-and-let-lifting}{%
\subsection{Beta-Reduction and
Let-Lifting}\label{beta-reduction-and-let-lifting}}

\begin{itemize}
\item
  We use the following data-type:

\begin{verbatim}
Fixpoint value' (with_lets : bool) (t : type)
  := match t with
     | type.base t
       => if with_lets then UnderLets (expr t) else expr t
     | type.arrow s d
       => value' false s -> value' true d
     end.
Definition value := value' false.
Definition value_with_lets := value' true.
\end{verbatim}
\item
  Here are some examples:

  \begin{itemize}
  \tightlist
  \item
    \texttt{value\ Z\ :=\ UnderLets\ (expr\ Z)}
  \item
    \texttt{value\ (Z\ -\textgreater{}\ Z)\ :=\ expr\ Z\ -\textgreater{}\ UnderLets\ (expr\ Z)}
  \item
    \texttt{value\ (Z\ -\textgreater{}\ Z\ -\textgreater{}\ Z)\ :=\ expr\ Z\ -\textgreater{}\ expr\ Z\ -\textgreater{}\ UnderLets\ (expr\ Z)}
  \item
    \texttt{value\ ((Z\ -\textgreater{}\ Z)\ -\textgreater{}\ Z)\ :=\ (expr\ Z\ -\textgreater{}\ UnderLets\ (expr\ Z))\ -\textgreater{}\ UnderLets\ (expr\ Z)}
  \end{itemize}
\item
  By converting expressions to values and using
  normalization-by-evaluation, we get beta reduction in the standard
  way.
\item
  We use a couple of splicing combinators to perform let-lifting:

  \begin{itemize}
  \tightlist
  \item
    \texttt{Fixpoint\ splice\ \{A\ B\}\ (x\ :\ UnderLets\ A)\ (e\ :\ A\ -\textgreater{}\ UnderLets\ B)\ :\ UnderLets\ B}
  \item
    \texttt{Fixpoint\ splice\_list\ \{A\ B\}\ (ls\ :\ list\ (UnderLets\ A))\ (e\ :\ list\ A\ -\textgreater{}\ UnderLets\ B)\ :\ UnderLets\ B}
  \item
    \texttt{Fixpoint\ splice\_under\_lets\_with\_value\ \{T\ t\}\ (x\ :\ UnderLets\ T)\ :\ (T\ -\textgreater{}\ value\_with\_lets\ t)\ -\textgreater{}\ value\_with\_lets\ t}
  \item
    \texttt{Definition\ splice\_value\_with\_lets\ \{t\ t\textquotesingle{}\}\ :\ value\_with\_lets\ t\ -\textgreater{}\ (value\ t\ -\textgreater{}\ value\_with\_lets\ t\textquotesingle{})\ -\textgreater{}\ value\_with\_lets\ t\textquotesingle{}}
  \end{itemize}
\item
  There's one additional building block, which is responsible for
  deciding which lets to inline:

  \begin{itemize}
  \tightlist
  \item
    \texttt{Fixpoint\ reify\_and\_let\_binds\_base\_cps\ \{t\ :\ base.type\}\ :\ expr\ t\ -\textgreater{}\ forall\ T,\ (expr\ t\ -\textgreater{}\ UnderLets\ T)\ -\textgreater{}\ UnderLets\ T}
  \end{itemize}
\item
  As is typical for NBE, we make use of a reify-reflect pair of
  functions:

\begin{verbatim}
Fixpoint reify {with_lets} {t} : value' with_lets t -> expr t
with reflect {with_lets} {t} : expr t -> value' with_lets t
\end{verbatim}
\item
  The NBE part of the rewriter, responsible for beta reduction and
  let-lifting, is now expressible:

\begin{verbatim}
Local Notation "e <---- e' ; f" := (splice_value_with_lets e' (fun e => f%under_lets)) : under_lets_scope.
Local Notation "e <----- e' ; f" := (splice_under_lets_with_value e' (fun e => f%under_lets)) : under_lets_scope.

Fixpoint rewrite_bottomup {t} (e : @expr value t) : value_with_lets t
  := match e with
     | expr.Ident t idc
       => rewrite_head _ idc
     | expr.App s d f x => let f : value s -> value_with_lets d := @rewrite_bottomup _ f in x <---- @rewrite_bottomup _ x; f x
     | expr.LetIn A B x f => x <---- @rewrite_bottomup A x;
                               xv <----- reify_and_let_binds_cps x _ UnderLets.Base;
                               @rewrite_bottomup B (f (reflect xv))
     | expr.Var t v => Base_value v
     | expr.Abs s d f => fun x : value s => @rewrite_bottomup d (f x)
     end%under_lets.
\end{verbatim}

  \hypertarget{rewriting}{%
  \subsection{Rewriting}\label{rewriting}}

  \hypertarget{there-are-three-parts-and-one-additional-detail-to-rewriting}{%
  \subsubsection{There are three parts and one additional detail to
  rewriting:}\label{there-are-three-parts-and-one-additional-detail-to-rewriting}}
\item
  Pattern matching compilation
\item
  Decision tree evaluation
\item
  Rewriting with a particular rewrite rule
\item
  Rewriting again in the output of a rewrite rule \#\#\# Overview
\item
  Rewrite rules are patterns (things like \texttt{??\ +\ \#?} meaning
  ``any variable added to any literal'') paired with dependently typed
  replacement values indexed over the pattern. The replacement value
  takes in types for each type variable, \texttt{value}s for each
  variable (\texttt{??}), and interpreted values for each literal
  wildcard. Additionally, any identifier that takes extra parameters
  will result in the parameters being passed into the rewrite rule. The
  return type for replacement values is an option UnderLets expr of the
  correct type.
\item
  A list of rewrite rules is compiled into (a) a decision tree, and (b)
  a rewriter that functions by evaluating that decision tree. \#\#\# The
  small extra detail: Rewriting again in the output of a rewrite rule
\item
  We tie the entire rewriter together with a fueled repeat\_rewrite; the
  fuel is set to the length of the list of rewrite rules. This means
  that as long as the intended rewrite sequences form a DAG, then the
  rewriter will find all occurrences.

\begin{verbatim}
Notation nbe := (@rewrite_bottomup (fun t idc => reflect (expr.Ident idc))).

Fixpoint repeat_rewrite
         (rewrite_head : forall (do_again : forall t : base.type, @expr value (type.base t) -> UnderLets (@expr var (type.base t)))
                                    t (idc : ident t), value_with_lets t)
         (fuel : nat) {t} e : value_with_lets t
  := @rewrite_bottomup
       (rewrite_head
          (fun t' e'
           => match fuel with
              | Datatypes.O => nbe e'
              | Datatypes.S fuel' => @repeat_rewrite rewrite_head fuel' (type.base t') e'
              end%under_lets))
       t e.
\end{verbatim}
\item
  This feature is used to rewrite again with the literal-list append
  rule (appending two lists of cons cells results in a single list of
  cons cells) in the output of the \texttt{flat\_map} rule
  (\texttt{flat\_map} on a literal list of cons cells maps the function
  over the list and joins the resulting lists with \texttt{List.app}).
  \#\#\# Pattern matching compilation
\item
  This part of the rewriter does not need to be verified, because the
  rewriter-compiler is proven correct independent of the decision tree
  used. Note that we could avoid this stage all together, and simply try
  each rewrite rule in sequence. We include this for efficiency. TODO:
  perf comparison of this method.
\item
  We follow \emph{Compiling Pattern Matching to Good Decision Trees} by
  Luc Maranget
  (http://moscova.inria.fr/\textasciitilde maranget/papers/ml05e-maranget.pdf),
  which describes compilation of pattern matches in OCaml to a decision
  tree that eliminates needless repeated work (for example, decomposing
  an expression into \texttt{x\ +\ y\ +\ z} only once even if two
  different rules match on that pattern).
\item
  We do \emph{not} bother implementing the optimizations that they
  describe for finding minimal decision trees. TODO: Mention something
  about future work? Perf testing?
\item
  The type of decision trees

  \begin{itemize}
  \item
    A \texttt{decision\_tree} describes how to match a vector (or list)
    of patterns against a vector of expressions. The cases of a
    \texttt{decision\_tree} are:

    \begin{itemize}
    \tightlist
    \item
      \texttt{TryLeaf\ k\ onfailure}: Try the kth rewrite rule; if it
      fails, keep going with \texttt{onfailure}
    \item
      \texttt{Failure}: Abort; nothing left to try
    \item
      \texttt{Switch\ icases\ app\_case\ default}: With the first
      element of the vector, match on its kind; if it is an identifier
      matching something in \texttt{icases}, remove the first element of
      the vector run that decision tree; if it is an application and
      \texttt{app\_case} is not \texttt{None}, try the
      \texttt{app\_case} decision\_tree, replacing the first element of
      each vector with the two elements of the function and the argument
      its applied to; otherwise, don't modify the vectors, and use the
      \texttt{default} decision tree.
    \item
      \texttt{Swap\ i\ cont}: Swap the first element of the vector with
      the ith element, and keep going with \texttt{cont}
    \end{itemize}
  \item
    The inductive type:

\begin{verbatim}
Inductive decision_tree :=
| TryLeaf (k : nat) (onfailure : decision_tree)
| Failure
| Switch (icases : list (raw_pident * decision_tree))
         (app_case : option decision_tree)
         (default : decision_tree)
| Swap (i : nat) (cont : decision_tree).
\end{verbatim}
  \end{itemize}
\item
  Raw identifiers

  \begin{itemize}
  \tightlist
  \item
    Note that the type of \texttt{icases}, the list of identifier cases
    in the \texttt{Switch} constructor above, maps what we call a
    \texttt{raw\_pident} (``p'' for ``pattern'') to a decision tree. The
    rewriter is parameterized over a type of \texttt{raw\_pident}s,
    which is instantiated with a python-generated inductive type which
    names all of the identifiers we care about, except without any
    arguments. We call them ``raw'' because they are not type-indexed.
  \item
    An example where this is important: We want to be able to express a
    decision tree for the pattern \texttt{fst\ (x,\ y)}. This involves
    an application of the identifier \texttt{fst} to a pair. We want to
    be able to talk about ``\texttt{fst}, of any type'' in the decision
    tree, without needing to list out all of the possible type arguments
    to \texttt{fst}.
  \end{itemize}
\item
  Swap vs indices

  \begin{itemize}
  \tightlist
  \item
    One design decision we copy from \emph{Compiling Pattern Matching to
    Good Decision Trees} is to have a \texttt{Swap} case. We could
    instead augment each \texttt{Switch} with the index in the vector
    being examined. If we did this, we'd need to talk about splicing a
    new list into the middle of an existing list, which is harder than
    talking about swapping two indices of a list.
  \item
    Note that swapping is \emph{significantly} more painful over typed
    patterns and terms than over untyped ones. If we index our vectors
    over a list of types, then we need to swap the types, and later swap
    them back (when reconstructing the term for evaluation), and then we
    need to unswap the terms in a way that has unswap (swap ls) on the
    term level \emph{judgmentally} indexed on the type level over the
    same index-list as ls. This is painful, and is an example of pain
    caused by picking the wrong abstraction, in a way that causes
    exponential blow-up with each extra layer of dependency added.
  \end{itemize}
\item
  The type of patterns

  \begin{itemize}
  \item
    Patterns describe the LHS of rewrite rules, or the LHS of cases of a
    match statement. We index patterns over their type:

\begin{verbatim}
Inductive pattern.base.type := var (p : positive) | type_base (t : Compilers.base.type.base) | prod (A B : type) | list (A : type).
\end{verbatim}
  \item
    The type of a pattern is either an arrow or a pattern.base.type, and
    a pattern.base.type is either a positive-indexed type-variable
    (written '1, '2, \ldots), or a product, a list, or a standard
    base.type (with no type-variables)
  \item
    A pattern is either a wildcard, an identifier, or an application of
    patterns. Note that our rewriter only handles fully applied
    patterns, i.e., only things of type
    \texttt{pattern\ (type.base\ t)}, not things of type
    \texttt{pattern\ t}. (This is not actually true. The rewriter can
    kind-of handle non-fully-applied patterns, but the Gallina won't
    reduce in the right places, so we restrict ourselves to
    fully-applied patterns.)

\begin{verbatim}
Inductive pattern {ident : type -> Type} : type -> Type :=
| Wildcard (t : type) : pattern t
| Ident {t} (idc : ident t) : pattern t
| App {s d} (f : pattern (s -> d)) (x : pattern s) : pattern d.
\end{verbatim}
  \item
    Pattern matching \emph{compilation} to decision trees actually uses
    a more raw version of patterns, which come from these patterns:

\begin{verbatim}
Module Raw.
  Inductive pattern {ident : Type} :=
  | Wildcard
  | Ident (idc : ident)
  | App (f x : pattern).
End Raw.
\end{verbatim}

    \begin{itemize}
    \tightlist
    \item
      This is because the pattern matching compilation algorithm is
      morally done over untyped patterns and terms.
    \end{itemize}
  \end{itemize}
\item
  The definitions

  \begin{itemize}
  \tightlist
  \item
    TODO: How much detail should I include about intermediate things?
  \item
    Pattern matching compilation at the top-level, takes in a list of
    patterns, and spits out a decision tree. Note that each
    \texttt{TryLeaf} node in the decision tree has an index \texttt{k},
    which denotes the index in this initial list of patterns of the
    chosen rewrite rule.
  \item
    The workhorse of pattern matching compilation is
    \texttt{Fixpoint\ compile\_rewrites\textquotesingle{}\ (fuel\ :\ nat)\ (pattern\_matrix\ :\ list\ (nat\ *\ list\ rawpattern))\ :\ option\ decision\_tree}.
    This takes the list rows of the matrix of patterns, each one
    containing a list (vector in the original source paper) of patterns
    to match against, and the original index of the rewrite rule that
    this list of patterns came from. Note that all of these lists are in
    fact the same length, but we do not track this invariant anywhere,
    because it would add additional overhead for little-to-no gain.

    \begin{itemize}
    \tightlist
    \item
      The \texttt{compile\_rewrites\textquotesingle{}} procedure
      operates as follows:

      \begin{itemize}
      \item
        If we are out of fuel, then we fail (return \texttt{None})
      \item
        If the \texttt{pattern\_matrix} is empty, we indicate
        \texttt{Failure} to match
      \item
        If the first row is made up entirely of wildcards, we indicate
        to \texttt{TryLeaf} with the rewrite rule corresponding to the
        first row, and then to continue on with the decision tree
        corresponding to the rest of the rows.
      \item
        If the first element of the first row is a wildcard, we
        \texttt{Swap} the first element with the index \texttt{i} of the
        first non-wildcard pattern in the first row. We then swap the
        first element with the \texttt{i}th element in every row of the
        matrix, and continue on with the result of compiling that
        matrix.
      \item
        If the first element of the first row is not a wildcard, we
        issue a \texttt{Switch}. We first split the pattern matrix by
        finding the first row where the first element in that row is a
        wildcard, and aggregating that row and all rows after it into
        the \texttt{default\_pattern\_matrix}. We partition the rows
        before that row into the ones where the first element is an
        application node and the ones where the first element is an
        identifier node. The application nodes get split into the
        pattern for the function, and the pattern for the argument, and
        these two are prepended to the row. We group the rows that start
        with identifier patterns into groups according to the pattern
        identifier at the beginning of the row, and then take the tail
        of each of these rows. We then compile all of these decision
        trees to make up the Switch case.
      \item
        In code, this looks like:

\begin{verbatim}
Definition compile_rewrites_step
           (compile_rewrites : list (nat * list rawpattern) -> option decision_tree)
           (pattern_matrix : list (nat * list rawpattern))
  : option decision_tree
  := match pattern_matrix with
     | nil => Some Failure
     | (n1, p1) :: ps
       => match get_index_of_first_non_wildcard p1 with
         | None (* p1 is all wildcards *)
           => (onfailure <- compile_rewrites ps;
                Some (TryLeaf n1 onfailure))
         | Some Datatypes.O
           => let '(pattern_matrix, default_pattern_matrix) := split_at_first_pattern_wildcard pattern_matrix in
              default_case <- compile_rewrites default_pattern_matrix;
                app_case <- (if contains_pattern_app pattern_matrix
                             then option_map Some (compile_rewrites (Option.List.map filter_pattern_app pattern_matrix))
                             else Some None);
                let pidcs := get_unique_pattern_ident pattern_matrix in
                let icases := Option.List.map
                                (fun pidc => option_map (pair pidc) (compile_rewrites (Option.List.map (filter_pattern_pident pidc) pattern_matrix)))
                                pidcs in
                Some (Switch icases app_case default_case)
         | Some i
           => let pattern_matrix'
                 := List.map
                      (fun '(n, ps)
                       => (n,
                          match swap_list 0 i ps with
                          | Some ps' => ps'
                          | None => nil (* should be impossible *)
                          end))
                      pattern_matrix in
             d <- compile_rewrites pattern_matrix';
               Some (Swap i d)
         end
     end%option.
\end{verbatim}
      \end{itemize}
    \item
      We wrap \texttt{compile\_rewrites\textquotesingle{}} in a
      definition \texttt{compile\_rewrites} which extracts the
      well-typed patterns from a list of rewrite rules, associates them
      to indices, and strips the typing information off of the patterns
      to create raw (untyped) patterns. \#\#\# Decision Tree Evaluation
    \end{itemize}
  \end{itemize}
\item
  The next step in rewriting is to evaluate the decision tree to
  construct Gallina procedure that takes in an unknown (at
  rewrite-rule-compile-time) AST and performs the rewrite. This is
  broken up into two steps. The first step is to create the
  \texttt{match} structure that exposes all of the relevant information
  in the AST, and picks which rewrite rules to try in which order, and
  glues together failure and success of rewriting. The second step is to
  actually try to rewrite with a given rule, under the assumption that
  enough structure has been exposed.
\item
  Because we have multiple phases of compilation, we need to track which
  information we have (and therefore can perform reduction on) when we
  know only the patterns but not the AST being rewritten in, and which
  reductions have to wait until we know the AST. The way we track this
  information is by creating a wrapper type for ASTs. Note that the
  wrapper type is not indexed over type codes, because pattern matching
  compilation naturally operates over untyped terms, and adjusting it to
  work when indexed over a vector of types is painful.
\item
  The wrapper type, and revealing structure

  \begin{itemize}
  \item
    Because different rewrite rules require different amounts of
    structure, we want to feed in only as much structure as is required
    for a given rewrite rule. For example, if we have one rewrite rule
    that is looking at \texttt{(?\ +\ ?)\ +\ ?}, and another that is
    looking at \texttt{?\ +\ 0}, we want to feed in the argument to the
    top-level \texttt{+} into the second rewrite rule, not a reassembled
    version of all of the different things an expression might be after
    checking for \texttt{+}-like structure of the first argument. If we
    did not do this, every rewrite rule replacement pattern would end up
    being as complicated as the deepest rewrite rule being considered,
    and we expect this would incur performance overhead. TODO: Perf
    testing?
  \item
    Because we want our rewrite-rule-compilation to happen by reduction
    in Coq, we define many operations in CPS-style, so that we can
    carefully manage the exact judgmental structures of the discriminees
    of \texttt{match} statements.
  \item
    An \texttt{Inductive\ rawexpr\ :\ Type} is one of the following
    things:

\begin{verbatim}
Inductive rawexpr : Type :=
| rIdent (known : bool) {t} (idc : ident t) {t'} (alt : expr t')
| rApp (f x : rawexpr) {t} (alt : expr t)
| rExpr {t} (e : expr t)
| rValue {t} (e : value t).
\end{verbatim}

    \begin{itemize}
    \tightlist
    \item
      \texttt{rIdent\ known\ t\ idc\ t\textquotesingle{}\ alt} - an
      identifier \texttt{idc\ :\ ident\ t}, whose unrevealed structure
      is \texttt{alt\ :\ expr\ t\textquotesingle{}}. The boolean
      \texttt{known} indicates if the identifier is known to be simple
      enough that we can fully reduce matching on its type arguments
      during rewrite-rule-compilation-time. For example, if we know an
      identifier to be \texttt{Z.add} (perhaps because we have matched
      on it already), we can reduce equality tests against the type.
      However, if an identifier is \texttt{nil\ T}, we are not
      guaranteed to know the type of the list judgmentally, and so we do
      not want to reduce type-equality tests against the list. Note that
      type-equality tests and type-transports are introduced as the
      least-evil thing we could find to cross the broken abstraction
      barrier between the untyped terms of pattern matching compilation,
      and the typed PHOASTs that we are operating on.
    \item
      \texttt{rApp\ f\ x\ t\ alt} is the application of the
      \texttt{rawexpr} \texttt{f} to the \texttt{rawexpr} \texttt{x},
      whose unrevealed structure is \texttt{alt\ :\ expr\ t}.
    \item
      \texttt{rExpr\ t\ e} is a not-as-yet revealed expression
      \texttt{e\ :\ expr\ t}.
    \item
      \texttt{rValue\ t\ e} is an unrevealed value
      \texttt{e\ :\ value\ t}. Such NBE-style values may contain thunked
      computation, such as deferred rewriting opportunities. This is
      essential for fully evaluating rewriting in expressions such as
      \texttt{map\ (fun\ x\ =\textgreater{}\ x\ +\ x\ +\ 0)\ ls}, where
      you want to rewrite away the \texttt{map} (when \texttt{ls} is a
      concrete list of cons cells), the \texttt{+\ 0} (always), and the
      \texttt{x\ +\ x} whenever \texttt{x} is a literal (which you do
      not know until you have distributed the function over the list).
      Allowing thunked computation in the ASTs allows us to do all of
      this rewriting in a single pass.
    \end{itemize}
  \item
    Revealing structure:
    \texttt{Definition\ reveal\_rawexpr\_cps\ (e\ :\ rawexpr)\ :\ \textasciitilde{}\textgreater{}\ rawexpr}

    \begin{itemize}
    \item
      For the sake of proofs, we actually define a slightly more general
      version of revealing structure, which allows us to specify whether
      or not we have already matched against the putative identifier at
      the top-level of the \texttt{rawexpr}.
    \item
      The code:

\begin{verbatim}
Definition reveal_rawexpr_cps_gen (assume_known : option bool)
           (e : rawexpr) : ~> rawexpr
  := fun T k
     => match e, assume_known with
        | rExpr _ e as r, _
        | rValue (type.base _) e as r, _
          => match e with
             | expr.Ident t idc => k (rIdent (match assume_known with Some known => known | _ => false end) idc e)
             | expr.App s d f x => k (rApp (rExpr f) (rExpr x) e)
             | _ => k r
             end
        | rIdent _ t idc t' alt, Some known => k (rIdent known idc alt)
        | e', _ => k e'
        end.
\end{verbatim}
    \item
      To reveal a \texttt{rawexpr}, CPS-style, we first match on the
      \texttt{rawexpr}.

      \begin{itemize}
      \tightlist
      \item
        If it is an \texttt{rExpr}, or a \texttt{rValue} at a base type
        (and thus just an expression), we then match on the resulting
        expression.

        \begin{itemize}
        \tightlist
        \item
          If it is an identifier or an application node, we encode that,
          and then invoke the continuation
        \item
          Otherwise, we invoke the continuation with the existing
          \texttt{rExpr} or \texttt{rValue}, because there was no more
          accessible structure to reveal; we do not allow matching on
          lambdas syntactically.
        \end{itemize}
      \item
        If it is an identifier and we are hard-coding the \texttt{known}
        status about if matches on the type of the identifier can be
        reduced, then we re-assemble the \texttt{rIdent} node with the
        new \texttt{known} status and invoke the continuation.
      \item
        Otherwise, we just invoke the continuation on the reassembled
        \texttt{rawexpr}.
      \end{itemize}
    \item
      Correctness conditions

      \begin{itemize}
      \tightlist
      \item
        There are a couple of properties that must hold of
        \texttt{reveal\_rawexpr\_cps}.
      \item
        The first is a \texttt{cps\_id} rule, which says that applying
        \texttt{reveal\_rawexpr\_cps} to any continuation is the same as
        invoking the continuation with \texttt{reveal\_rawexpr\_cps}
        applied to the identity continuation.
      \item
        The next rule talks about a property we call
        \texttt{rawexpr\_types\_ok}. To say that this property holds is
        to say that the \texttt{rawexper}s are well-typed in accordance
        with the unrevealed expressions stored in the tree.

        \begin{itemize}
        \item
          Code:

\begin{verbatim}
Fixpoint rawexpr_types_ok (r : @rawexpr var) (t : type) : Prop
  := match r with
     | rExpr t' _
     | rValue t' _
       => t' = t
     | rIdent _ t1 _ t2 _
       => t1 = t /\ t2 = t
     | rApp f x t' alt
       => t' = t
          /\ match alt with
             | expr.App s d _ _
               => rawexpr_types_ok f (type.arrow s d)
                  /\ rawexpr_types_ok x s
             | _ => False
             end
     end.
\end{verbatim}
        \item
          We must then have that a \texttt{rawexpr} is
          \texttt{rawexpr\_types\_ok} if and only if the result of
          revealing one layer of structure via
          \texttt{reveal\_rawexpr\_cps} is \texttt{rawexpr\_types\_ok}.
        \end{itemize}
      \item
        We also define a relation \texttt{rawexpr\_equiv} which says
        that two \texttt{rawexpr}s represent the same expression, up to
        different amounts of revealed structure.

        \begin{itemize}
        \item
          Code:

\begin{verbatim}
Local Notation "e1 === e2" := (existT expr _ e1 = existT expr _ e2) : type_scope.

Fixpoint rawexpr_equiv_expr {t0} (e1 : expr t0) (r2 : rawexpr) {struct r2} : Prop
  := match r2 with
     | rIdent _ t idc t' alt
       => alt === e1 /\ expr.Ident idc === e1
     | rApp f x t alt
       => alt === e1
          /\ match e1 with
             | expr.App _ _ f' x'
               => rawexpr_equiv_expr f' f /\ rawexpr_equiv_expr x' x
             | _ => False
             end
     | rExpr t e
     | rValue (type.base t) e
       => e === e1
     | rValue t e => False
     end.

Fixpoint rawexpr_equiv (r1 r2 : rawexpr) : Prop
  := match r1, r2 with
     | rExpr t e, r
     | r, rExpr t e
     | rValue (type.base t) e, r
     | r, rValue (type.base t) e
       => rawexpr_equiv_expr e r
     | rValue t1 e1, rValue t2 e2
       => existT _ t1 e1 = existT _ t2 e2
     | rIdent _ t1 idc1 t'1 alt1, rIdent _ t2 idc2 t'2 alt2
       => alt1 === alt2
          /\ (existT ident _ idc1 = existT ident _ idc2)
     | rApp f1 x1 t1 alt1, rApp f2 x2 t2 alt2
       => alt1 === alt2
          /\ rawexpr_equiv f1 f2
          /\ rawexpr_equiv x1 x2
     | rValue _ _, _
     | rIdent _ _ _ _ _, _
     | rApp _ _ _ _, _
       => False
     end.
\end{verbatim}
        \item
          The relation \texttt{rawexpr\_equiv} is effectively the
          recursive closure of \texttt{reveal\_rawexpr\_cps}, and we
          must prove that \texttt{reveal\_rawexpr\ e} and \texttt{e} are
          \texttt{rawexpr\_equiv}. \textless!---
        \end{itemize}
      \item
        Finally, we define a notation of \texttt{wf} for
        \texttt{rawexpr}s called \texttt{wf\_rawexpr}, and we prove that
        if two \texttt{rawexpr}s are \texttt{wf\_rawexpr}-related, then
        the results of calling \texttt{reveal\_rawexpr} on both of them
        are \texttt{wf\_rawexpr}-related.

        \begin{itemize}
        \item
          The definition of \texttt{wf\_rawexpr} is:

\begin{verbatim}
Inductive wf_rawexpr : list { t : type & var1 t * var2 t }%type -> forall {t}, @rawexpr var1 -> @expr var1 t -> @rawexpr var2 -> @expr var2 t -> Prop :=
| Wf_rIdent {t} G known (idc : ident t) : wf_rawexpr G (rIdent known idc (expr.Ident idc)) (expr.Ident idc) (rIdent known idc (expr.Ident idc)) (expr.Ident idc)
| Wf_rApp {s d} G
          f1 (f1e : @expr var1 (s -> d)) x1 (x1e : @expr var1 s)
          f2 (f2e : @expr var2 (s -> d)) x2 (x2e : @expr var2 s)
  : wf_rawexpr G f1 f1e f2 f2e
    -> wf_rawexpr G x1 x1e x2 x2e
    -> wf_rawexpr G
                  (rApp f1 x1 (expr.App f1e x1e)) (expr.App f1e x1e)
                  (rApp f2 x2 (expr.App f2e x2e)) (expr.App f2e x2e)
| Wf_rExpr {t} G (e1 e2 : expr t)
  : expr.wf G e1 e2 -> wf_rawexpr G (rExpr e1) e1 (rExpr e2) e2
| Wf_rValue {t} G (v1 v2 : value t)
  : wf_value G v1 v2
    -> wf_rawexpr G (rValue v1) (reify v1) (rValue v2) (reify v2).
\end{verbatim}

          ---\textgreater{}
        \end{itemize}
      \end{itemize}
    \end{itemize}
  \end{itemize}
\item
  Evaluating the decision tree

  \begin{itemize}
  \item
    Decision tree evaluation is performed by a single monolithic
    recursive function:
    \texttt{Fixpoint\ eval\_decision\_tree\ \{T\}\ (ctx\ :\ list\ rawexpr)\ (d\ :\ decision\_tree)\ (cont\ :\ nat\ -\textgreater{}\ list\ rawexpr\ -\textgreater{}\ option\ T)\ \{struct\ d\}\ :\ option\ T}

\begin{verbatim}
Fixpoint eval_decision_tree {T} (ctx : list rawexpr) (d : decision_tree) (cont : nat -> list rawexpr -> option T) {struct d} : option T
  := match d with
     | TryLeaf k onfailure
       => let res := cont k ctx in
          match onfailure with
          | Failure => res
          | _ => res ;; (@eval_decision_tree T ctx onfailure cont)
          end
     | Failure => None
     | Switch icases app_case default_case
       => match ctx with
          | nil => None
          | ctx0 :: ctx'
            => let res
                   := reveal_rawexpr_cps
                        ctx0 _
                        (fun ctx0'
                         => match ctx0' with
                            | rIdent known t idc t' alt
                              => fold_right
                                   (fun '(pidc, icase) rest
                                    => let res
                                           := if known
                                              then
                                                (args <- invert_bind_args _ idc pidc;
                                                   @eval_decision_tree
                                                     T ctx' icase
                                                     (fun k ctx''
                                                      => cont k (rIdent
                                                                   (raw_pident_is_simple pidc)
                                                                   (raw_pident_to_typed pidc args) alt :: ctx'')))
                                              else
                                                @eval_decision_tree
                                                  T ctx' icase
                                                  (fun k ctx''
                                                   => option_bind'
                                                        (invert_bind_args_unknown _ idc pidc)
                                                        (fun args
                                                         => cont k (rIdent
                                                                      (raw_pident_is_simple pidc)
                                                                      (raw_pident_to_typed pidc args) alt :: ctx'')))
                                       in
                                       match rest with
                                       | None => Some res
                                       | Some rest => Some (res ;; rest)
                                       end)
                                   None
                                   icases;;;
                                   None
                            | rApp f x t alt
                              => match app_case with
                                 | Some app_case
                                   => @eval_decision_tree
                                        T (f :: x :: ctx') app_case
                                        (fun k ctx''
                                         => match ctx'' with
                                            | f' :: x' :: ctx'''
                                              => cont k (rApp f' x' alt :: ctx''')
                                            | _ => None
                                            end)
                                 | None => None
                                 end
                            | rExpr t e
                            | rValue t e
                              => None
                            end) in
               match default_case with
               | Failure => res
               | _ => res ;; (@eval_decision_tree T ctx default_case cont)
               end
          end
     | Swap i d'
       => match swap_list 0 i ctx with
          | Some ctx'
            => @eval_decision_tree
                 T ctx' d'
                 (fun k ctx''
                  => match swap_list 0 i ctx'' with
                     | Some ctx''' => cont k ctx'''
                     | None => None
                     end)
          | None => None
          end
     end%option.
\end{verbatim}
  \item
    This function takes a list (vector in the original source paper)
    \texttt{ctx} of \texttt{rawexpr}s to match against, a
    \texttt{decision\_tree} \texttt{d} to evaluate, and a
    ``continuation'' \texttt{cont} which tries a given rewrite rule
    based on the index of the rewrite rule (in the original list of
    rewrite rules) and the list of \texttt{rawexpr}s to feed into the
    rewrite rule. This continuation is threaded through the decision
    tree evaluation procedure, and each time we split up the structure
    of the pattern matrix (virtually, in the decision tree) and the list
    of \texttt{rawexpr}s (concretely, as an argument), we add a bit to
    the continuation that ``undoes'' the splitting. In the end, the
    top-level ``continuation'' gets fed a singleton list containing a
    \texttt{rawexpr} with enough structure for the rewrite rule it is
    trying. TODO: Figure out how to be more clear here; I anticipate
    this is unclear, and I'm not sure how to fix it except by randomly
    throwing more sentences in to try to explain it in various different
    ways.
  \item
    Correctness conditions

    \begin{itemize}
    \tightlist
    \item
      There are two correctness conditions for
      \texttt{eval\_decision\_tree}: one for \texttt{wf}, and the other
      for \texttt{Interp}.

      \begin{itemize}
      \tightlist
      \item
        The interpretation-correctness rule says that either
        \texttt{eval\_decision\_tree} returns \texttt{None}, or it
        returns the result of calling the continuation on some index and
        with some list of \texttt{rawexpr}s which is element-wise
        \texttt{rawexpr\_equiv} to the input list. In other words,
        \texttt{eval\_decision\_tree} does nothing more than revealing
        some structure, and then eventually calling the continuation
        (which is to be filled in with ``rewrite with this rule'') on
        the revealed \texttt{rawexpr}.
      \end{itemize}
    \item
      The \texttt{wf} correctness condition is significantly more
      verbose to state, but it boils down to saying that as long as the
      continuation behaves ``the same'' (for some parameterized notion
      of ``the same'') on \texttt{wf\_rawexpr}-related
      \texttt{rawexpr}s, then \texttt{eval\_decision\_tree} will
      similarly behave ``the same'' on element-wise
      \texttt{wf\_rawexpr}-related lists of \texttt{rawexpr}s.
    \end{itemize}
  \item
    Definition

    \begin{itemize}
    \tightlist
    \item
      The \texttt{eval\_decision\_tree} procedure proceeds recursively
      on the structure of the \texttt{decision\_tree}.

      \begin{itemize}
      \tightlist
      \item
        If the decision tree is a \texttt{TryLeaf\ k\ onfailure}, then
        we try the continuation on the \texttt{k}th rewrite rule. If it
        fails (by returning \texttt{None}), we proceed with
        \texttt{onfailure}. In the code, there is a bit of extra care
        taken to simplify the resulting output code when
        \texttt{onfailure} is just \texttt{Failure}, i.e., no remaining
        matches to try. This probably does not impact performance, but
        it makes the output of the rewrite-rule-compilation procedure
        slightly easier to read and debug.
      \item
        If the decision tree is \texttt{Failure} return \texttt{None},
        i.e., we did not succeed in rewriting.
      \item
        If the decision tree starts with
        \texttt{Swap\ i\ d\textquotesingle{}}, we swap the first element
        with the \texttt{i}th element in the list of \texttt{rawexpr}s
        we are matching on (to mirror the swapping in the pattern matrix
        that happened when compiling the decision tree), and then
        continue on evaluating \texttt{d\textquotesingle{}}. We augment
        the continuation by reversing the swap in the list of
        \texttt{rawexpr}s passed in at the beginning, to cancel out the
        swap we did ``on the outside'' before continuing with evaluation
        of the decision tree. Note that here we are jumping through some
        extra hoops to get the right reduction behavior at
        rewrite-rule-compilation time.
      \item
        If none of the above match, the decision tree must begin with
        \texttt{Switch\ icases\ app\_case\ default\_case}. In this case,
        we start by revealing the structure of the first element of the
        list of \texttt{rawexpr}s. (If there is no first element, which
        should never happen, we indicate failure by returning
        \texttt{None}.) In the continuation of
        \texttt{reveal\_rawexpr\_cps}, we check which sort of
        \texttt{rawexpr} we have. Note that in all cases of failure
        below, we try again with the \texttt{default\_case}.

        \begin{itemize}
        \tightlist
        \item
          If we have no accessible structure (\texttt{rExpr} or
          \texttt{rValue}), then we fail with \texttt{None}.
        \item
          If we have an application, we take the two arguments of
          \texttt{rApp}, the function and its argument, and prepend them
          to the tail of the list of \texttt{rawexpr}s. We then continue
          evaluation with \texttt{app\_case} (if it is
          non-\texttt{None}), and, in the continuation, we reassemble
          the \texttt{rawexpr} by taking the first two elements of the
          passed-in-list, and combining them in a new \texttt{rApp}
          node. We keep the unrevealed structure in \texttt{alt} the
          same as it was in the \texttt{rApp} that we started with.
        \item
          If we have an identifier, then we look at \texttt{icases}. We
          fold through the list of identifiers, looking to see if any of
          them match the identifier that we have. If the identifier is
          \texttt{known}, then we perform the match before evaluating
          the corresponding decision tree, because we want to avoid
          revealing useless structure. If the identifier is not
          \texttt{known}, then first we reveal all of the necessary
          structure for this identifier by continuing decision tree
          evaluation, and only then in the continuation do we try to
          match against the identifier.

          \begin{itemize}
          \tightlist
          \item
            In both cases, we drop the first element of the list of
            \texttt{rawexpr}s being matched against when continuing
            evaluation, to mirror the dropping that happens in
            compilation of the decision tree. We then prepend a re-built
            identifier onto the head of the list inside the
            continuation. We have a table of which pattern identifiers
            have \texttt{known} types, and we have conversion functions
            between pattern identifiers and PHOAST identifiers
            (autogenerated in Python) which allow us to extract the
            arguments from the PHOAST identifier and re-insert them into
            the pattern identifier. For example, this will extract the
            list type from \texttt{nil} (because the pattern-identifier
            version does not specify what the type of the list is---we
            will say more about this in the next section), or the
            literal value from the \texttt{Literal} identifier, and
            allow recreating the fully-typed identifier from the
            pattern-identifier with these arguments. This allows more
            rewriter-compile-time reduction opportunities which allows
            us to deduplicate matches against the same identifier. Note
            that we have two different constants that we use for binding
            these arguments; they do the same thing, but one is reduced
            away completely at rewrite-rule-compilation time, and the
            other is preserved. \#\#\# Rewriting with a particular
            rewrite rule
          \end{itemize}
        \end{itemize}
      \end{itemize}
    \end{itemize}
  \end{itemize}
\item
  The final big piece of the rewriter is to rewrite with a particular
  rule, given a \texttt{rawexpr} with enough revealed structure, a
  \texttt{pattern} against which we bind arguments, and a replacement
  rule which is a function indexed over the \texttt{pattern}. We saw
  above the inductive type of patterns. Let us now discuss the structure
  of the replacement rules.
\item
  Replacement rule types

  \begin{itemize}
  \item
    The data for a replacement rule is indexed over a pattern-type
    \texttt{t} and a \texttt{p\ :\ pattern\ t}. It has three options, in
    addition to the actual replacement rule:

\begin{verbatim}
Record rewrite_rule_data {t} {p : pattern t} :=
  { rew_should_do_again : bool;
    rew_with_opt : bool;
    rew_under_lets : bool;
    rew_replacement : @with_unif_rewrite_ruleTP_gen value t p rew_should_do_again rew_with_opt rew_under_lets }.
\end{verbatim}

    \begin{itemize}
    \tightlist
    \item
      \texttt{rew\_should\_do\_again} determines whether or not to
      rewrite again in the output of this rewrite rule. For example, the
      rewrite rule for \texttt{flat\_map} on a concrete list of cons
      cells maps the function over the list, and joins the resulting
      list of lists with append. We want to rewrite again with the rule
      for \texttt{List.app} in the output of this replacement.
    \item
      \texttt{rew\_with\_opt} determines whether or not the rewrite rule
      might fail. For example, rewrite rules like
      \texttt{x\ +\ 0\ \textasciitilde{}\textgreater{}\ x} are encoded
      by talking about the pattern of a wildcard added to a literal, and
      say that the rewrite only succeeds if the literal is 0.
      Additionally, as another example, all rewrite rules involving
      casts fail if bounds on the input do not line up; in the pattern
      \texttt{Z.cast\ @\ ((Z.cast\ @\ ??)\ +\ (Z.cast\ @\ ??))} the cast
      node in front of an addition must be loose enough to hold the sum
      of the ranges taken from the two cast nodes in front of each of
      the wildcards.
    \item
      \texttt{rew\_under\_lets} determines whether or not the
      replacement rule returns an explicit \texttt{UnderLets} structure.
      This can be used for let-binding a part of the replacement value.
    \end{itemize}
  \item
    The rewrite rule replacement itself is a function. It takes in first
    all type variables which are mentioned in the pattern, and then, in
    an in-order traversal of the pattern syntax tree, the non-type
    arguments for each identifier (e.g., interpreted values of literals,
    ranges of cast nodes) and a
    \texttt{value\ (pattern.type.subst\_default\ t\ evm)} for each
    wildcard of type \texttt{t} (that is, we plug in the known type
    variables into the pattern-type, and use \texttt{unit} for any
    unknown type variables). It may return a thing in the
    \texttt{option} and/or \texttt{UnderLets} monads, depending on
    \texttt{rew\_with\_opt} and \texttt{rew\_under\_lets}. Underneath
    these possible monads, it returns an \texttt{expr} of the correct
    type (we substitute the type variables we take in into the type of
    the pattern), whose \texttt{var} type is either \texttt{@value\ var}
    (if \texttt{rew\_should\_do\_again}) or \texttt{var} (if not
    \texttt{rew\_should\_do\_again}). The different \texttt{var} types
    are primarily to make the type of the output of the rewrite rule
    line up with the expression type that is fed into the rewriter as a
    whole. We have a number of definitions that describe this in a
    dependently typed mess:

    \begin{itemize}
    \item
      We aggregate the type variables into a \texttt{PositiveSet.t} with
      \texttt{Fixpoint\ pattern.base.collect\_vars\ (t\ :\ base.type)\ :\ PositiveSet.t}
      and
      \texttt{Fixpoint\ pattern.type.collect\_vars\ (t\ :\ type)\ :\ PositiveSet.t}:

\begin{verbatim}
Module base.
  Fixpoint collect_vars (t : type) : PositiveSet.t
    := match t with
       | type.var p => PositiveSet.add p PositiveSet.empty
       | type.type_base t => PositiveSet.empty
       | type.prod A B => PositiveSet.union (collect_vars A) (collect_vars B)
       | type.list A => collect_vars A
       end.
End base.
Module type.
    Fixpoint collect_vars (t : type) : PositiveSet.t
      := match t with
         | type.base t => base.collect_vars t
         | type.arrow s d => PositiveSet.union (collect_vars s) (collect_vars d)
         end.
End type.
\end{verbatim}
    \item
      We quantify over type variables for each of the numbers in the
      \texttt{PositiveSet.t} and aggregate the bound types into a
      \texttt{PositiveMap.t} with \texttt{pattern.type.forall\_vars}.
      Note that we use the possibly ill-chosen abbreviation
      \texttt{EvarMap} for \texttt{PositiveMap.t\ Compilers.base.type}.

\begin{verbatim}
Local Notation forall_vars_body K LS EVM0
  := (fold_right
        (fun i k evm => forall t : Compilers.base.type, k (PositiveMap.add i t evm))
        K
        LS
        EVM0).

Definition forall_vars (p : PositiveSet.t) (k : EvarMap -> Type)
  := forall_vars_body k (List.rev (PositiveSet.elements p)) (PositiveMap.empty _).
\end{verbatim}
    \item
      We take in the context variable
      \texttt{pident\_arg\_types\ :\ forall\ t,\ pident\ t\ -\textgreater{}\ list\ Type}
      which describes the arguments bound for a given pattern
      identifier.
    \item
      We then quantify over identifier arguments and wildcard values
      with \texttt{with\_unification\_resultT}:

\begin{verbatim}
Local Notation type_of_list_cps
  := (fold_right (fun a K => a -> K)).

Fixpoint with_unification_resultT' {var} {t} (p : pattern t) (evm : EvarMap) (K : Type) : Type
  := match p return Type with
     | pattern.Wildcard t => var (pattern.type.subst_default t evm) -> K
     | pattern.Ident t idc => type_of_list_cps K (pident_arg_types t idc)
     | pattern.App s d f x
       => @with_unification_resultT' var _ f evm (@with_unification_resultT' var _ x evm K)
     end%type.

Definition with_unification_resultT {var t} (p : pattern t) (K : type -> Type) : Type
  := pattern.type.forall_vars
       (@pattern.collect_vars _ t p)
       (fun evm => @with_unification_resultT' var t p evm (K (pattern.type.subst_default t evm))).
\end{verbatim}
    \item
      Finally, we can define the type of rewrite replacement rules:

\begin{verbatim}
Local Notation deep_rewrite_ruleTP_gen' should_do_again with_opt under_lets t
  := (match (@expr.expr base.type ident (if should_do_again then value else var) t) with
      | x0 => match (if under_lets then UnderLets x0 else x0) with
              | x1 => if with_opt then option x1 else x1
              end
      end).

Definition deep_rewrite_ruleTP_gen (should_do_again : bool) (with_opt : bool) (under_lets : bool) t
  := deep_rewrite_ruleTP_gen' should_do_again with_opt under_lets t.

Definition with_unif_rewrite_ruleTP_gen {var t} (p : pattern t) (should_do_again : bool) (with_opt : bool) (under_lets : bool)
  := @with_unification_resultT var t p (fun t => deep_rewrite_ruleTP_gen' should_do_again with_opt under_lets t).
\end{verbatim}

      Whence we have

\begin{verbatim}
rew_replacement : @with_unif_rewrite_ruleTP_gen value t p rew_should_do_again rew_with_opt rew_under_lets
\end{verbatim}
    \end{itemize}
  \end{itemize}
\item
  There are two steps to rewriting with a rule, both conceptually simple
  but in practice complicated by dependent types. We must unify a
  pattern with an expression, gathering binding data for the replacement
  rule as we go; and we must apply the replacement rule to the binding
  data (which is non-trivial because the rewrite rules are expressed as
  curried dependently-typed towers indexed over the rewrite rule
  pattern). In order to state the correctness conditions for gathering
  binding data, we must first talk about applying replacement rules to
  binding data.
\item
  Applying binding data

  \begin{itemize}
  \tightlist
  \item
    The general strategy for applying binding data is to define an
    uncurried package (sigma type, or dependent pair) holding all of the
    arguments, and to define an application function that applies the
    replacement rule (at various stages of construction) to the binding
    data package.
  \item
    The uncurried package types

    \begin{itemize}
    \item
      To turn a list of Types into a Type, we define
      \texttt{Local\ Notation\ type\_of\_list\ :=\ (fold\_right\ (fun\ a\ b\ =\textgreater{}\ prod\ a\ b)\ unit)}.
    \item
      The type \texttt{unification\_resultT\textquotesingle{}} describes
      the binding data for a pattern, given a map of pattern type
      variables to types:

\begin{verbatim}
Fixpoint unification_resultT' {var} {t} (p : pattern t) (evm : EvarMap) : Type
  := match p return Type with
     | pattern.Wildcard t => var (pattern.type.subst_default t evm)
     | pattern.Ident t idc => type_of_list (pident_arg_types t idc)
     | pattern.App s d f x
       => @unification_resultT' var _ f evm * @unification_resultT' var _ x evm
     end%type.
\end{verbatim}
    \item
      A \texttt{unification\_resultT} packages up the type variable
      replacement map with the bound values:

\begin{verbatim}
Definition unification_resultT {var t} (p : pattern t) : Type
  := { evm : EvarMap & @unification_resultT' var t p evm }.
\end{verbatim}
    \end{itemize}
  \item
    The application functions

    \begin{itemize}
    \item
      The definition \texttt{app\_type\_of\_list} applies a CPS-type
      \texttt{type\_of\_list\_cps} function to uncurried arguments:

\begin{verbatim}
Definition app_type_of_list {K} {ls : list Type} (f : type_of_list_cps K ls) (args : type_of_list ls) : K
  := list_rect
       (fun ls
        => type_of_list_cps K ls -> type_of_list ls -> K)
       (fun v _ => v)
       (fun T Ts rec f x
        => rec (f (fst x)) (snd x))
       ls
       f args.
\end{verbatim}
    \item
      Given two different maps of type variables (another instance of
      abstraction-barrier-breaking), we can apply a
      \texttt{with\_unification\_resultT\textquotesingle{}} to a
      \texttt{unification\_resultT\textquotesingle{}} by inserting casts
      in the appropriate places:

\begin{verbatim}
(** TODO: Maybe have a fancier version of this that doesn't
     actually need to insert casts, by doing a fixpoint on the
     list of elements / the evar map *)
Fixpoint app_transport_with_unification_resultT'_cps {var t p evm1 evm2 K} {struct p}
  : @with_unification_resultT' var t p evm1 K -> @unification_resultT' var t p evm2 -> forall T, (K -> option T) -> option T
  := fun f x T k
     => match p return with_unification_resultT' p evm1 K -> unification_resultT' p evm2 -> option T with
        | pattern.Wildcard t
          => fun f x
             => (tr <- type.try_make_transport_cps base.try_make_transport_cps var _ _;
                   (tr <- tr;
                      k (f (tr x)))%option)%cps
     | pattern.Ident t idc => fun f x => k (app_type_of_list f x)
     | pattern.App s d f x
       => fun F (xy : unification_resultT' f _ * unification_resultT' x _)
          => @app_transport_with_unification_resultT'_cps
               _ _ f _ _ _ F (fst xy) T
               (fun F'
                => @app_transport_with_unification_resultT'_cps
                     _ _ x _ _ _ F' (snd xy) T
                     (fun x' => k x'))
     end%option f x.
\end{verbatim}
    \item
      We can apply a \texttt{forall\_vars} tower over the type variables
      in a pattern to a particular mapping of type variables to types,
      with a headache of dependently typed code:

\begin{verbatim}
Fixpoint app_forall_vars_gen {k : EvarMap -> Type}
           (evm : EvarMap)
           (ls : list PositiveMap.key)
  : forall evm0, forall_vars_body k ls evm0
                 -> option (k (fold_right (fun i k evm'
                                           => k (match PositiveMap.find i evm with Some v => PositiveMap.add i v evm' | None => evm' end))
                                          (fun evm => evm)
                                          ls
                                          evm0))
  := match ls return forall evm0, forall_vars_body k ls evm0
                                  -> option (k (fold_right (fun i k evm'
                                                            => k (match PositiveMap.find i evm with Some v => PositiveMap.add i v evm' | None => evm' end))
                                                           (fun evm => evm)
                                                           ls
                                                           evm0)) with
     | nil => fun evm0 val => Some val
     | cons x xs
       => match PositiveMap.find x evm as xt
                return (forall evm0,
                           (forall t, fold_right _ k xs (PositiveMap.add x t evm0))
                           -> option (k (fold_right
                                           _ _ xs
                                           match xt with
                                           | Some v => PositiveMap.add x v evm0
                                           | None => evm0
                                           end)))
          with
          | Some v => fun evm0 val => @app_forall_vars_gen k evm xs _ (val v)
          | None => fun evm0 val => None
          end
     end.

Definition app_forall_vars {p : PositiveSet.t} {k : EvarMap -> Type}
           (f : forall_vars p k)
           (evm : EvarMap)
  : option (k (fold_right (fun i k evm'
                           => k (match PositiveMap.find i evm with Some v => PositiveMap.add i v evm' | None => evm' end))
                          (fun evm => evm)
                          (List.rev (PositiveSet.elements p))
                          (PositiveMap.empty _)))
  := @app_forall_vars_gen
       k evm
       (List.rev (PositiveSet.elements p))
       (PositiveMap.empty _)
       f.
\end{verbatim}
    \item
      Finally, we can apply a \texttt{with\_unification\_resultT} to a
      \texttt{unification\_resultT} package in the obvious way,
      inserting casts as needed:

\begin{verbatim}
Definition app_with_unification_resultT_cps {var t p K}
  : @with_unification_resultT var t p K -> @unification_resultT var t p -> forall T, ({ evm' : _ & K (pattern.type.subst_default t evm') } -> option T) -> option T
  := fun f x T k
     => (f' <- pattern.type.app_forall_vars f (projT1 x);
           app_transport_with_unification_resultT'_cps
             f' (projT2 x) _
             (fun fx
              => k (existT _ _ fx)))%option.
\end{verbatim}
    \end{itemize}
  \end{itemize}
\item
  Unifying patterns with expressions

  \begin{itemize}
  \tightlist
  \item
    First, we unify the types, in continuation-passing-style, returning
    an optional \texttt{PositiveMap.t} from type variable indices to
    types.

    \begin{itemize}
    \tightlist
    \item
      This is actually done in two steps, so that
      rewrite-rule-compilation can reduce away all occurrences of
      patterns. First, we check that the expression has the right
      structure, and extract all of the relevant types both from the
      pattern and from the expression. Then we connect the types with
      \texttt{type.arrow} (used simply for convenience, so we don't have
      to unify lists of types, only individual types), and we unify the
      two resulting types, extracting a \texttt{PositiveMap.t}
      describing the assignments resulting from the unification problem.

      \begin{itemize}
      \item
        We first define a few helper definitions that should be
        self-explanatory:

        \begin{itemize}
        \item
          The function \texttt{type\_of\_rawexpr} gets the type of a
          \texttt{rawexpr}:

\begin{verbatim}
Definition type_of_rawexpr (e : rawexpr) : type
  := match e with
     | rIdent _ t idc t' alt => t'
     | rApp f x t alt => t
     | rExpr t e => t
     | rValue t e => t
     end.
\end{verbatim}
        \item
          The functions \texttt{pattern.base.relax} and
          \texttt{pattern.type.relax} take a PHOAST type and turn it
          into a pattern type, which just happens to have no pattern
          type variables.

\begin{verbatim}
Module base.
  Fixpoint relax (t : Compilers.base.type) : type
    := match t with
       | Compilers.base.type.type_base t => type.type_base t
       | Compilers.base.type.prod A B => type.prod (relax A) (relax B)
       | Compilers.base.type.list A => type.list (relax A)
       end.
End base.
Module type.
  Fixpoint relax (t : type.type Compilers.base.type) : type
    := match t with
       | type.base t => type.base (base.relax t)
       | type.arrow s d => type.arrow (relax s) (relax d)
       end.
End type.
\end{verbatim}
        \end{itemize}
      \item
        The function responsible for checking the structure of patterns
        and extracting the types to be unified is
        \texttt{preunify\_types\ \{t\}\ (e\ :\ rawexpr)\ (p\ :\ pattern\ t)\ :\ \ option\ (option\ (ptype\ *\ type))}.
        It will return \texttt{None} if the structure does not match,
        \texttt{Some\ None} if the type of an identifier of known type
        in the \texttt{rawexpr} does not match the type of the
        identifier in the pattern (which is guaranteed to always be
        known, and thus this comparison is safe to perform at
        rewriter-rule-compilation time), and will return
        \texttt{Some\ (Some\ (t1,\ t2))} if the structures match, where
        \texttt{t1} and \texttt{t2} are the types to be unified.

\begin{verbatim}
Fixpoint preunify_types {t} (e : rawexpr) (p : pattern t) {struct p}
  : option (option (ptype * type))
  := match p, e with
     | pattern.Wildcard t, _
       => Some (Some (t, type_of_rawexpr e))
     | pattern.Ident pt pidc, rIdent known t idc _ _
       => if andb known (type.type_beq _ pattern.base.type.type_beq pt (pattern.type.relax t)) (* relies on evaluating to `false` if `known` is `false` *)
          then Some None
          else Some (Some (pt, t))
     | pattern.App s d pf px, rApp f x _ _
       => (resa <- @preunify_types _ f pf;
             resb <- @preunify_types _ x px;
             Some match resa, resb with
                  | None, None => None
                  | None, Some t
                  | Some t, None => Some t
                  | Some (a, a'), Some (b, b')
                    => Some (type.arrow a b, type.arrow a' b')
                  end)
     | pattern.Ident _ _, _
     | pattern.App _ _ _ _, _
       => None
     end%option.
\end{verbatim}
      \item
        We have two correctness conditions on \texttt{preunify\_types}.

        \begin{itemize}
        \item
          The \texttt{wf} correctness condition says that if two
          \texttt{rawexpr}s are \texttt{wf\_rawexpr}-related, then the
          result of pre-unifying one of them with a pattern \texttt{p}
          is the same as the result of pre-unifying the other with the
          same pattern \texttt{p}.
        \item
          Second, for interpretation-correctness, we define a recursive
          proposition encoding the well-matching of patterns with
          \texttt{rawexpr}s under a given map of pattern type variables
          to types:

\begin{verbatim}
Fixpoint types_match_with (evm : EvarMap) {t} (e : rawexpr) (p : pattern t) {struct p} : Prop
  := match p, e with
     | pattern.Wildcard t, e
       => pattern.type.subst t evm = Some (type_of_rawexpr e)
     | pattern.Ident t idc, rIdent known t' _ _ _
       => pattern.type.subst t evm = Some t'
     | pattern.App s d f x, rApp f' x' _ _
       => @types_match_with evm _ f' f
          /\ @types_match_with evm _ x' x
     | pattern.Ident _ _, _
     | pattern.App _ _ _ _, _
       => False
     end.
\end{verbatim}
        \item
          Then we prove that for any map \texttt{evm} of pattern type
          variables to types, if \texttt{preunify\_types\ re\ p} returns
          \texttt{Some\ (Some\ (pt,\ t\textquotesingle{}))}, and the
          result of substituting into \texttt{pt} the pattern type
          variables in the given map is \texttt{t\textquotesingle{}},
          then \texttt{types\_match\_with\ evm\ re\ p} holds.
          Symbolically, this is

\begin{verbatim}
Lemma preunify_types_to_match_with {t re p evm}
  : match @preunify_types ident var pident t re p with
    | Some None => True
    | Some (Some (pt, t')) => pattern.type.subst pt evm = Some t'
    | None => False
    end
    -> types_match_with evm re p.
\end{verbatim}
        \end{itemize}
      \item
        In a possibly-gratuitous use of dependent typing to ensure that
        no uses of \texttt{PositiveMap.t} remain after
        rewrite-rule-compilation, we define a dependently typed data
        structure indexed over the pattern type which holds the mapping
        of each pattern type variable to a corresponding type. This step
        cannot be fully reduced at rewrite-rule-compilation time,
        because we may not know enough type structure in the
        \texttt{rawexpr}. We then collect these variables into a
        \texttt{PositiveMap.t}; this step \emph{can} be fully reduced at
        rewrite-rule-compilation time, because the pattern always has a
        well-defined type structure, and so we know \emph{which} type
        variables will have assignments in the \texttt{PositiveMap.t},
        even if we don't necessarily know concretely (at
        rewrite-rule-compilation time) \emph{what} those type variables
        will be assigned to. We must also add a final check that
        substituting into the pattern type according the resulting
        \texttt{PositiveMap.t} actually does give the expected type; we
        do not want
        \texttt{\textquotesingle{}1\ -\textgreater{}\ \textquotesingle{}1}
        and \texttt{nat\ -\textgreater{}\ bool} to unify. We could check
        at each addition to the \texttt{PositiveMap.t} that we are not
        replacing one type with a different type. However, the proofs
        are much simpler if we simply do a wholesale check at the very
        end. We eventually perform this check in \texttt{unify\_types}.

        \begin{itemize}
        \item
          We thus define the dependently typed structures:

\begin{verbatim}
Module base.
  Fixpoint var_types_of (t : type) : Set
    := match t with
       | type.var _ => Compilers.base.type
       | type.type_base _ => unit
       | type.prod A B => var_types_of A * var_types_of B
       | type.list A => var_types_of A
       end%type.

  Fixpoint add_var_types_cps {t : type} (v : var_types_of t) (evm : EvarMap) : ~> EvarMap
    := fun T k
       => match t return var_types_of t -> T with
          | type.var p
            => fun t => k (PositiveMap.add p t evm)
          | type.prod A B
            => fun '(a, b) => @add_var_types_cps A a evm _ (fun evm => @add_var_types_cps B b evm _ k)
          | type.list A => fun t => @add_var_types_cps A t evm _ k
          | type.type_base _ => fun _ => k evm
          end v.
End base.
Module type.
  Fixpoint var_types_of (t : type) : Set
    := match t with
       | type.base t => base.var_types_of t
       | type.arrow s d => var_types_of s * var_types_of d
       end%type.

  Fixpoint add_var_types_cps {t : type} (v : var_types_of t) (evm : EvarMap) : ~> EvarMap
    := fun T k
       => match t return var_types_of t -> T with
          | type.base t => fun v => @base.add_var_types_cps t v evm _ k
          | type.arrow A B
            => fun '(a, b) => @add_var_types_cps A a evm _ (fun evm => @add_var_types_cps B b evm _ k)
          end v.
End type.
\end{verbatim}
        \item
          We can now write down the unifier that produces
          \texttt{var\_types\_of} from a unification problem; it is
          straightforward:

\begin{verbatim}
Module base.
  Fixpoint unify_extracted
           (ptype : type) (etype : Compilers.base.type)
    : option (var_types_of ptype)
    := match ptype, etype return option (var_types_of ptype) with
       | type.var p, _ => Some etype
       | type.type_base t, Compilers.base.type.type_base t'
         => if base.type.base_beq t t'
            then Some tt
            else None
       | type.prod A B, Compilers.base.type.prod A' B'
         => a <- unify_extracted A A';
              b <- unify_extracted B B';
              Some (a, b)
       | type.list A, Compilers.base.type.list A'
         => unify_extracted A A'
       | type.type_base _, _
       | type.prod _ _, _
       | type.list _, _
         => None
       end%option.
End base.
Module type.
  Fixpoint unify_extracted
           (ptype : type) (etype : type.type Compilers.base.type)
    : option (var_types_of ptype)
    := match ptype, etype return option (var_types_of ptype) with
       | type.base t, type.base t'
         => base.unify_extracted t t'
       | type.arrow A B, type.arrow A' B'
         => a <- unify_extracted A A';
              b <- unify_extracted B B';
              Some (a, b)
       | type.base _, _
       | type.arrow _ _, _
         => None
       end%option.
End type.
\end{verbatim}
        \end{itemize}
      \item
        Finally, we can write down the type-unifier for patterns and
        \texttt{rawexpr}s. Note that the final equality check, described
        and motivated above, is performed in this function.

\begin{verbatim}
(* for unfolding help *)
Definition option_type_type_beq := option_beq (type.type_beq _ base.type.type_beq).

Definition unify_types {t} (e : rawexpr) (p : pattern t) : ~> option EvarMap
  := fun T k
     => match preunify_types e p with
        | Some (Some (pt, t))
          => match pattern.type.unify_extracted pt t with
             | Some vars
               => pattern.type.add_var_types_cps
                    vars (PositiveMap.empty _) _
                    (fun evm
                     => (* there might be multiple type variables which map to incompatible types; we check for that here *)
                       if option_type_type_beq (pattern.type.subst pt evm) (Some t)
                       then k (Some evm)
                       else k None)
             | None => k None
             end
        | Some None
          => k (Some (PositiveMap.empty _))
        | None => k None
        end.
\end{verbatim}
      \end{itemize}
    \end{itemize}
  \item
    Now that we have unified the types and gotten a
    \texttt{PositiveMap.t} of pattern type variables to types, we are
    ready to unify the patterns, and extract the identifier arguments
    and \texttt{value}s from the \texttt{rawexpr}. Because it would be
    entirely too painful to track at the type-level that the type
    unifier guarantees a match on structure and types, we instead
    sprinkle type transports all over this definition to get the types
    to line up. Here we pay the price of an imperfect abstraction
    barrier (that we have types lying around, and we rely in some places
    on types lining up, but do not track everywhere that types line up).
    Most of the other complications in this function come from (a)
    working in continuation-passing-style (for getting the right
    reduction behavior) or (b) tracking the differences between things
    we can reduce at rewrite-rule-compilation time, and things we can't.

    \begin{itemize}
    \item
      We first describe some helper definitions and context variables.

      \begin{itemize}
      \item
        The context variable
        \texttt{pident\_arg\_types\ :\ forall\ t,\ pident\ t\ -\textgreater{}\ list\ Type}
        describes for each pattern identifier what arguments should be
        bound for it.
      \item
        The context variables
        \texttt{(pident\_unify\ pident\_unify\_unknown\ :\ forall\ t\ t\textquotesingle{}\ (idc\ :\ pident\ t)\ (idc\textquotesingle{}\ :\ ident\ t\textquotesingle{}),\ option\ (type\_of\_list\ (pident\_arg\_types\ t\ idc)))}
        are the to-be-unfolded and not-to-be-unfolded versions of
        unifying a pattern identifier with a PHOAST identifier.
      \item
        We can convert a \texttt{rawexpr} into a \texttt{value} or an
        \texttt{expr}:

\begin{verbatim}
Definition expr_of_rawexpr (e : rawexpr) : expr (type_of_rawexpr e)
  := match e with
     | rIdent _ t idc t' alt => alt
     | rApp f x t alt => alt
     | rExpr t e => e
     | rValue t e => reify e
     end.
Definition value_of_rawexpr (e : rawexpr) : value (type_of_rawexpr e)
  := Eval cbv `expr_of_rawexpr` in
      match e with
      | rValue t e => e
      | e => reflect (expr_of_rawexpr e)
      end.
\end{verbatim}
      \end{itemize}
    \item
      We can now write down the pattern-expression-unifier:

\begin{verbatim}
Definition option_bind' {A B} := @Option.bind A B. (* for help with unfolding *)

Fixpoint unify_pattern' {t} (e : rawexpr) (p : pattern t) (evm : EvarMap) {struct p}
  : forall T, (unification_resultT' p evm -> option T) -> option T
  := match p, e return forall T, (unification_resultT' p evm -> option T) -> option T with
     | pattern.Wildcard t', _
       => fun T k
          => (tro <- type.try_make_transport_cps (@base.try_make_transport_cps) value (type_of_rawexpr e) (pattern.type.subst_default t' evm);
                (tr <- tro;
                   _ <- pattern.type.subst t' evm; (* ensure that we did not fall into the default case *)
                   (k (tr (value_of_rawexpr e))))%option)%cps
     | pattern.Ident t pidc, rIdent known _ idc _ _
       => fun T k
          => (if known
              then Option.bind (pident_unify _ _ pidc idc)
              else option_bind' (pident_unify_unknown _ _ pidc idc))
               k
     | pattern.App s d pf px, rApp f x _ _
       => fun T k
          => @unify_pattern'
               _ f pf evm T
               (fun fv
                => @unify_pattern'
                     _ x px evm T
                     (fun xv
                      => k (fv, xv)))
     | pattern.Ident _ _, _
     | pattern.App _ _ _ _, _
       => fun _ k => None
     end%option.
\end{verbatim}

      \begin{itemize}
      \tightlist
      \item
        We have three correctness conditions on
        \texttt{unify\_pattern\textquotesingle{}}:

        \begin{itemize}
        \item
          It must be the case that if we invoke
          \texttt{unify\_pattern\textquotesingle{}} with any
          continuation, the result is the same as invoking it with the
          continuation \texttt{Some}, binding the result in the option
          monad, and then invoking the continuation on the bound value.
        \item
          There is the \texttt{wf} correctness condition, which says
          that if two \texttt{rawexpr}s are
          \texttt{wf\_rawexpr}-related, then invoking
          \texttt{unify\_pattern\textquotesingle{}} with the
          continuation \texttt{Some} either results in \texttt{None} on
          both of them, or it results in two
          \texttt{wf\_unification\_resultT\textquotesingle{}}-related
          results. We define
          \texttt{wf\_unification\_resultT\textquotesingle{}} as

\begin{verbatim}
Fixpoint wf_value' {with_lets : bool} G {t : type} : value'1 with_lets t -> value'2 with_lets t -> Prop
  := match t, with_lets with
     | type.base t, true => UnderLets.wf (fun G' => expr.wf G') G
     | type.base t, false => expr.wf G
     | type.arrow s d, _
       => fun f1 f2
          => (forall seg G' v1 v2,
                 G' = (seg ++ G)%list
                 -> @wf_value' false seg s v1 v2
                 -> @wf_value' true G' d (f1 v1) (f2 v2))
     end.

Definition wf_value G {t} : value1 t -> value2 t -> Prop := @wf_value' false G t.
Definition wf_value_with_lets G {t} : value_with_lets1 t -> value_with_lets2 t -> Prop := @wf_value' true G t.

Fixpoint related_unification_resultT' {var1 var2} (R : forall t, var1 t -> var2 t -> Prop) {t p evm}
  : @unification_resultT' var1 t p evm -> @unification_resultT' var2 t p evm -> Prop
  := match p in pattern.pattern t return @unification_resultT' var1 t p evm -> @unification_resultT' var2 t p evm -> Prop with
     | pattern.Wildcard t => R _
     | pattern.Ident t idc => eq
     | pattern.App s d f x
       => fun (v1 : unification_resultT' f evm * unification_resultT' x evm)
              (v2 : unification_resultT' f evm * unification_resultT' x evm)
          => @related_unification_resultT' _ _ R _ _ _ (fst v1) (fst v2)
             /\ @related_unification_resultT' _ _ R _ _ _ (snd v1) (snd v2)
     end.

Definition wf_unification_resultT' (G : list {t1 : type & (var1 t1 * var2 t1)%type}) {t p evm}
  : @unification_resultT' value t p evm -> @unification_resultT' value t p evm -> Prop
  := @related_unification_resultT' _ _ (fun _ => wf_value G) t p evm.
\end{verbatim}
        \item
          The interp-correctness condition is (a bit more than) a bit of
          a mouthful, and requires some auxiliary definitions.

          \begin{itemize}
          \item
            It is a bit hard to say what makes an expression
            interp-related to an interpreted value. Under the assumption
            of function extensionality, an expression is interp-related
            to a interpreted value if and only if the interpretation of
            the expression is equal to the interpreted value. Thus
            \texttt{expr.interp\_related} is an attempt to avoid
            function extensionality that is not fully successful, likely
            because I cannot say in words what exactly it is supposed to
            mean. The definition is

\begin{verbatim}
Section with_interp.
  Context {base_type : Type}
          {ident : type base_type -> Type}
          {interp_base_type : base_type -> Type}
          (interp_ident : forall t, ident t -> type.interp interp_base_type t).

  Fixpoint interp_related_gen
           {var : type base_type -> Type}
           (R : forall t, var t -> type.interp interp_base_type t -> Prop)
           {t} (e : @expr base_type ident var t)
    : type.interp interp_base_type t -> Prop
    := match e in expr t return type.interp interp_base_type t -> Prop with
       | expr.Var t v1 => R t v1
       | expr.App s d f x
         => fun v2
            => exists fv xv,
                @interp_related_gen var R _ f fv
                /\ @interp_related_gen var R _ x xv
                /\ fv xv = v2
       | expr.Ident t idc
         => fun v2 => interp_ident _ idc == v2
       | expr.Abs s d f1
         => fun f2
            => forall x1 x2,
                R _ x1 x2
                -> @interp_related_gen var R d (f1 x1) (f2 x2)
       | expr.LetIn s d x f (* combine the App rule with the Abs rule *)
         => fun v2
            => exists fv xv,
                @interp_related_gen var R _ x xv
                /\ (forall x1 x2,
                       R _ x1 x2
                       -> @interp_related_gen var R d (f x1) (fv x2))
                /\ fv xv = v2
       end.

  Definition interp_related {t} (e : @expr base_type ident (type.interp interp_base_type) t) : type.interp interp_base_type t -> Prop
    := @interp_related_gen (type.interp interp_base_type) (@type.eqv) t e.
End with_interp.
\end{verbatim}
          \item
            A term in the \texttt{UnderLets} monad is
            \texttt{UnderLets.interp\_related} to an interpreted value
            \texttt{v} if and only if converting the \texttt{UnderLets}
            expression to an \texttt{expr} (by replacing all of the
            \texttt{UnderLets}-let-binders with
            \texttt{expr}-let-binders) results in an expression that is
            \texttt{expr.interp\_related} to \texttt{v}.
          \item
            A \texttt{value} is \texttt{value\_interp\_related} to an
            interpreted value \texttt{v} whenever it sends
            \texttt{interp\_related} things to \texttt{interp\_related}
            things (the arrow case), and satisfies the appropriate
            notion of \texttt{interp\_related} in the base case:

\begin{verbatim}
Fixpoint value_interp_related {t with_lets} : @value' var with_lets t -> type.interp base.interp t -> Prop
  := match t, with_lets with
     | type.base _, true => UnderLets_interp_related
     | type.base _, false => expr_interp_related
     | type.arrow s d, _
       => fun (f1 : @value' _ _ s -> @value' _ _ d) (f2 : type.interp _ s -> type.interp _ d)
          => forall x1 x2,
              @value_interp_related s _ x1 x2
              -> @value_interp_related d _ (f1 x1) (f2 x2)
     end.
\end{verbatim}
          \item
            A \texttt{rawexpr} is \texttt{rawexpr\_interp\_related} to
            an interpreted value \texttt{v} if both the revealed and
            unrevealed structures are appropriately
            \texttt{interp\_related} to \texttt{v}. This one, too, is a
            bit hard to explain in any detail without simply displaying
            the code:

\begin{verbatim}
Fixpoint rawexpr_interp_related (r1 : rawexpr) : type.interp base.interp (type_of_rawexpr r1) -> Prop
  := match r1 return type.interp base.interp (type_of_rawexpr r1) -> Prop with
     | rExpr _ e1
     | rValue (type.base _) e1
       => expr_interp_related e1
     | rValue t1 v1
       => value_interp_related v1
     | rIdent _ t1 idc1 t'1 alt1
       => fun v2
          => expr.interp ident_interp alt1 == v2
             /\ existT expr t1 (expr.Ident idc1) = existT expr t'1 alt1
     | rApp f1 x1 t1 alt1
       => match alt1 in expr.expr t return type.interp base.interp t -> Prop with
          | expr.App s d af ax
            => fun v2
               => exists fv xv (pff : type.arrow s d = type_of_rawexpr f1) (pfx : s = type_of_rawexpr x1),
                   @expr_interp_related _ af fv
                   /\ @expr_interp_related _ ax xv
                   /\ @rawexpr_interp_related f1 (rew pff in fv)
                   /\ @rawexpr_interp_related x1 (rew pfx in xv)
                   /\ fv xv = v2
          | _ => fun _ => False
          end
     end.
\end{verbatim}
          \item
            We can say when a
            \texttt{unification\_resultT\textquotesingle{}} returning an
            \texttt{expr} whose \texttt{var} type is
            \texttt{@value\ (type.interp\ base.interp)} is
            interp-related to a
            \texttt{unification\_resultT\textquotesingle{}} returning an
            \texttt{expr} whose \texttt{var} type is
            \texttt{type.interp\ base.interp} in the semi-obvious way:

\begin{verbatim}
Local Notation var := (type.interp base.interp) (only parsing).

Definition unification_resultT'_interp_related {t p evm}
  : @unification_resultT' (@value var) t p evm -> @unification_resultT' var t p evm -> Prop
  := related_unification_resultT' (fun t => value_interp_related).
\end{verbatim}
          \item
            We say that a \texttt{rawexpr}'s types are ok if the
            revealed and unrevealed structure match on the type level:

\begin{verbatim}
Fixpoint rawexpr_types_ok (r : @rawexpr var) (t : type) : Prop
  := match r with
     | rExpr t' _
     | rValue t' _
       => t' = t
     | rIdent _ t1 _ t2 _
       => t1 = t /\ t2 = t
     | rApp f x t' alt
       => t' = t
          /\ match alt with
             | expr.App s d _ _
               => rawexpr_types_ok f (type.arrow s d)
                  /\ rawexpr_types_ok x s
             | _ => False
             end
     end.
\end{verbatim}
          \item
            We can define a transformation that takes in a
            \texttt{PositiveMap.t} of pattern type variables to types,
            together with a \texttt{PositiveSet.t} of type variables
            that we care about, and re-creates a new
            \texttt{PositiveMap.t} in accordance with the
            \texttt{PositiveSet.t}. This is required to get some theorem
            types to line up, and is possibly an indication of a leaky
            abstraction barrier.

\begin{verbatim}
Local Notation mk_new_evm0 evm ls
  := (fold_right
        (fun i k evm'
         => k match PositiveMap.find i evm with
              | Some v => PositiveMap.add i v evm'
              | None => evm'
              end) (fun evm' => evm')
        (List.rev ls)) (only parsing).

Local Notation mk_new_evm' evm ps
  := (mk_new_evm0
        evm
        (PositiveSet.elements ps)) (only parsing).

Local Notation mk_new_evm evm ps
  := (mk_new_evm' evm ps (PositiveMap.empty _)) (only parsing).
\end{verbatim}
          \item
            Given a proof of \texttt{@types\_match\_with\ evm\ t\ re\ p}
            that the types of \texttt{re\ :\ rawexpr} and
            \texttt{p\ :\ pattern\ t} line up under the mapping
            \texttt{evm}, and a proof of
            \texttt{rawexpr\_types\_ok\ re\ (type\_of\_rawexpr\ re)}, we
            can prove that
            \texttt{type\_of\_rawexpr\ re\ =\ pattern.type.subst\_default\ t\ mk\_new\_evm\ evm\ (pattern\_collect\_vars\ p)}.
            We call this theorem
            \texttt{eq\_type\_of\_rawexpr\_of\_types\_match\_with\textquotesingle{}}.
          \item
            The final and perhaps most important auxiliary component is
            the notation of the default interpretation of a pattern.
            This is a
            \texttt{with\_unification\_resultT\textquotesingle{}} which
            returns the obvious interpreted value after getting all of
            its data; application nodes become applications, identifiers
            get interpreted, wildcards are passed through.

            \begin{itemize}
            \item
              This definition itself needs a few auxiliary definitions
              and context variables.
            \item
              We have a context variable
              \texttt{(pident\_to\_typed\ :\ forall\ t\ (idc\ :\ pident\ t)\ (evm\ :\ EvarMap),\ type\_of\_list\ (pident\_arg\_types\ t\ idc)\ -\textgreater{}\ ident\ (pattern.type.subst\_default\ t\ evm))}
              which takes in a pattern identifier, a mapping of type
              variables to types, and the arguments bound for that
              identifier, and returns a PHOAST identifier of the correct
              type. We require that all type-instantiations of type
              variables of pattern identifiers be valid; this means that
              it doesn't matter if some type variables are missing from
              the mapping and we fill them in with \texttt{unit}
              instead.
            \item
              We define \texttt{lam\_type\_of\_list} to convert between
              the \texttt{cps} and non-cps versions of type lists:

\begin{verbatim}
Local Notation type_of_list
  := (fold_right (fun a b => prod a b) unit).
Local Notation type_of_list_cps
  := (fold_right (fun a K => a -> K)).

Definition lam_type_of_list {ls K} : (type_of_list ls -> K) -> type_of_list_cps K ls
  := list_rect
       (fun ls => (type_of_list ls -> K) -> type_of_list_cps K ls)
       (fun f => f tt)
       (fun T Ts rec k t => rec (fun ts => k (t, ts)))
       ls.
\end{verbatim}
            \item
              We may now define the default interpretation:

\begin{verbatim}
Fixpoint pattern_default_interp' {K t} (p : pattern t) evm {struct p} : (var (pattern.type.subst_default t evm) -> K) -> @with_unification_resultT' var t p evm K
  := match p in pattern.pattern t return (var (pattern.type.subst_default t evm) -> K) -> @with_unification_resultT' var t p evm K with
     | pattern.Wildcard t => fun k v => k v
     | pattern.Ident t idc
       => fun k
          => lam_type_of_list (fun args => k (ident_interp _(pident_to_typed _ idc _ args)))
     | pattern.App s d f x
       => fun k
          => @pattern_default_interp'
               _ _ f evm
               (fun ef
                => @pattern_default_interp'
                     _ _ x evm
                     (fun ex
                      => k (ef ex)))
     end.
\end{verbatim}
            \item
              To define the unprimed version, which also accounts for
              the type variables, we must first define the lambda of
              \texttt{forall\_vars}:

\begin{verbatim}
Fixpoint lam_forall_vars_gen {k : EvarMap -> Type}
         (f : forall evm, k evm)
         (ls : list PositiveMap.key)
  : forall evm0, forall_vars_body k ls evm0
  := match ls return forall evm0, forall_vars_body k ls evm0 with
     | nil => f
     | cons x xs => fun evm t => @lam_forall_vars_gen k f xs _
     end.

Definition lam_forall_vars {p : PositiveSet.t} {k : EvarMap -> Type}
           (f : forall evm, k evm)
  : forall_vars p k
  := @lam_forall_vars_gen k f _ _.
\end{verbatim}
            \item
              Now we can define the default interpretation as a
              \texttt{with\_unification\_resultT}:

\begin{verbatim}
Definition pattern_default_interp {t} (p : pattern t)
  : @with_unification_resultT var t p var
  := pattern.type.lam_forall_vars
       (fun evm
        => pattern_default_interp' p evm id).
\end{verbatim}
            \end{itemize}
          \item
            Now, finally, we may state the interp-correctness condition
            of the pattern unifier:

\begin{verbatim}
Lemma interp_unify_pattern' {t re p evm res v}
      (Hre : rawexpr_interp_related re v)
      (H : @unify_pattern' t re p evm _ (@Some _) = Some res)
      (Ht : @types_match_with evm t re p)
      (Ht' : rawexpr_types_ok re (type_of_rawexpr re))
      (evm' := mk_new_evm evm (pattern_collect_vars p))
      (Hty : type_of_rawexpr re = pattern.type.subst_default t evm'
       := eq_type_of_rawexpr_of_types_match_with' Ht Ht')
  : exists resv : _,
      unification_resultT'_interp_related res resv
      /\ app_transport_with_unification_resultT'_cps
           (pattern_default_interp' p evm' id) resv _ (@Some _)
         = Some (rew Hty in v).
\end{verbatim}
          \end{itemize}
        \end{itemize}
      \end{itemize}
    \item
      We can now glue the type pattern-unifier with the expression
      pattern-unifier in a straightforward way. Note that this pattern
      unifier also has three correctness conditions.

\begin{verbatim}
Definition unify_pattern {t} (e : rawexpr) (p : pattern t)
  : forall T, (unification_resultT p -> option T) -> option T
  := fun T cont
     => unify_types
          e p _
          (fun evm
           => evm <- evm;
                unify_pattern'
                  e p evm T (fun v => cont (existT _ _ v)))%option.
\end{verbatim}

      \begin{itemize}
      \tightlist
      \item
        The first correctness condition is again the cps-identity rule:
        if you invoke \texttt{unify\_pattern} with any continuation,
        that must be the same as invoking it with \texttt{Some}, binding
        the value in the option monad, and then invoking the
        continuation on the bound value.
      \item
        The \texttt{wf} correctness condition requires us to define a
        notion of \texttt{wf} for \texttt{unification\_resultT}.

        \begin{itemize}
        \item
          We say that two \texttt{unification\_resultT}s are
          \texttt{wf}-related if their type-variable-maps are equal, and
          their identifier-arguments and wildcard binding values are
          appropriately \texttt{wf}-related:

\begin{verbatim}
Definition related_sigT_by_eq {A P1 P2} (R : forall x : A, P1 x -> P2 x -> Prop)
           (x : @sigT A P1) (y : @sigT A P2)
  : Prop
  := { pf : projT1 x = projT1 y
     | R _ (rew pf in projT2 x) (projT2 y) }.

          Definition related_unification_resultT {var1 var2} (R : forall t, var1 t -> var2 t -> Prop) {t p}
            : @unification_resultT _ t p -> @unification_resultT _ t p -> Prop
            := related_sigT_by_eq (@related_unification_resultT' _ _ R t p).

          Definition wf_unification_resultT (G : list {t1 : type & (var1 t1 * var2 t1)%type}) {t p}
            : @unification_resultT (@value var1) t p -> @unification_resultT (@value var2) t p -> Prop
            := @related_unification_resultT _ _ (fun _ => wf_value G) t p.
\end{verbatim}
        \item
          The \texttt{wf} correctness condition is then that if we have
          two \texttt{wf\_rawexpr}-related \texttt{rawexpr}s, invoking
          \texttt{unify\_pattern} on each \texttt{rawexpr} to unify it
          with a singular pattern \texttt{p}, with continuation
          \texttt{Some}, results either in \texttt{None} in both cases,
          or in two \texttt{unification\_resultT}s which are
          \texttt{wf\_unification\_resultT}-related.
        \item
          The interpretation correctness condition is a bit of a
          mouthful.

          \begin{itemize}
          \item
            We say that two \texttt{unification\_resultT}s are
            interp-related if their mappings of type variables to types
            are equal, and their packages of non-type binding data are
            appropriately interp-related.

\begin{verbatim}
Local Notation var := (type.interp base.interp) (only parsing).

Definition unification_resultT_interp_related {t p}
  : @unification_resultT (@value var) t p -> @unification_resultT var t p -> Prop
  := related_unification_resultT (fun t => value_interp_related).
\end{verbatim}
          \item
            We can now state the interpretation correctness condition,
            which is a bit hard for me to meaningfully talk about in
            English words except by saying ``it does the right thing for
            a good notion of `right'\,'':

\begin{verbatim}
Lemma interp_unify_pattern {t re p v res}
      (Hre : rawexpr_interp_related re v)
      (Ht' : rawexpr_types_ok re (type_of_rawexpr re))
      (H : @unify_pattern t re p _ (@Some _) = Some res)
      (evm' := mk_new_evm (projT1 res) (pattern_collect_vars p))
  : exists resv,
    unification_resultT_interp_related res resv
    /\ exists Hty, (app_with_unification_resultT_cps (@pattern_default_interp t p) resv _ (@Some _) = Some (existT (fun evm => type.interp base.interp (pattern.type.subst_default t evm)) evm' (rew Hty in v))).
\end{verbatim}
          \end{itemize}
        \end{itemize}
      \end{itemize}
    \end{itemize}
  \end{itemize}
\item
  Plugging in the arguments to a rewrite rule: Take 2

  \begin{itemize}
  \item
    There is one more definition before we put all of the rewrite
    replacement rule pieces together: we describe a way to handle the
    fact that we are underneath zero, one, or two monads. The way we
    handle this is by just assuming that we are underneath two monads,
    and issuing monad-return statements as necessary to correct:

\begin{verbatim}
Definition normalize_deep_rewrite_rule {should_do_again with_opt under_lets t}
  : deep_rewrite_ruleTP_gen should_do_again with_opt under_lets t
    -> deep_rewrite_ruleTP_gen should_do_again true true t
  := match with_opt, under_lets with
     | true , true  => fun x => x
     | false, true  => fun x => Some x
     | true , false => fun x => (x <- x; Some (UnderLets.Base x))%option
     | false, false => fun x => Some (UnderLets.Base x)
     end%cps.
\end{verbatim}

    \begin{itemize}
    \item
      The \texttt{wf} correctness condition, unsurprisingly, just says
      that if two rewrite replacement rules are appropriately
      \texttt{wf}-related, then their normalizations are too. This is
      quite verbose to state, though, because it requires traversing
      multiple layers of monads and pesky dependent types. TODO: should
      this code actually be included?

\begin{verbatim}
Definition wf_maybe_do_again_expr
           {t}
           {rew_should_do_again1 rew_should_do_again2 : bool}
           (G : list {t : _ & (var1 t * var2 t)%type})
  : expr (var:=if rew_should_do_again1 then @value var1 else var1) t
    -> expr (var:=if rew_should_do_again2 then @value var2 else var2) t
    -> Prop
  := match rew_should_do_again1, rew_should_do_again2
           return expr (var:=if rew_should_do_again1 then @value var1 else var1) t
                  -> expr (var:=if rew_should_do_again2 then @value var2 else var2) t
                  -> Prop
     with
     | true, true
       => fun e1 e2
          => exists G',
              (forall t' v1' v2', List.In (existT _ t' (v1', v2')) G' -> wf_value G v1' v2')
              /\ expr.wf G' e1 e2
     | false, false => expr.wf G
     | _, _ => fun _ _ => False
     end.

Definition wf_maybe_under_lets_expr
           {T1 T2}
           (P : list {t : _ & (var1 t * var2 t)%type} -> T1 -> T2 -> Prop)
           (G : list {t : _ & (var1 t * var2 t)%type})
           {rew_under_lets1 rew_under_lets2 : bool}
  : (if rew_under_lets1 then UnderLets var1 T1 else T1)
    -> (if rew_under_lets2 then UnderLets var2 T2 else T2)
    -> Prop
  := match rew_under_lets1, rew_under_lets2
           return (if rew_under_lets1 then UnderLets var1 T1 else T1)
                  -> (if rew_under_lets2 then UnderLets var2 T2 else T2)
                  -> Prop
     with
     | true, true
       => UnderLets.wf P G
     | false, false
       => P G
     | _, _ => fun _ _ => False
     end.

Definition maybe_option_eq {A B} {opt1 opt2 : bool} (R : A -> B -> Prop)
  : (if opt1 then option A else A) -> (if opt2 then option B else B) -> Prop
  := match opt1, opt2 with
     | true, true => option_eq R
     | false, false => R
     | _, _ => fun _ _ => False
     end.

Definition wf_deep_rewrite_ruleTP_gen
           (G : list {t : _ & (var1 t * var2 t)%type})
           {t}
           {rew_should_do_again1 rew_with_opt1 rew_under_lets1 : bool}
           {rew_should_do_again2 rew_with_opt2 rew_under_lets2 : bool}
  : deep_rewrite_ruleTP_gen1 rew_should_do_again1 rew_with_opt1 rew_under_lets1 t
    -> deep_rewrite_ruleTP_gen2 rew_should_do_again2 rew_with_opt2 rew_under_lets2 t
    -> Prop
  := maybe_option_eq
       (wf_maybe_under_lets_expr
          wf_maybe_do_again_expr
          G).

Lemma wf_normalize_deep_rewrite_rule
      {G}
      {t}
      {should_do_again1 with_opt1 under_lets1}
      {should_do_again2 with_opt2 under_lets2}
      {r1 r2}
      (Hwf : @wf_deep_rewrite_ruleTP_gen G t should_do_again1 with_opt1 under_lets1 should_do_again2 with_opt2 under_lets2 r1 r2)
  : option_eq
      (UnderLets.wf (fun G' => wf_maybe_do_again_expr G') G)
      (normalize_deep_rewrite_rule r1) (normalize_deep_rewrite_rule r2).
\end{verbatim}
    \item
      We do not require any interp-correctness condition on
      \texttt{normalize\_deep\_rewrite\_rule}. Instead, we bake
      \texttt{normalize\_deep\_rewrite\_rule} into the per-rewrite-rule
      correctness conditions that a user must prove of every individual
      rewrite rule.
    \end{itemize}
  \item
    Actually, I lied. We need to define the type of a rewrite rule
    before we can say what it means for one to be correct.

    \begin{itemize}
    \item
      An \texttt{anypattern} is a dynamically-typed pattern. This is
      used so that we can talk about \texttt{list}s of rewrite rules.

\begin{verbatim}
Record > anypattern {ident : type -> Type}
  := { type_of_anypattern : type;
       pattern_of_anypattern :> @pattern ident type_of_anypattern }.
\end{verbatim}
    \item
      A \texttt{rewrite\_ruleT} is just a sigma of a pattern of any
      type, with \texttt{rewrite\_rule\_data} over that pattern:

\begin{verbatim}
Definition rewrite_ruleTP
  := (fun p : anypattern => @rewrite_rule_data _ (pattern.pattern_of_anypattern p)).
Definition rewrite_ruleT := sigT rewrite_ruleTP.
Definition rewrite_rulesT
  := (list rewrite_ruleT).
\end{verbatim}
    \end{itemize}
  \item
    We now define a helper definition to support rewriting again in the
    output of a rewrite rule. This is a separate definition mostly to
    make dependent types slightly less painful.

\begin{verbatim}
Definition maybe_do_againT (should_do_again : bool) (t : base.type)
  := ((@expr.expr base.type ident (if should_do_again then value else var) t) -> UnderLets (expr t)).
Definition maybe_do_again
           (do_again : forall t : base.type, @expr.expr base.type ident value t -> UnderLets (expr t))
           (should_do_again : bool) (t : base.type)
  := if should_do_again return maybe_do_againT should_do_again t
     then do_again t
     else UnderLets.Base.
\end{verbatim}

    \begin{itemize}
    \item
      You might think that the correctness condition for this is
      trivial. And, indeed, the \texttt{wf} correctness condition is
      straightforward. In fact, we have already seen it above in
      \texttt{wf\_maybe\_do\_again\_expr}, as there is no proof, only a
      definition of what it means for things to be related depending on
      whether or not we are rewriting again.
    \item
      The interpretation correctness rule, on the other hand, is
      surprisingly subtle. You may have noticed above that
      \texttt{expr.interp\_related} is parameterized on an arbitrary
      \texttt{var} type, and an arbitrary relation between the
      \texttt{var} type and \texttt{type.interp\ base.interp}. I said
      that it is equivalent to equality of interpretation under the
      assumption of function extensionality, but that is only the case
      if \texttt{var} is instantiated to \texttt{type.interp} and the
      relation is equality or pointwise/extensional equivalence. Here,
      we must instantiate the \texttt{var} type with
      \texttt{@value\ var}, and the relation with
      \texttt{value\_interp\_related}. We then prove that for any
      ``good'' notion of rewriting again, if our input value is
      interp-related to an interpreted value, the result of maybe
      rewriting again is also interp-related to that interpreted value.

\begin{verbatim}
Lemma interp_maybe_do_again
      (do_again : forall t : base.type, @expr.expr base.type ident value t -> UnderLets (expr t))
      (Hdo_again : forall t e v,
          expr.interp_related_gen ident_interp (fun t => value_interp_related) e v
          -> UnderLets_interp_related (do_again t e) v)
      {should_do_again : bool} {t e v}
      (He : (if should_do_again return @expr.expr _ _ (if should_do_again then _ else _) _ -> _
             then expr.interp_related_gen ident_interp (fun t => value_interp_related)
             else expr_interp_related) e v)
  : UnderLets_interp_related (@maybe_do_again _ _ do_again should_do_again t e) v.
\end{verbatim}
    \end{itemize}
  \item
    For the purposes of ensuring that reduction does not get blocked
    where it should not, we only allow rewrite rules to match on fully
    applied patterns, and to return base-typed expressions. We patch
    this broken abstraction barrier with

\begin{verbatim}
Local Notation base_type_of t
  := (match t with type.base t' => t' | type.arrow _ __ => base.type.unit end).
\end{verbatim}
  \item
    Finally, we can define what it means to rewrite with a particular
    rewrite rule. It is messy primarily due to continuation passing
    style, optional values, and type casts. Note that we use
    \texttt{\textless{}-} to mean ``bind in whatever monad is the
    top-most scope''. Other than these complications, it just unifies
    the pattern with the \texttt{rawexpr} to get binding data, applies
    the rewrite replacement rule to the binding data, normalizes the
    applied rewrite replacement rule, calls the rewriter again on the
    output if it should, and returns the result.
    \texttt{coq\ \ \ Definition\ rewrite\_with\_rule\ \{t\}\ e\textquotesingle{}\ (pf\ :\ rewrite\_ruleT)\ \ \ ~\ :\ option\ (UnderLets\ (expr\ t))\ \ \ ~\ :=\ let\ \textquotesingle{}existT\ p\ f\ :=\ pf\ in\ \ \ ~\ ~\ ~let\ should\_do\_again\ :=\ rew\_should\_do\_again\ f\ in\ \ \ ~\ ~\ ~unify\_pattern\ \ \ ~\ ~\ ~\ ~e\textquotesingle{}\ (pattern.pattern\_of\_anypattern\ p)\ \_\ \ \ ~\ ~\ ~\ ~(fun\ x\ \ \ ~\ ~\ ~\ ~\ =\textgreater{}\ app\_with\_unification\_resultT\_cps\ \ \ ~\ ~\ ~\ ~\ ~\ ~\ ~(rew\_replacement\ f)\ x\ \_\ \ \ ~\ ~\ ~\ ~\ ~\ ~\ ~(fun\ f\textquotesingle{}\ \ \ ~\ ~\ ~\ ~\ ~\ ~\ ~\ =\textgreater{}\ (tr\ \textless{}-\ type.try\_make\_transport\_cps\ (@base.try\_make\_transport\_cps)\ \_\ \_\ \_;\ \ \ ~\ ~\ ~\ ~\ ~\ ~\ ~\ ~\ ~\ ~\ (tr\ \textless{}-\ tr;\ \ \ ~\ ~\ ~\ ~\ ~\ ~\ ~\ ~\ ~\ ~\ ~\ ~(tr\textquotesingle{}\ \textless{}-\ type.try\_make\_transport\_cps\ (@base.try\_make\_transport\_cps)\ \_\ \_\ \_;\ \ \ ~\ ~\ ~\ ~\ ~\ ~\ ~\ ~\ ~\ ~\ ~\ ~\ ~\ (tr\textquotesingle{}\ \textless{}-\ tr\textquotesingle{};\ \ \ ~\ ~\ ~\ ~\ ~\ ~\ ~\ ~\ ~\ ~\ ~\ ~\ ~\ ~\ ~option\_bind\textquotesingle{}\ \ \ ~\ ~\ ~\ ~\ ~\ ~\ ~\ ~\ ~\ ~\ ~\ ~\ ~\ ~\ ~\ ~(normalize\_deep\_rewrite\_rule\ (projT2\ f\textquotesingle{}))\ \ \ ~\ ~\ ~\ ~\ ~\ ~\ ~\ ~\ ~\ ~\ ~\ ~\ ~\ ~\ ~\ ~(fun\ fv\ \ \ ~\ ~\ ~\ ~\ ~\ ~\ ~\ ~\ ~\ ~\ ~\ ~\ ~\ ~\ ~\ ~\ =\textgreater{}\ Some\ (fv\ \textless{}-\/-\ fv;\ \ \ ~\ ~\ ~\ ~\ ~\ ~\ ~\ ~\ ~\ ~\ ~\ ~\ ~\ ~\ ~\ ~\ ~\ ~\ ~\ ~\ ~\ ~fv\ \textless{}-\/-\ maybe\_do\_again\ should\_do\_again\ (base\_type\_of\ (type\_of\_rawexpr\ e\textquotesingle{}))\ (tr\ fv);\ \ \ ~\ ~\ ~\ ~\ ~\ ~\ ~\ ~\ ~\ ~\ ~\ ~\ ~\ ~\ ~\ ~\ ~\ ~\ ~\ ~\ ~\ ~UnderLets.Base\ (tr\textquotesingle{}\ fv))\%under\_lets))\%option)\%cps)\%option)\%cps)\%cps).}

    \begin{itemize}
    \tightlist
    \item
      We once again do not have any \texttt{wf} correctness condition
      for \texttt{rewrite\_with\_rule}; we merely unfold it as needed.
    \item
      To write down the correctness condition for
      \texttt{rewrite\_with\_rule}, we must first define what it means
      for \texttt{rewrite\_rule\_data} to be ``good''.

      \begin{itemize}
      \item
        Here is where we use \texttt{normalize\_deep\_rewrite\_rule}.
        Replacement rule data is good with respect to an interpretation
        value if normalizing it gives an appropriately interp-related
        thing to that interpretation value:

\begin{verbatim}
Local Notation var := (type.interp base.interp) (only parsing).

Definition deep_rewrite_ruleTP_gen_good_relation
           {should_do_again with_opt under_lets : bool} {t}
           (v1 : @deep_rewrite_ruleTP_gen should_do_again with_opt under_lets t)
           (v2 : var t)
  : Prop
  := let v1 := normalize_deep_rewrite_rule v1 in
     match v1 with
     | None => True
     | Some v1 => UnderLets.interp_related
                    ident_interp
                    (if should_do_again
                        return (@expr.expr base.type ident (if should_do_again then @value var else var) t) -> _
                     then expr.interp_related_gen ident_interp (fun t => value_interp_related)
                     else expr_interp_related)
                    v1
                    v2
     end.
\end{verbatim}
      \item
        Rewrite rule data is good if, for any interp-related binding
        data, the replacement function applied to the value-binding-data
        is interp-related to the default interpretation of the pattern
        applied to the interpreted-value-binding-data:

\begin{verbatim}
Definition rewrite_rule_data_interp_goodT
           {t} {p : pattern t} (r : @rewrite_rule_data t p)
  : Prop
  := forall x y,
    related_unification_resultT (fun t => value_interp_related) x y
    -> option_eq
         (fun fx gy
          => related_sigT_by_eq
               (fun evm
                => @deep_rewrite_ruleTP_gen_good_relation
                     (rew_should_do_again r) (rew_with_opt r) (rew_under_lets r) (pattern.type.subst_default t evm))
               fx gy)
         (app_with_unification_resultT_cps (rew_replacement r) x _ (@Some _))
         (app_with_unification_resultT_cps (pattern_default_interp p) y _ (@Some _)).
\end{verbatim}
      \item
        The interpretation correctness condition then says that if the
        rewrite rule is good, the \texttt{rawexpr} \texttt{re} has ok
        types, the ``rewrite again'' function is good, and
        \texttt{rewrite\_with\_rule} succeeds and outputs an expression
        \texttt{v1}, then \texttt{v1} is interp-related to any
        interpreted value which \texttt{re} is interp-related to:

\begin{verbatim}
Lemma interp_rewrite_with_rule
      (do_again : forall t : base.type, @expr.expr base.type ident value t -> UnderLets (expr t))
      (Hdo_again : forall t e v,
          expr.interp_related_gen ident_interp (fun t => value_interp_related) e v
          -> UnderLets_interp_related (do_again t e) v)
      (rewr : rewrite_ruleT)
      (Hrewr : rewrite_rule_data_interp_goodT (projT2 rewr))
      t e re v1 v2
      (Ht : t = type_of_rawexpr re)
      (Ht' : rawexpr_types_ok re (type_of_rawexpr re))
  : @rewrite_with_rule do_again t re rewr = Some v1
    -> rawexpr_interp_related re (rew Ht in v2)
    -> UnderLets_interp_related v1 v2.
\end{verbatim}

        \hypertarget{tying-it-all-together}{%
        \subsubsection{Tying it all
        together}\label{tying-it-all-together}}
      \end{itemize}
    \end{itemize}
  \end{itemize}
\item
  We can now say what it means to rewrite with a decision tree in a
  given \texttt{rawexpr} \texttt{re}. We evaluate the decision tree, and
  whenever we are asked to try the \texttt{k}th rewrite rule, we look
  for it in our list of rewrite rules, and invoke
  \texttt{rewrite\_with\_rule}. By default, if rewriting fails, we will
  eventually return \texttt{expr\_of\_rawexpr\ re}.

\begin{verbatim}
Definition eval_rewrite_rules
           (d : decision_tree)
           (rews : rewrite_rulesT)
           (e : rawexpr)
  : UnderLets (expr (type_of_rawexpr e))
  := let defaulte := expr_of_rawexpr e in
     (eval_decision_tree
        (e::nil) d
        (fun k ctx
         => match ctx return option (UnderLets (expr (type_of_rawexpr e))) with
            | e'::nil
              => (pf <- nth_error rews k; rewrite_with_rule e' pf)%option
            | _ => None
            end);;;
        (UnderLets.Base defaulte))%option.
\end{verbatim}

  \begin{itemize}
  \tightlist
  \item
    To define the correctness conditions, we must first define what it
    means for lists of rewrite rules to be good.

    \begin{itemize}
    \item
      For \texttt{wf}, we need to catch up a bit before getting to lists
      of rewrite rules. These say the obvious things:

\begin{verbatim}
          Definition wf_with_unif_rewrite_ruleTP_gen
                     (G : list {t : _ & (var1 t * var2 t)%type})
                     {t} {p : pattern t}
                     {rew_should_do_again1 rew_with_opt1 rew_under_lets1}
                     {rew_should_do_again2 rew_with_opt2 rew_under_lets2}
            : with_unif_rewrite_ruleTP_gen1 p rew_should_do_again1 rew_with_opt1 rew_under_lets1
              -> with_unif_rewrite_ruleTP_gen2 p rew_should_do_again2 rew_with_opt2 rew_under_lets2
              -> Prop
            := fun f g
               => forall x y,
                   wf_unification_resultT G x y
                   -> option_eq
                        (fun (fx : { evm : _ & deep_rewrite_ruleTP_gen1 rew_should_do_again1 rew_with_opt1 rew_under_lets1 _ })
                             (gy : { evm : _ & deep_rewrite_ruleTP_gen2 rew_should_do_again2 rew_with_opt2 rew_under_lets2 _ })
                         => related_sigT_by_eq
                              (fun _ => wf_deep_rewrite_ruleTP_gen G) fx gy)
                        (app_with_unification_resultT_cps f x _ (@Some _))
                        (app_with_unification_resultT_cps g y _ (@Some _)).

          Definition wf_rewrite_rule_data
                     (G : list {t : _ & (var1 t * var2 t)%type})
                     {t} {p : pattern t}
                     (r1 : @rewrite_rule_data1 t p)
                     (r2 : @rewrite_rule_data2 t p)
            : Prop
            := wf_with_unif_rewrite_ruleTP_gen G (rew_replacement r1) (rew_replacement r2).
\end{verbatim}
    \item
      Two lists of rewrite rules are \texttt{wf} related if they have
      the same length, and if any pair of rules in their zipper
      (\texttt{List.combine}) have equal patterns and
      \texttt{wf}-related data:

\begin{verbatim}
Definition rewrite_rules_goodT
           (rew1 : rewrite_rulesT1) (rew2 : rewrite_rulesT2)
  : Prop
  := length rew1 = length rew2
     /\ (forall p1 r1 p2 r2,
            List.In (existT _ p1 r1, existT _ p2 r2) (combine rew1 rew2)
            -> { pf : p1 = p2
               | forall G,
                   wf_rewrite_rule_data
                     G
                     (rew [fun tp => @rewrite_rule_data1 _ (pattern.pattern_of_anypattern tp)] pf in r1)
                     r2 }).
\end{verbatim}
    \item
      A list of rewrite rules is good for interpretation if every
      rewrite rule in that list is good for interpretation:

\begin{verbatim}
Definition rewrite_rules_interp_goodT
           (rews : rewrite_rulesT)
  : Prop
  := forall p r,
    List.In (existT _ p r) rews
    -> rewrite_rule_data_interp_goodT r.
\end{verbatim}
    \item
      The \texttt{wf}-correctness condition for
      \texttt{eval\_rewrite\_rules} says the obvious thing: for
      \texttt{wf}-related ``rewrite again'' functions,
      \texttt{wf}-related lists of rewrite rules, and
      \texttt{wf}-related \texttt{rawexpr}s, the output of
      \texttt{eval\_rewrite\_rules} is \texttt{wf}-related:

\begin{verbatim}
Lemma wf_eval_rewrite_rules
      (do_again1 : forall t : base.type, @expr.expr base.type ident (@value var1) t -> @UnderLets var1 (@expr var1 t))
      (do_again2 : forall t : base.type, @expr.expr base.type ident (@value var2) t -> @UnderLets var2 (@expr var2 t))
      (wf_do_again : forall G (t : base.type) e1 e2,
          (exists G', (forall t v1 v2, List.In (existT _ t (v1, v2)) G' -> Compile.wf_value G v1 v2) /\ expr.wf G' e1 e2)
          -> UnderLets.wf (fun G' => expr.wf G') G (@do_again1 t e1) (@do_again2 t e2))
      (d : @decision_tree raw_pident)
      (rew1 : rewrite_rulesT1) (rew2 : rewrite_rulesT2)
      (Hrew : rewrite_rules_goodT rew1 rew2)
      (re1 : @rawexpr var1) (re2 : @rawexpr var2)
      {t} G e1 e2
      (Hwf : @wf_rawexpr G t re1 e1 re2 e2)
  : UnderLets.wf
      (fun G' => expr.wf G')
      G
      (rew [fun t => @UnderLets var1 (expr t)] (proj1 (eq_type_of_rawexpr_of_wf Hwf)) in (eval_rewrite_rules1 do_again1 d rew1 re1))
      (rew [fun t => @UnderLets var2 (expr t)] (proj2 (eq_type_of_rawexpr_of_wf Hwf)) in (eval_rewrite_rules2 do_again2 d rew2 re2)).
\end{verbatim}
    \item
      The interpretation correctness is also the expected one: for a
      ``rewrite again'' function that preserves interp-relatedness, a
      good-for-interp list of rewrite rules, a \texttt{rawexpr} whose
      types are ok and which is interp-related to a value \texttt{v},
      the result of \texttt{eval\_rewrite\_rules} is interp-related to
      \texttt{v}:

\begin{verbatim}
Lemma interp_eval_rewrite_rules
      (do_again : forall t : base.type, @expr.expr base.type ident value t -> UnderLets (expr t))
      (d : decision_tree)
      (rew_rules : rewrite_rulesT)
      (re : rawexpr) v
      (Hre : rawexpr_types_ok re (type_of_rawexpr re))
      (res := @eval_rewrite_rules do_again d rew_rules re)
      (Hdo_again : forall t e v,
          expr.interp_related_gen ident_interp (fun t => value_interp_related) e v
          -> UnderLets_interp_related (do_again t e) v)
      (Hr : rawexpr_interp_related re v)
      (Hrew_rules : rewrite_rules_interp_goodT rew_rules)
  : UnderLets_interp_related res v.
\end{verbatim}
    \end{itemize}
  \end{itemize}
\item
  Only one piece remains (other than defining particular rewrite rules
  and proving them good). If you were following carefully, you might
  note that \texttt{eval\_rewrite\_rules} takes in a \texttt{rawexpr}
  and produces an \texttt{UnderLets\ expr}, while
  \texttt{rewrite\_bottomup} expects a function
  \texttt{rewrite\_head\ :\ forall\ t\ (idc\ :\ ident\ t),\ value\_with\_lets\ t}.
  From a PHOAST identifier, we must construct a
  \texttt{value\_with\_lets} which collects all of the \texttt{value}
  arguments to the identifier and performs \texttt{eval\_rewrite\_rules}
  once the identifier is fully applied. We call this function
  \texttt{assemble\_identifier\_rewriters}, and it is built out of a
  small number of pieces.

  \begin{itemize}
  \item
    We define a convenience function that takes a \texttt{value} and an
    \texttt{expr} at the same type, and produces a \texttt{rawexpr} by
    using \texttt{rExpr} on the expr if the type is a base type, and
    \texttt{rValue} on the \texttt{value} otherwise. Morally, the
    \texttt{expr} and the \texttt{value} should be the same term, modulo
    \texttt{reify} and/or \texttt{reflect}:

\begin{verbatim}
Definition rValueOrExpr2 {t} : value t -> expr t -> rawexpr
  := match t with
     | type.base _ => fun v e => @rExpr _ e
     | type.arrow _ _ => fun v e => @rValue _ v
     end.
\end{verbatim}
  \item
    We take in a context variable (eventually autogenerated by python)
    which eta-iota-expands a function over an identifier by producing a
    \texttt{match} on the identifier. Its specification is that it is
    pointwise-equal to function application; it exists entirely so that
    we can perform rewrite-rule-compilation time reduction on the
    rewrite rules by writing down the cases for every head identifier
    separately. The context variable is
    \texttt{eta\_ident\_cps\ :\ forall\ \{T\ :\ type.type\ base.type\ -\textgreater{}\ Type\}\ \{t\}\ (idc\ :\ ident\ t)\ (f\ :\ forall\ t\textquotesingle{},\ ident\ t\textquotesingle{}\ -\textgreater{}\ T\ t\textquotesingle{}),\ T\ t},
    and we require that
    \texttt{forall\ T\ t\ idc\ f,\ @eta\_ident\_cps\ T\ t\ idc\ f\ =\ f\ t\ idc}.
  \item
    We can now carefully define the function that turns
    \texttt{eval\_rewrite\_rules} into a thing that can be plugged into
    \texttt{rewrite\_head}. We take care to preserve thunked computation
    in \texttt{rValue} nodes, while describing the alternative structure
    via \texttt{reify}. In general, the stored values are only
    interp-related to the same things that the ``unrevealed structure''
    expressions are interp-related to. There is no other relation (that
    we've found) between the values and the expressions, and this caused
    a great deal of pain when trying to specify the interpretation
    correctness properties.

\begin{verbatim}
Section with_do_again.
  Context (dtree : decision_tree)
          (rewrite_rules : rewrite_rulesT)
          (default_fuel : nat)
          (do_again : forall t : base.type, @expr.expr base.type ident value t -> UnderLets (expr t)).

  Let dorewrite1 (e : rawexpr) : UnderLets (expr (type_of_rawexpr e))
    := eval_rewrite_rules do_again dtree rewrite_rules e.

  Fixpoint assemble_identifier_rewriters' (t : type) : forall e : rawexpr, (forall P, P (type_of_rawexpr e) -> P t) -> value_with_lets t
    := match t return forall e : rawexpr, (forall P, P (type_of_rawexpr e) -> P t) -> value_with_lets t with
       | type.base _
         => fun e k => k (fun t => UnderLets (expr t)) (dorewrite1 e)
       | type.arrow s d
         => fun f k (x : value' _ _)
            => let x' := reify x in
               @assemble_identifier_rewriters' d (rApp f (rValueOrExpr2 x x') (k _ (expr_of_rawexpr f) @ x'))%expr (fun _ => id)
       end%under_lets.

  Definition assemble_identifier_rewriters {t} (idc : ident t) : value_with_lets t
    := eta_ident_cps _ _ idc (fun t' idc' => assemble_identifier_rewriters' t' (rIdent true idc' #idc') (fun _ => id)).
End with_do_again.
\end{verbatim}

    \begin{itemize}
    \item
      The \texttt{wf}-correctness condition for
      \texttt{assemble\_identifier\_rewriters\textquotesingle{}} says
      that if two \texttt{rawexpr}s are \texttt{wf}-related, and both
      continuations are extensionally/pointwise equal to the identity
      function transported across the appropriate equality proof, then
      the results of
      \texttt{assemble\_identifier\_rewriters\textquotesingle{}} are
      \texttt{wf}-related, under the assumption that the ``rewrite
      again'' functions are appropriately \texttt{wf}-related and the
      list of rewrite rules is good.

\begin{verbatim}
Section with_do_again.
  Context (dtree : @decision_tree raw_pident)
          (rew1 : rewrite_rulesT1)
          (rew2 : rewrite_rulesT2)
          (Hrew : rewrite_rules_goodT rew1 rew2)
          (do_again1 : forall t : base.type, @expr.expr base.type ident (@value var1) t -> @UnderLets var1 (@expr var1 t))
          (do_again2 : forall t : base.type, @expr.expr base.type ident (@value var2) t -> @UnderLets var2 (@expr var2 t))
          (wf_do_again : forall G G' (t : base.type) e1 e2,
              (forall t v1 v2, List.In (existT _ t (v1, v2)) G' -> Compile.wf_value G v1 v2)
              -> expr.wf G' e1 e2
              -> UnderLets.wf (fun G' => expr.wf G') G (@do_again1 t e1) (@do_again2 t e2)).

  Lemma wf_assemble_identifier_rewriters' G t re1 e1 re2 e2
        K1 K2
        (He : @wf_rawexpr G t re1 e1 re2 e2)
        (HK1 : forall P v, K1 P v = rew [P] (proj1 (eq_type_of_rawexpr_of_wf He)) in v)
        (HK2 : forall P v, K2 P v = rew [P] (proj2 (eq_type_of_rawexpr_of_wf He)) in v)
    : wf_value_with_lets
        G
        (@assemble_identifier_rewriters' var1 rew1 do_again1 t re1 K1)
        (@assemble_identifier_rewriters' var2 rew2 do_again2 t re2 K2).
\end{verbatim}
    \item
      The \texttt{wf}-correctness condition for
      \texttt{assemble\_identifier\_rewriters} merely says that the
      outputs are always \texttt{wf}-related, again under the assumption
      that the ``rewrite again'' functions are appropriately
      \texttt{wf}-related and the list of rewrite rules is good.

\begin{verbatim}
Lemma wf_assemble_identifier_rewriters G t (idc : ident t)
  : wf_value_with_lets
      G
      (@assemble_identifier_rewriters var1 rew1 do_again1 t idc)
      (@assemble_identifier_rewriters var2 rew2 do_again2 t idc).
Proof.
\end{verbatim}
    \item
      The interpretation correctness condition says that for a good
      ``rewrite again'' function, a good-for-interpretation list of
      rewrite rules, a \texttt{rawexpr} \texttt{re} whose types are ok
      and which is interp-related to an interpreted value \texttt{v},
      the result of
      \texttt{assemble\_identifier\_rewriters\textquotesingle{}} is
      interp-related to \texttt{v}. The actual statement is slightly
      more obscure, parameterizing over types which are equal to
      computed things, primarily for ease of induction in the proof.

\begin{verbatim}
Lemma interp_assemble_identifier_rewriters'
      (do_again : forall t : base.type, @expr.expr base.type ident value t -> UnderLets (expr t))
      (dt : decision_tree)
      (rew_rules : rewrite_rulesT)
      t re K
      (res := @assemble_identifier_rewriters' dt rew_rules do_again t re K)
      (Hre : rawexpr_types_ok re (type_of_rawexpr re))
      (Ht : type_of_rawexpr re = t)
      v
      (HK : K = (fun P v => rew [P] Ht in v))
      (Hdo_again : forall t e v,
          expr.interp_related_gen ident_interp (fun t => value_interp_related) e v
          -> UnderLets_interp_related (do_again t e) v)
      (Hrew_rules : rewrite_rules_interp_goodT rew_rules)
      (Hr : rawexpr_interp_related re v)
  : value_interp_related res (rew Ht in v).
\end{verbatim}
    \item
      The interpretation correctness condition for
      \texttt{assemble\_identifier\_rewriters} is very similar, where
      the \texttt{rawexpr\_interp\_related} hypothesis is replaced by an
      pointwise equality between the interpretation of the identifier
      and the interpreted value.

\begin{verbatim}
Lemma interp_assemble_identifier_rewriters
      (do_again : forall t : base.type, @expr.expr base.type ident value t -> UnderLets (expr t))
      (d : decision_tree)
      (rew_rules : rewrite_rulesT)
      t idc v
      (res := @assemble_identifier_rewriters d rew_rules do_again t idc)
      (Hdo_again : forall t e v,
          expr.interp_related_gen ident_interp (fun t => value_interp_related) e v
          -> UnderLets_interp_related (do_again t e) v)
      (Hrew_rules : rewrite_rules_interp_goodT rew_rules)
      (Hv : ident_interp t idc == v)
  : value_interp_related res v.
\end{verbatim}
    \end{itemize}
  \item
    We have not talked about correctness conditions for the functions we
    looked at in the very beginning, \texttt{rewrite\_bottomup} and
    \texttt{repeat\_rewrite}, but the correctness conditions for these
    two are straightforward, so we state them without explanation.

    \begin{itemize}
    \item
      The \texttt{wf} correctness conditions are

\begin{verbatim}
Section with_rewrite_head.
  Context (rewrite_head1 : forall t (idc : ident t), @value_with_lets var1 t)
          (rewrite_head2 : forall t (idc : ident t), @value_with_lets var2 t)
          (wf_rewrite_head : forall G t (idc1 idc2 : ident t),
              idc1 = idc2 -> wf_value_with_lets G (rewrite_head1 t idc1) (rewrite_head2 t idc2)).

  Local Notation rewrite_bottomup1 := (@rewrite_bottomup var1 rewrite_head1).
  Local Notation rewrite_bottomup2 := (@rewrite_bottomup var2 rewrite_head2).

  Lemma wf_rewrite_bottomup G G' {t} e1 e2 (Hwf : expr.wf G e1 e2)
        (HG : forall t v1 v2, List.In (existT _ t (v1, v2)) G -> wf_value G' v1 v2)
    : wf_value_with_lets G' (@rewrite_bottomup1 t e1) (@rewrite_bottomup2 t e2).
End with_rewrite_head.

Local Notation nbe var := (@rewrite_bottomup var (fun t idc => reflect (expr.Ident idc))).

Lemma wf_nbe G G' {t} e1 e2
      (Hwf : expr.wf G e1 e2)
      (HG : forall t v1 v2, List.In (existT _ t (v1, v2)) G -> wf_value G' v1 v2)
  : wf_value_with_lets G' (@nbe var1 t e1) (@nbe var2 t e2).

Section with_rewrite_head2.
  Context (rewrite_head1 : forall (do_again : forall t : base.type, @expr (@value var1) (type.base t) -> @UnderLets var1 (@expr var1 (type.base t)))
                                  t (idc : ident t), @value_with_lets var1 t)
          (rewrite_head2 : forall (do_again : forall t : base.type, @expr (@value var2) (type.base t) -> @UnderLets var2 (@expr var2 (type.base t)))
                                  t (idc : ident t), @value_with_lets var2 t)
          (wf_rewrite_head
           : forall
              do_again1
              do_again2
              (wf_do_again
               : forall G' G (t : base.type) e1 e2
                        (HG : forall t v1 v2, List.In (existT _ t (v1, v2)) G -> wf_value G' v1 v2),
                  expr.wf G e1 e2
                  -> UnderLets.wf (fun G' => expr.wf G') G' (do_again1 t e1) (do_again2 t e2))
              G t (idc1 idc2 : ident t),
              idc1 = idc2 -> wf_value_with_lets G (rewrite_head1 do_again1 t idc1) (rewrite_head2 do_again2 t idc2)).

  Lemma wf_repeat_rewrite fuel
    : forall {t} G G' e1 e2
             (Hwf : expr.wf G e1 e2)
             (HG : forall t v1 v2, List.In (existT _ t (v1, v2)) G -> wf_value G' v1 v2),
      wf_value_with_lets G' (@repeat_rewrite var1 rewrite_head1 fuel t e1) (@repeat_rewrite var2 rewrite_head2 fuel t e2).
\end{verbatim}
    \item
      The interpretation correctness conditions are

\begin{verbatim}
Section with_rewrite_head.
  Context (rewrite_head : forall t (idc : ident t), value_with_lets t)
          (interp_rewrite_head : forall t idc v, ident_interp idc == v -> value_interp_related (rewrite_head t idc) v).

  Lemma interp_rewrite_bottomup {t e v}
        (He : expr.interp_related_gen (@ident_interp) (fun t => value_interp_related) e v)
    : value_interp_related (@rewrite_bottomup var rewrite_head t e) v.
End with_rewrite_head.

Local Notation nbe := (@rewrite_bottomup var (fun t idc => reflect (expr.Ident idc))).

Lemma interp_nbe {t e v}
      (He : expr.interp_related_gen (@ident_interp) (fun t => value_interp_related) e v)
  : value_interp_related (@nbe t e) v.

Lemma interp_repeat_rewrite
      {rewrite_head fuel t e v}
      (retT := value_interp_related (@repeat_rewrite _ rewrite_head fuel t e) v)
      (Hrewrite_head
       : forall do_again
                (Hdo_again : forall t e v,
                    expr.interp_related_gen (@ident_interp) (fun t => value_interp_related) e v
                    -> UnderLets.interp_related (@ident_interp) (expr.interp_related (@ident_interp)) (do_again t e) v)
                t idc v,
          ident_interp idc == v
          -> value_interp_related (@rewrite_head do_again t idc) v)
      (He : expr.interp_related_gen (@ident_interp) (fun t => value_interp_related) e v)
  : retT.
\end{verbatim}
    \end{itemize}
  \end{itemize}
\end{itemize}

