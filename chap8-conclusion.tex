\chapter{Concluding Remarks}

\begin{subappendices}
    
    \section{transcript bits from Adam}
And then they'll be the conclusion which I'm thinking of having as a like what are the next steps in performance of previous distance and I think this will. My current inclination is to like, Sort of point towards the paper that under has been talking about writing that's like okay, so there's a sense in which in order to do program transformation and rewriting we took the entire non-trusted part and we threw it out. 

The like. And like part of that is because most uses of the non-trusted part you just cobble something together that works but if you're like tracking every single time you invoke conversion and you're like carefully piecing together something it should be possible to make something that scales. And it's not clear if that's currently even the case. 

Unlike investigating that it's sort of the next wave of. Performance issues to look at. Okay. Well when you get to the conclusion of this sort of document you've pretty freehand to speculate on things and go where your heart takes you so I'm not too worried about I haven't feel like that part of the relatively quick for you to write and free of difficult choice points. 

Yeah, that seems that seems mostly true. I feel like I'll have a little bit of trouble with the like first paragraph on the last paragraph of the conclusion. I'd like the transition points and they'll like actually tying it up but the body of it seems I don't expect to have that much trouble, okay? 

    
    \section{transcript bits from Rajee}

\todo{Decide between options, maybe add more text}
\paragraph{Option 1}
Perhaps this thesis has inspired you to write your own performance system and we remind you about the things you should look out for when implementing it. 

\paragraph{Option 2}

The End

\todo{insert category theory diagram of an End here}

\paragraph{Option 3 (best so far)}

What are the next steps in proof assistant performance.  There's a paper that Andres has floated writing that I think is a good next paper to write. 

Ah, that is something like okay, so you've like, Followed all the tenants that I've laid out to like have fast APIs you're like very careful about where you're having called two things. And then you start hitting so brief historical perspective. I've described a bunch of like quadratic or exponential behaviors where like you're hitting. 

Areas of the system that aren't scaling nicely. There was a previous generation to this where pretty much everything was quadratic or exponential in like everything and so you couldn't do anything beyond a certain scale because everything would start blowing up on you I see and there was someone before me named George got there who when working on he was the one who led the team at Microsoft Research to you formalized the four color theorem. 

I think now not the four color theorem the odd order theorem. In call okay took them about ten years. I think you've mentioned this yeah oh and he went on the cocktails and they fixed these like everything is terrible and everything. So now we're heading like problems that maybe maybe are more fundamental to proof assistance. 

But like then you you design your things carefully and you're careful about which parts the system you use and you'd like count for every step. And then you start hitting the next class of problems, which is I have a couple thousand things. A couple thousand variables and I want to introduce them all oops adding a couple thousand like adding n variables is quadratic or cubic in the number of variables that I'm introducing that's unfortunate. 

Um, or you're like I want to like change my goal state oops making a new goal state is linear in how many variables there that's sad now. I'm now by running time is quadratic in the number of goal states or something mm-hmm. 

And like you hit all of these like the fundamental building blocks. Are too slow. And. That's sort of the next area to investigate of like how do you build a proof assistant so like what are the fundamental building blocks? How are they too slow? Huh the how do we know there are two slow what what are the factors that they're too slow and like can we show that there's like no way to get anything to actually scale without completely re-implementing the profanion because that's basically what I what I said for program transformation. 

I'm like look the existing thing it's quadratic it's real sad let's throw it out and write a new one and stick it in the part of the system that's fast. So like yeah, you can do that for all your proofs you can throw out the entire pure system and write a new one. 

But like, Would be nice if you didn't have to do that we say that again. So like you're like, okay, I was trying to do this thing no just the last sentence, oh it would be nice if we didn't have to do that yeah. So the alternative is to the to the alternative is that you figure out what the primitives are what they're too slow and why they're too slow and how do you design a proof assistant like a proof engine with primitives that are actually performant that if you're carefully accounting for all of the primitive steps that you're doing in your proof then you can actually get a proof with reasonable performance. 

Like all the things that I've been describing are. You slap something together and it works on small things and then you increase your you try to scale it and it's suddenly stops working because of exponential behavior. And like, Maybe there isn't a hope of fixing that if you slap something together. 

But if you're like carefully engineering your proof, you should be able to avoid that. What is the careful part like can you describe that or is it just like the thing so? Okay, so here's how here's how beginners pure things in caulk. Their teacher tells them what they're trying to prove. 

They look at what they're trying to prove they look at the list of things they can do they're like, oh I'm trying to prove for all X something. I know a tactic to use. I'm gonna use interest. Oh I'm trying I have a conjunction in my hypothesis that I know a tactic to use I'm gonna use destruct. 

I'm trying to prove something about natural numbers, how do I prove something about natural numbers by induction? Where you have this very simple pattern match that are matching program that's running in a brain that you're like how do I do this thing one step at a time? I'm just gonna try a thing and see what works we have some arithmetic, let's try simple let's see if cock knows how to make it simpler there's a tactic called simple without the, Okay, um, sometimes it makes things much nicer. 

Sometimes it makes things explode, sometimes it runs forever and gives you nothing it doesn't actually ever run forever pretty much. But running for a year is about as good as running forever. 

And so you'll try it and if it works then you're like great it worked. I can keep going yeah and if it doesn't work then you're like, oh I guess it didn't work, let me try something else instead. And like this is this is how beginners implement proofs and like the way I do proofs is I'm like, okay, let me figure out why this thing should be true. 

And let me figure out what gets me closer to my understanding of why it should be true and then I run the same kind of simple program that um that beginners run that's like, oh this should be true by induction on this variable. I'm not just doing induction randomly. 

I know why I'm doing induction on what and I'm like, oh I have this conjunction. I can split it apart. I have this disjunction I can split it apart and like I keep making steps and at each point. I'm like, am I still convinced that this theorem is true? 

And if I have ever I'm like, oh doesn't these seem like this true anymore that I'd like backup but otherwise I just keep going as long as I'm convinced that the theorem is still reasonable. Where you say something like you do things by figuring out why something should be true is that like. 

Is that like constructing approved sketching your head and then doing it versus someone being like oh I know what tactic to implement them, therefore. I will try to construct yeah it's like using a proof method to generate a proof versus knowing what you want to prove and then writing it need to or something oh where is this something different like how does it apply to the engineering case? 

I think it's something like that okay, so the thing that I'm doing is I'm like do I believe that this is true when I explain to a very intelligent five-year-old why this is true. And then I'll make steps unlike if at any point. I hit a theorem that I or like I hit a state where I'm like. 

This is doesn't seem true anymore. That I'll like back up but I and like I have a big sense of the proof in my head okay, oh but it's like I'm like, okay this this should follow by arithmetic. So then I do a bunch of arithmetic like things and eventually hopefully I get out a thing that's true, but it's like if I want to prove that something is true by arithmetic. 

I can just like look at my thing take a step that makes the thing simpler and if the thing still seems true that I'm like great I made it simpler now what and like I can keep taking steps to make it simpler until it's done and I don't have to have a like entire proof in my head. 

That's interesting. Don't yeah yeah, this is because I got lined by line feedback on my prefixes. I go along it's great. Yeah. The problem with doing things this way is that they don't scale yeah it seems hard like. It seemed like I feel like with most problems you have to kind of have a proof in your head and then use the syntax to like. 

Make it so let me let me also clarify yeah the sorts of proofs currently that you need to do in caulk or way simpler and the more tedious and the sorts of proofs that you're thinking of here's an example of a proof that you might have to do in caulk, um, this is this is like on the interesting end of proofs okay, oh if you, Have a loop that adds up all the numbers between one and m. 

It's the same thing as multiplying n times, m, plus one dividing by two, okay? This is the interesting proof here's another interesting proof, we're merged sword and bubble sort give you the same list if you give them the same list then. In both cases do just do it you you prove so the things you need to prove is you need to prove that they're included or do you just run both things and say no you can't run both things because you need to prove that it's true for every single list, right? 

So yeah so the way that you would prove this is you define what it means to be sorted better to be what it means to be a stable sorting of a particular list, maybe you don't need that. I think you can just define what it means to be sorted and what it means for like two lists to be the same up to permutation and you're like for any list there's a unique list, that is the same up to permutation and also sorted. 

Look both of these sorting methods produce that list. Okay, oh and like this is at the interesting end the like standard end or things like, um, 

If I have a binary tree that holds numbers. And I add one to all the leaves then I take the sum of all the leaves. The number that I get is the number of leaves plus the sum of all the leaves before I added one. 

You know, it's these sort of like trivial structural properties. 

A lot of time is that proving trivial structural properties. I see so so it's not like you like there's like a it's often not like you're missing a concept and understanding how to generate the proof or something but you need like an elegant way to like structure the proof or something and that's the part where you're saying that the beginner would just be like here's tactic. 

I will apply it anywhere like oh what's the good structure to do this or something? I mean, even I'm not what's the good structure to this? I'm like what's what's the structure to do this that isn't wrong? Okay, the beginner is like, Like cargo culting Margo what cargo holding that means? 

I maybe it's originated from Richard Feynman and that. There are some places where the. Ah, like livelihood of the tribes depended on like airplane deliveries of cargo. And. Like there was always a ritual associated with the cargo showing up where you like wave the lights in the air so that the airplane can land on the landing strip. 

And so you can like there there were some cults so I hear I don't know how accurate this is that developed around this where people would waive the lights in the air hoping that this would make the airplanes in the cargo show up. I see right so you can do the same sort of thing with programming you're like, oh I found the program that does the thing. 

I want maybe I can take the code and maybe it'll do the thing. I want. And like you'll also include all the other care and like you're like, why is this other code here? Well, because it was in this other program that did the thing that I want. I don't know if I need it. 

So like why are you doing this proof like this because the example proof that I saw had this structure and it worked through the system was like, okay. Yeah, right. So I am I'm at a more advanced level where I'm like, okay, I know that's a proof things about binary trees. 

I'm gonna go by induction on the binary tree and I'll do something with the number of leaves and I'll figure the rest of the details out as I go. 

And this is how most proofs get done. You do them piecewise. And you like don't account for like how much work call has to do at each step, you're like try it. So work it works then it's good doesn't work that it's bad. And like you could carefully in your head design the entire proof and like carefully account for how much work you expect cock to have to do at each step and make sure that cock shouldn't have to do any work that you don't think it should do. 

But very few people designed proofs like this. But it should be possible to get a proof that is fast if you design it like this and like the next wave of performance issues is going to be that even when you're designing proofs like this things are still not fast enough. 

And so then that's Cox problem. That's the like, how do you design a proof assistant with good enough primitives? Right, yeah, right. I'm like, look you can just. Throughout throw out the prevention. Do this other thing instead works nicely. 

But like it would be nice if you don't have to do that. To get things to scale. 

And yeah, that's that's the sort of next step. Right. Now, that's not like a nicer conclusion. Next step is all these very inspiring. 

\end{subappendices}