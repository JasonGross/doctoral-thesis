% -*-latex-*-
%
% For questions, comments, concerns or complaints:
% thesis@mit.edu
%
%
% $Log: cover.tex,v $
% Revision 1.8  2008/05/13 15:02:15  jdreed
% Degree month is June, not May.  Added note about prevdegrees.
% Arthur Smith's title updated
%
% Revision 1.7  2001/02/08 18:53:16  boojum
% changed some \newpages to \cleardoublepages
%
% Revision 1.6  1999/10/21 14:49:31  boojum
% changed comment referring to documentstyle
%
% Revision 1.5  1999/10/21 14:39:04  boojum
% *** empty log message ***
%
% Revision 1.4  1997/04/18  17:54:10  othomas
% added page numbers on abstract and cover, and made 1 abstract
% page the default rather than 2.  (anne hunter tells me this
% is the new institute standard.)
%
% Revision 1.4  1997/04/18  17:54:10  othomas
% added page numbers on abstract and cover, and made 1 abstract
% page the default rather than 2.  (anne hunter tells me this
% is the new institute standard.)
%
% Revision 1.3  93/05/17  17:06:29  starflt
% Added acknowledgements section (suggested by tompalka)
%
% Revision 1.2  92/04/22  13:13:13  epeisach
% Fixes for 1991 course 6 requirements
% Phrase "and to grant others the right to do so" has been added to
% permission clause
% Second copy of abstract is not counted as separate pages so numbering works
% out
%
% Revision 1.1  92/04/22  13:08:20  epeisach

% NOTE:
% These templates make an effort to conform to the MIT Thesis specifications,
% however the specifications can change.  We recommend that you verify the
% layout of your title page with your thesis advisor and/or the MIT
% Libraries before printing your final copy.
\title{Performance Engineering of Proof-Based Software Systems}
% Alternative (kind-of joke) title: Coq Performance Issues
% Alternative alternative joke title: How To Avoid Hen Performance Issues
% (hen, because people are ``too obsessed with Coq performance'', so we'll have to name our next proof assistant after hens rather than roosters)

\author{Jason S.~Gross}
% If you wish to list your previous degrees on the cover page, use the
% previous degrees command:
%       \prevdegrees{A.A., Harvard University (1985)}
% You can use the \\ command to list multiple previous degrees
%       \prevdegrees{B.S., University of California (1978) \\
%                    S.M., Massachusetts Institute of Technology (1981)}
\department{Department of Electrical Engineering and Computer Science}

% If the thesis is for two degrees simultaneously, list them both
% separated by \and like this:
% \degree{Doctor of Philosophy \and Master of Science}
\degree{Doctor of Philosophy in Computer Science and Engineering}

% As of the 2007-08 academic year, valid degree months are September,
% February, or June.  The default is June.
\degreemonth{June}
\degreeyear{2020}
\thesisdate{(draft)}%August 19, 2015}


%% By default, the thesis will be copyrighted to MIT.  If you need to copyright
%% the thesis to yourself, just specify the `vi' documentclass option.  If for
%% some reason you want to exactly specify the copyright notice text, you can
%% use the \copyrightnoticetext command.
%\copyrightnoticetext{\copyright IBM, 1990.  Do not open till Xmas.}

% If there is more than one supervisor, use the \supervisor command
% once for each.
\supervisor{Adam Chlipala}{Associate Professor of Computer Science}

% This is the department committee chairman, not the thesis committee
% chairman.  You should replace this with your Department's Committee
% Chairman.
\chairman{Leslie A.~Kolodziejski}{Chair, Department Committee on Graduate Students}
\todo{Is the ``department committee chairman'' still Leslie A. Kolodziejski?}

% Make the titlepage based on the above information.  If you need
% something special and can't use the standard form, you can specify
% the exact text of the titlepage yourself.  Put it in a titlepage
% environment and leave blank lines where you want vertical space.
% The spaces will be adjusted to fill the entire page.  The dotted
% lines for the signatures are made with the \signature command.
\maketitle

% The abstractpage environment sets up everything on the page except
% the text itself.  The title and other header material are put at the
% top of the page, and the supervisors are listed at the bottom.  A
% new page is begun both before and after.  Of course, an abstract may
% be more than one page itself.  If you need more control over the
% format of the page, you can use the abstract environment, which puts
% the word "Abstract" at the beginning and single spaces its text.

%% You can either \input (*not* \include) your abstract file, or you can put
%% the text of the abstract directly between the \begin{abstractpage} and
%% \end{abstractpage} commands.

% First copy: start a new page, and save the page number.
\cleardoublepage
% Uncomment the next line if you do NOT want a page number on your
% abstract and acknowledgments pages.
% \pagestyle{empty}
\setcounter{savepage}{\thepage}
\begin{abstractpage}
% $Log: abstract.tex,v $
% Revision 1.1  93/05/14  14:56:25  starflt
% Initial revision
%
% Revision 1.1  90/05/04  10:41:01  lwvanels
% Initial revision
%
%
%% The text of your abstract and nothing else (other than comments) goes here.
%% It will be single-spaced and the rest of the text that is supposed to go on
%% the abstract page will be generated by the abstractpage environment.  This
%% file should be \input (not \include 'd) from cover.tex.
%%
% Quoting
% http://www.eecs.mit.edu/docs/grad/EECS_Thesis_Proposal_and_Thesis_Guidelines.pdf#page=9
%
% After a thesis has been completed, its further value is largely
% dependent on the extent to which it is read and found useful by
% others.  It is important to supply a well-written abstract, which
% outlines the scope and achievements of the thesis so that
% prospective readers can determine whether or not they should read
% any further. An additional advantage is gained because the abstract
% will in many cases enable the library staff to catalogue the work
% more fully and more accurately.  Accordingly, the Committee on
% Graduate Programs requires that each thesis contain an
% abstract--preferably one typewritten page (single-spaced), but in no
% case more than two such pages--in which is given a description of
% the problem and of the procedure used in the investigation, together
% with a brief statement of the results found or of the conclusions
% reached. Other material may be included in the summar y if you find
% it pertinent. Your objective is to inform another engineer or
% scientist, who is not necessarily a specialist in your field, what
% you worked on, how you did it, and what one may expect to learn
% about the problem by reading further
%
% For submission to MIT library: Abstracts should be no longer than
% 350 words, longer abstracts will be edited by ProQuest
%
% https://libraries.mit.edu/distinctive-collections/thesis-specs/#graduate
\todo{fix abstract}
In critical software systems there are opposing pressures to innovate and to let things be as they are.


Formal verification is increasingly valuable as our world comes to rely more on software for critical infrastructure.
A significant and under studied cost of developing mechanized proofs, especially at scale, is the computer performance of proof generation.
This dissertation aims to be a partial guide to identifying and resolving performance bottlenecks in dependently typed tactic-driven proof assistants like Coq.

We present a survey of the landscape of performance issues in Coq, with a number of micro- and macro-benchmarks.
We describe various metrics that allow prediction of performance, such as term size, goal size, and number of binders, and note the occasional surprising lack of a bottleneck for some factors, such as total proof term size.
To our knowledge such a roadmap to performance bottlenecks is a new contribution of this dissertation.

The central new technical contribution presented by this dissertation is a reflective framework for partial evaluation and rewriting, already used to compile a code generator for field-arithmetic cryptographic primitives which generates code currently used in Google Chrome.
We believe this prototype is the first scalably performant realization of an approach for code specialization which does not require adding to the trusted code base.
Our extensible engine, which combines the traditional concepts of tailored term reduction and automatic rewriting from hint databases with on-the-fly generation of inductive codes for constants, is also of interest to replace these ingredients in proof assistants' proof checkers and tactic engines.
Additionally, we use the development of this framework itself as a case study for the various performance issues that can arise when designing large proof libraries.

We identify three main categories of workarounds and partial solutions to performance problems: design of APIs of Gallina libraries; changes to Coq's type theory, implementation, or tooling; and automation design patterns, including proof by reflection.
We present lessons drawn from the case studies of a category-theory library, a proof-producing parser generator, and a verified compiler and code generator for low-level cryptographic primitives.

Finally, we present a novel method of simple and fast reification, developed and published during the course of doctoral study.

\end{abstractpage}

% Additional copy: start a new page, and reset the page number.  This way,
% the second copy of the abstract is not counted as separate pages.
% Uncomment the next 6 lines if you need two copies of the abstract
% page.
% \setcounter{page}{\thesavepage}
% \begin{abstractpage}
% % $Log: abstract.tex,v $
% Revision 1.1  93/05/14  14:56:25  starflt
% Initial revision
%
% Revision 1.1  90/05/04  10:41:01  lwvanels
% Initial revision
%
%
%% The text of your abstract and nothing else (other than comments) goes here.
%% It will be single-spaced and the rest of the text that is supposed to go on
%% the abstract page will be generated by the abstractpage environment.  This
%% file should be \input (not \include 'd) from cover.tex.
%%
% Quoting
% http://www.eecs.mit.edu/docs/grad/EECS_Thesis_Proposal_and_Thesis_Guidelines.pdf#page=9
%
% After a thesis has been completed, its further value is largely
% dependent on the extent to which it is read and found useful by
% others.  It is important to supply a well-written abstract, which
% outlines the scope and achievements of the thesis so that
% prospective readers can determine whether or not they should read
% any further. An additional advantage is gained because the abstract
% will in many cases enable the library staff to catalogue the work
% more fully and more accurately.  Accordingly, the Committee on
% Graduate Programs requires that each thesis contain an
% abstract--preferably one typewritten page (single-spaced), but in no
% case more than two such pages--in which is given a description of
% the problem and of the procedure used in the investigation, together
% with a brief statement of the results found or of the conclusions
% reached. Other material may be included in the summar y if you find
% it pertinent. Your objective is to inform another engineer or
% scientist, who is not necessarily a specialist in your field, what
% you worked on, how you did it, and what one may expect to learn
% about the problem by reading further
%
% For submission to MIT library: Abstracts should be no longer than
% 350 words, longer abstracts will be edited by ProQuest
%
% https://libraries.mit.edu/distinctive-collections/thesis-specs/#graduate
\todo{fix abstract}
In critical software systems there are opposing pressures to innovate and to let things be as they are.


Formal verification is increasingly valuable as our world comes to rely more on software for critical infrastructure.
A significant and under studied cost of developing mechanized proofs, especially at scale, is the computer performance of proof generation.
This dissertation aims to be a partial guide to identifying and resolving performance bottlenecks in dependently typed tactic-driven proof assistants like Coq.

We present a survey of the landscape of performance issues in Coq, with a number of micro- and macro-benchmarks.
We describe various metrics that allow prediction of performance, such as term size, goal size, and number of binders, and note the occasional surprising lack of a bottleneck for some factors, such as total proof term size.
To our knowledge such a roadmap to performance bottlenecks is a new contribution of this dissertation.

The central new technical contribution presented by this dissertation is a reflective framework for partial evaluation and rewriting, already used to compile a code generator for field-arithmetic cryptographic primitives which generates code currently used in Google Chrome.
We believe this prototype is the first scalably performant realization of an approach for code specialization which does not require adding to the trusted code base.
Our extensible engine, which combines the traditional concepts of tailored term reduction and automatic rewriting from hint databases with on-the-fly generation of inductive codes for constants, is also of interest to replace these ingredients in proof assistants' proof checkers and tactic engines.
Additionally, we use the development of this framework itself as a case study for the various performance issues that can arise when designing large proof libraries.

We identify three main categories of workarounds and partial solutions to performance problems: design of APIs of Gallina libraries; changes to Coq's type theory, implementation, or tooling; and automation design patterns, including proof by reflection.
We present lessons drawn from the case studies of a category-theory library, a proof-producing parser generator, and a verified compiler and code generator for low-level cryptographic primitives.

Finally, we present a novel method of simple and fast reification, developed and published during the course of doctoral study.

% \end{abstractpage}

\cleardoublepage

\section*{Acknowledgments}

\todo{uniform style}
\todo{rearrange}
Thank you, Mom, for encouraging me from my youth and supporting me in all that I do.  Last, and most of all, thank you, Adam Chlipala, for your patience, guidance, advice, and wisdom, during the writing of this thesis, and through my research career.
\todo{Add more acknowledgments}
I want to thank Andres Erbsen for pointing out to me some of the particular performance bottlenecks in Coq that I made use of in this thesis, including those of subsubsection \nameref{sec:sharing} in \autoref{sec:sharing} and those of subsections \nameref{sec:name-resolution}, \nameref{sec:perf:capture-avoiding-subst}, \nameref{sec:perf:quadratic-evar-subst}, and \nameref{sec:perf:quadratic-application} in \autoref{sec:perf:binder-count}.
\todo{cite various grants}

% these from category-coq-experience
This work was supported in part by the MIT bigdata@CSAIL initiative, NSF grant CCF-1253229, ONR grant N000141310260, and AFOSR grant FA9550-14-1-0031.
We also thank Benedikt Ahrens, Daniel R.~Grayson, Robert Harper, Bas Spitters, and Edward Z.~Yang for feedback on \citetitle{category-coq-experience}~\cite{category-coq-experience}.

\todo{rearrange}
A significant fraction of the text of this thesis is taken from papers I've co-authored during my PhD, sometimes with major edits, other times with only minor edits to conform to the flow of the thesis.
In particular:
\todo{how to format citations here}
\Autoref{sec:category-theory-library,sec:duality-conversion,sec:associators,sec:equality-reflection,sec:abstraction-barriers} are taken from \citetitle{category-coq-experience}~\cite{category-coq-experience}.

\Autoref{sec:proof-by-reflection,sec:reification-by-parametricity:intro} is based on the introduction to \textcite{reification-by-parametricity}.

%%%%%%%%%%%%%%%%%%%%%%%%%%%%%%%%%%%%%%%%%%%%%%%%%%%%%%%%%%%%%%%%%%%%%%
% -*-latex-*-
