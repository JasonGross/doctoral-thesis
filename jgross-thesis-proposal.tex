%% This file is for producing a Doctoral Thesis proposal.  It should be fairly
%% self-explanatory.

\documentclass[twoside]{article}
\usepackage{hyperref}
\newcommand{\proposaldate}{September 15, 2019}


\begin{document}

\pagestyle{empty}
\markboth{{\sc thesis proposal}}{{\sc thesis proposal}}
\def\title{Performance Engineering of Proof-Based Software Systems}
\def\author{Jason Gross}
\def\addrone{258 Prospect St, \#1L}
\def\addrtwo{Cambridge, MA 02139}

\def\degree{Doctor of Philosophy}
%% Added \deptname for PhD proposals in other fields.
%% Krishna Sethuraman (1990)
\def\deptname{Electrical Engineering \\ and Computer Science}
\def\laboratory{Computer Science and Artificial Intelligence Laboratory}

\def\submissiondate{\proposaldate}
\def\completiondate{May 2020}

\def\supervisor{Professor Adam Chlipala}
\def\supertitleone{Associate Professor of Computer Science}
\def\supertitletwo{}%and Computer Science}

% Suggestions from Adam: Thesis committee: Nickolai And Frans (Coq experience), maybe Saman (compilers producing fast code), Armando (general formal methods), maybe Mike Carbon
\def\readerone{Professor TODO}
\def\readeronetitleone{Professor of Electrical Engineering}
\def\readeronetitletwo{and Computer Science}

\def\readertwo{Professor TODO}
\def\readertwotitleone{Assistant Professor of Electrical Engineering}
\def\readertwotitletwo{and Computer Science}

\def\readerthree{Professor TODO}
\def\readerthreetitleone{Associate Professor of Electrical Engineering}
\def\readerthreetitletwo{and Computer Science}


\def\abstract{% $Log: abstract.tex,v $
% Revision 1.1  93/05/14  14:56:25  starflt
% Initial revision
%
% Revision 1.1  90/05/04  10:41:01  lwvanels
% Initial revision
%
%
%% The text of your abstract and nothing else (other than comments) goes here.
%% It will be single-spaced and the rest of the text that is supposed to go on
%% the abstract page will be generated by the abstractpage environment.  This
%% file should be \input (not \include 'd) from cover.tex.
The proposed research
is a study of processor architecture for
large scale parallel computer systems.
The thesis introduces mechanisms
for fast context switching,
synchronization between tasks,
and run-time binding of
variable names to processor memory.
Various design tradeoffs are evaluated
through simulation of a processor running a typical load.
This work contains
estimates of
the speed and complexity of the different
alternatives as implemented in  VLSI.
}

%%%%%%%%%%%%%%%%%%%%%%%%%%%%%%%%%%%%%%%%%%%%%%%%%%%%%%%%%%%%%%%%%%%%%%%%%%%%
%%%%%%%%%% You Should Not Need To Modify Anything Below Here %%%%%%%%%%%%%%%
%%%%%%%%%%%%%%%%%%%%%%%%%%%%%%%%%%%%%%%%%%%%%%%%%%%%%%%%%%%%%%%%%%%%%%%%%%%%

%%%%%%%%%%%%%%%%%%%%%%%%%%%%%%%%%
%%% Cover Page - Author signs %%%
%%%%%%%%%%%%%%%%%%%%%%%%%%%%%%%%%

\begin{center}
{\Large \bf
   Massachusetts Institute of Technology
\\ Department of \deptname \\}
\vspace{.25in}
{\Large \bf
   Proposal for Thesis Research in Partial Fulfillment
\\ of the Requirements for the Degree of
\\ \degree \\}
\end{center}

\vspace{.5in}

\def\sig{{\small \sc (Signature of Author)}}

\begin{tabular}{rlc}
   {\small \sc Title:}                       & \multicolumn{2}{l}{\title}
\\ {\small \sc Submitted by:}
                            & \author  & \\
                            & \addrone & \\
                            & \addrtwo & \\ \cline{3-3}
			    &	       & \makebox[2in][c]{\sig}
\\ {\small \sc Date of Submission:}          & \multicolumn{2}{l}{\submissiondate}
\\ {\small \sc Expected Date of Completion:} & \multicolumn{2}{l}{\completiondate}
\\ {\small \sc Laboratory:}                  & \multicolumn{2}{l}{\laboratory}
\end{tabular}


\vspace{.75in}
{\bf \sc Brief Statement of the Problem:}

\abstract

                 %%%%%%%%%%%%%%%%%%%%%%%%%%%%%
\cleardoublepage %%% Supervision Agreement %%%
                 %%%%%%%%%%%%%%%%%%%%%%%%%%%%%

\begin{flushright}
   Massachusetts Institute of Technology
\\ Department of \deptname
\\ Cambridge, Massachusetts 02139
\end{flushright}

\underline{\bf Doctoral Thesis Supervision Agreement}

\vspace{.25in}
\begin{tabular}{rl}
   {\small \sc To:}   & Committee on Graduate Students
\\ {\small \sc From:} & \supervisor
\end{tabular}

\vspace{.25in}
The program outlined in the proposal:

\vspace{.25in}
\begin{tabular}{rl}
   {\small \sc Title:}  & \title
\\ {\small \sc Author:} & \author
\\ {\small \sc Date:}   & \submissiondate
\end{tabular}

\vspace{.25in}
is adequate for a Doctoral thesis.
I believe that appropriate readers for this thesis would be:

\vspace{.25in}
\begin{tabular}{rl}
   {\small \sc Reader 1:} & \readerone
\\ {\small \sc Reader 2:} & \readertwo
%\\ {\small \sc Reader 3:} & \readerthree
\end{tabular}

\vspace{.25in}
Facilities and support for the research outlined in the proposal are available.
I am willing to supervise the thesis and evaluate the thesis report.

\vspace{.25in}
\begin{tabular}{crc}
  \hspace{2in} & {\sc Signed:} & \\ \cline{3-3}
               &               & {\small \sc \supertitleone} \\
               &               & {\small \sc \supertitletwo} \\
               &               &                             \\
               & {\sc Date:}   & \\ \cline{3-3}
\end{tabular}

\vspace{0in plus 1fill}

Comments: \\
\begin{tabular}{c}
  \hspace{6.25in} \\
  \mbox{} \\ \cline{1-1} \mbox{} \\
  \mbox{} \\ \cline{1-1} \mbox{} \\
  \mbox{} \\ \cline{1-1} \mbox{} \\
%  \mbox{} \\ \cline{1-1} \mbox{} \\
%  \mbox{} \\ \cline{1-1} \mbox{} \\
%  \mbox{} \\ \cline{1-1} \mbox{} \\
\end{tabular}

                 %%%%%%%%%%%%%%%%%%%%%%%%%%%%%
\cleardoublepage %%% Supervision Agreement %%%
                 %%%%%%%%%%%%%%%%%%%%%%%%%%%%%

\begin{flushright}
   Massachusetts Institute of Technology
\\ Department of \deptname
\\ Cambridge, Massachusetts 02139
\end{flushright}

\underline{\bf Doctoral Thesis Reader Agreement}

\vspace{.25in}
\begin{tabular}{rl}
   {\small \sc To:}   & Committee on Graduate Students
\\ {\small \sc From:} & \readerone
\end{tabular}

\vspace{.25in}
The program outlined in the proposal:

\vspace{.25in}
\begin{tabular}{rl}
   {\small \sc Title:}          & \title
\\ {\small \sc Author:}         & \author
\\ {\small \sc Date:}           & \submissiondate
\\ {\small \sc Supervisor:}     & \supervisor
\\ {\small \sc Other Reader:}   & \readertwo
%\\ {\small \sc Other Reader:}   & \readerthree
\end{tabular}

\vspace{.25in}
is adequate for a Doctoral thesis.
I am willing to aid in guiding the research
and in evaluating the thesis report as a reader.

\vspace{.25in}
\begin{tabular}{crc}
  \hspace{2in} & {\sc Signed:} & \\ \cline{3-3}
               &               & {\small \sc \readeronetitleone} \\
               &               & {\small \sc \readeronetitletwo} \\
               &               &                                 \\
               & {\sc Date:}   & \\ \cline{3-3}
\end{tabular}

\vspace{0in plus 1fill}

Comments: \\
\begin{tabular}{c}
  \hspace{6.25in} \\
  \mbox{} \\ \cline{1-1} \mbox{} \\
  \mbox{} \\ \cline{1-1} \mbox{} \\
  \mbox{} \\ \cline{1-1} \mbox{} \\
  \mbox{} \\ \cline{1-1} \mbox{} \\
  \mbox{} \\ \cline{1-1} \mbox{} \\
  \mbox{} \\ \cline{1-1} \mbox{} \\
\end{tabular}


                  %%%%%%%%%%%%%%%%%%%%%%%%%%%
\cleardoublepage  %%% Reader II Agreement %%%
                  %%%%%%%%%%%%%%%%%%%%%%%%%%%


\begin{flushright}
   Massachusetts Institute of Technology
\\ Department of \deptname
\\ Cambridge, Massachusetts 02139
\end{flushright}

\underline{\bf Doctoral Thesis Reader Agreement}

\vspace{.25in}
\begin{tabular}{rl}
   {\small \sc To:}   & Committee on Graduate Students
\\ {\small \sc From:} & \readertwo
\end{tabular}

\vspace{.25in}
The program outlined in the proposal:

\vspace{.25in}
\begin{tabular}{rl}
   {\small \sc Title:}          & \title
\\ {\small \sc Author:}         & \author
\\ {\small \sc Date:}           & \submissiondate
\\ {\small \sc Supervisor:}     & \supervisor
\\ {\small \sc Other Reader:}   & \readerone
%\\ {\small \sc Other Reader:}   & \readerthree
\end{tabular}

\vspace{.25in}
is adequate for a Doctoral thesis.
I am willing to aid in guiding the research
and in evaluating the thesis report as a reader.

\vspace{.25in}
\begin{tabular}{crc}
  \hspace{2in} & {\sc Signed:} & \\ \cline{3-3}
               &               & {\small \sc \readertwotitleone} \\
               &               & {\small \sc \readertwotitletwo} \\
               &               &                                 \\
               & {\sc Date:}   & \\ \cline{3-3}
\end{tabular}

\vspace{0in plus 1fill}

Comments: \\
\begin{tabular}{c}
  \hspace{6.25in} \\
  \mbox{} \\ \cline{1-1} \mbox{} \\
  \mbox{} \\ \cline{1-1} \mbox{} \\
  \mbox{} \\ \cline{1-1} \mbox{} \\
  \mbox{} \\ \cline{1-1} \mbox{} \\
  \mbox{} \\ \cline{1-1} \mbox{} \\
  \mbox{} \\ \cline{1-1} \mbox{} \\
\end{tabular}

                  %%%%%%%%%%%%%%%%%%%%%%%%%%%%
\cleardoublepage  %%% Reader III Agreement %%%
                  %%%%%%%%%%%%%%%%%%%%%%%%%%%%


\begin{flushright}
   Massachusetts Institute of Technology
\\ Department of \deptname
\\ Cambridge, Massachusetts 02139
\end{flushright}

\underline{\bf Doctoral Thesis Reader Agreement}

\vspace{.25in}
\begin{tabular}{rl}
   {\small \sc To:}   & Committee on Graduate Students
\\ {\small \sc From:} & \readerthree
\end{tabular}

\vspace{.25in}
The program outlined in the proposal:

\vspace{.25in}
\begin{tabular}{rl}
   {\small \sc Title:}          & \title
\\ {\small \sc Author:}         & \author
\\ {\small \sc Date:}           & \submissiondate
\\ {\small \sc Supervisor:}     & \supervisor
\\ {\small \sc Other Reader:}   & \readerone
\\ {\small \sc Other Reader:}   & \readertwo
\end{tabular}

\vspace{.25in}
is adequate for a Doctoral thesis.
I am willing to aid in guiding the research
and in evaluating the thesis report as a reader.

\vspace{.25in}
\begin{tabular}{crc}
  \hspace{2in} & {\sc Signed:} & \\ \cline{3-3}
               &               & {\small \sc \readerthreetitleone} \\
               &               & {\small \sc \readerthreetitletwo} \\
               &               &                                 \\
               & {\sc Date:}   & \\ \cline{3-3}
\end{tabular}

\vspace{0in plus 1fill}

Comments: \\
\begin{tabular}{c}
  \hspace{6.25in} \\
  \mbox{} \\ \cline{1-1} \mbox{} \\
  \mbox{} \\ \cline{1-1} \mbox{} \\
  \mbox{} \\ \cline{1-1} \mbox{} \\
  \mbox{} \\ \cline{1-1} \mbox{} \\
  \mbox{} \\ \cline{1-1} \mbox{} \\
  \mbox{} \\ \cline{1-1} \mbox{} \\
\end{tabular}
\cleardoublepage

\section{Background}

Proof automation is important for verification of large software systems in Coq, as well as for proofs in large-scale mathematical libraries.
Verification is human-developer-hour intensive, and automation allows engineers to construct larger proofs with less per-proof time investment.
In verifying large software systems, automation is necessary for building general-purpose libraries which allow proving things about many different programs without having to write the proofs for each new program from scratch.
In both software and math, automation allows code-reuse and permits factoring out common patterns and proof strategies and permits reasoning at a higher level of abstraction.

Unfortunately, as automation gets more complex or has to deal with more complex goals, the time spent compiling the proof script becomes significant, both repeatedly as the engineer updates their code and works interactively to prove a goal.
The edit-compile-debug loop available when proof automation takes less than a tenth of a second is very, very different from the experience when each tactic takes multiple hours to run.
Moreover, some proof automation simply stops working as the goals get large enough, failing to finish even after hundreds of hours unless carefully tuned to avoid performance issues.

Optimization in Coq is especially hard for a couple of reasons.
Coq has a lot of magic.
In most languages, you optimize the code, and the types are just there to catch mistakes;
in Coq, you have to performance-optimize both the code and the types, simultaneously, whenever dependent types are in use.
In most languages, leaky abstraction barriers cause pain only to the user;
in Coq leaky abstraction barriers also incur a performance overhead every time they are used.

I propose to lay out a map of common Coq performance issues, describe methods and guidelines for working around these performance issues discovered in the course of doing research, and present the research as case-studies.
The primary case studies will be the formalization of category theory in Coq, a rudimentary parser-synthesizer, and the synthesis of efficient verified C code for field arithmetic used in cryptographic primitives.
I will also present a novel way of performing reification, which is a method of constructing abstract syntax trees for terms, used for taking advantage of Coq's computation in employing verified proof procedures or code transformations within the context of larger proofs.
Finally, I will present a verified tool developed in the course of research for replacing proof-producing rewriting and slow custom reduction strategies with a faster reflective method which combines normalization by evaluation with pattern matching complication for reflective rewriting.

\section{Timeline}
\begin{itemize}
  \item
    May 2012--July 2014:
    learn Coq;
    learn Ltac automation;
    formalize some category theory in Coq;
    extract lessons about what causes slowness in math libraries in Coq
  \item
    September 2014--June 2017:
    become acquainted with proof by refinement;
    work on formally verified automatically generated efficient parsers;
    struggle with Coq slowness in software engineering projects
  \item
    February 2016--October 2019:
    work on compiler pipeline for formally verified C code synthesis of modular field arithmetic for cryptographic primitives;
    work on reflective automation
  \item
    $\approx$ November 2017--February 2018:
    discover and codify reification by parametricity
  \item
    $\approx$ November 2017--October 2019:
    work on reflective rewriting framework for Fiat-Crypto;
  \item
    $\approx$ March 2019--October 2019:
    factor out the rewriting framework into its own tool
  \item
    October 22, 2019--November 9, 2019:
    performance testing and evaluation, and write-up, of rewriting framework
  \item
    November 2019--March 2020???:
    thesis-writing
\end{itemize}

\section{The Parts in More Detail}



%IF this is uncommented, update the Makefile
%\bibliographystyle{plain}
%\nocite{*}
%\bibliography{references}

\end{document}
