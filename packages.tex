\usepackage{iftex}
\usepackage{amsmath}
\usepackage{amssymb}
\usepackage{amsthm}
\usepackage{hyperref}
\usepackage{tikz}
\usepackage{pgfplots}
\usepackage{gnuplottex}[2016/08/21]%must be before listings package; min date is for compat with tikzexternalize
\usepackage{pgfkeys}
\usepackage{pgfplotstable}
\pgfplotsset{compat=1.15}
\usepackage{unicode-math}
\usepackage{appendix}
\usepackage{minted}
\usepackage{currfile}
\usepackage[color]{coqdoc}
\usepackage{xparse}
\usepackage{xcolor}
\usepackage{xstring}
\usepackage{bussproofs}
\usepackage{graphicx}
\usepackage{listings}
\usepackage{marvosym}
\usepackage{mathtools}
\usepackage{bcprules}
\usepackage{multicol}
\usepackage{apptools}
\usepackage{adjustbox}
%\usepackage[utf8x]{inputenc}
\usepackage{booktabs}   %% For formal tables:
%% http://ctan.org/pkg/booktabs
\usepackage{float}
\usepackage{wrapfig}
%\usepackage{subcaption} %% For complex figures with subfigures/subcaptions
\usepackage{subfig}
%\usepackage{microtype}
\usepackage{textcomp}
\usepackage{attachfile2}
\usepackage{etoolbox}
\usepackage{xpatch}
\usepackage[maxbibnames=99,backend=biber]{biblatex}% backend=bibtex,
%\usepackage{embedfile}
%\usepackage{hypgotoe}
\floatstyle{boxed}
\restylefloat{figure}
\usepackage{verbatim}
\usepackage{upquote}
\usepackage{enumerate}
\usepackage{cleveref}
\usepackage[all,cmtip]{xy}
\usepackage{stmaryrd}
\usepackage{morefloats}
%\usepackage[autogenerated,mathletters]{ucs}
%Package inputenc Warning: inputenc package ignored with utf8 based engines.
\ifLuaTeX
\else
\usepackage[utf8]{inputenc}
\fi
\usepackage{textgreek}

\makeatletter
\newcommand{\ensuretext}[1]{%
    \ifmmode
    \expandafter\@firstoftwo
    \else
    \expandafter\@secondoftwo
    \fi
    {\text{#1}}%
    {#1}%
}
\newcommand{\ensuretexttt}[1]{\ensuretext{\texttt{#1}}}
\makeatother

\usepackage{newunicodechar}
\newcommand{\newunicodecharpdftex}[2]{\newunicodechar{#1}{\texorpdfstring{#2}{#1}}}
\newunicodecharpdftex{Α}{\ensuremath{\Alpha}}
\newunicodecharpdftex{Β}{\ensuremath{\Beta}}
\newunicodecharpdftex{Γ}{\ensuremath{\Gamma}}
\newunicodecharpdftex{Δ}{\ensuremath{\Delta}}
\newunicodecharpdftex{Ε}{\ensuremath{\Epsilon}}
\newunicodecharpdftex{Ζ}{\ensuremath{\Zeta}}
\newunicodecharpdftex{Η}{\ensuremath{\Eta}}
\newunicodecharpdftex{Θ}{\ensuremath{\Theta}}
\newunicodecharpdftex{Ι}{\ensuremath{\Iota}}
\newunicodecharpdftex{Κ}{\ensuremath{\Kappa}}
\newunicodecharpdftex{Λ}{\ensuremath{\Lambda}}
\newunicodecharpdftex{Μ}{\ensuremath{\Mu}}
\newunicodecharpdftex{Ν}{\ensuremath{\Nu}}
\newunicodecharpdftex{Ξ}{\ensuremath{\Xi}}
\newunicodecharpdftex{Ο}{\ensuremath{\Omicron}}
\newunicodecharpdftex{Π}{\ensuremath{\Pi}}
\newunicodecharpdftex{Ρ}{\ensuremath{\Rho}}
\newunicodecharpdftex{Σ}{\ensuremath{\Sigma}}
\newunicodecharpdftex{Τ}{\ensuremath{\Tau}}
\newunicodecharpdftex{Υ}{\ensuremath{\Upsilon}}
\newunicodecharpdftex{Φ}{\ensuremath{\Phi}}
\newunicodecharpdftex{Χ}{\ensuremath{\Chi}}
\newunicodecharpdftex{Ψ}{\ensuremath{\Psi}}
\newunicodecharpdftex{Ω}{\ensuremath{\Omega}}
\newunicodecharpdftex{α}{\ensuremath{\alpha}}
\newunicodecharpdftex{β}{\ensuremath{\beta}}
\newunicodecharpdftex{γ}{\ensuremath{\gamma}}
\newunicodecharpdftex{δ}{\ensuremath{\delta}}
\newunicodecharpdftex{ϵ}{\ensuremath{\epsilon}}
\newunicodecharpdftex{ε}{\ensuremath{\varepsilon}}
\newunicodecharpdftex{ζ}{\ensuremath{\zeta}}
\newunicodecharpdftex{η}{\ensuremath{\eta}}
\newunicodecharpdftex{θ}{\ensuremath{\theta}}
\newunicodecharpdftex{ϑ}{\ensuremath{\vartheta}}
\newunicodecharpdftex{ι}{\ensuremath{\iota}}
\newunicodecharpdftex{κ}{\ensuremath{\kappa}}
\newunicodecharpdftex{ϰ}{\ensuremath{\varkappa}}
\newunicodecharpdftex{λ}{\ensuremath{\lambda}}
\newunicodecharpdftex{μ}{\ensuremath{\mu}}
\newunicodecharpdftex{ν}{\ensuremath{\nu}}
\newunicodecharpdftex{ξ}{\ensuremath{\xi}}
\newunicodecharpdftex{ο}{\ensuremath{\omicron}}
\newunicodecharpdftex{π}{\ensuremath{\pi}}
\newunicodecharpdftex{ϖ}{\ensuremath{\varpi}}
\newunicodecharpdftex{ρ}{\ensuremath{\rho}}
\newunicodecharpdftex{ϱ}{\ensuremath{\varrho}}
\newunicodecharpdftex{σ}{\ensuremath{\sigma}}
\newunicodecharpdftex{ς}{\ensuremath{\varsigma}}
\newunicodecharpdftex{τ}{\ensuremath{\tau}}
\newunicodecharpdftex{υ}{\ensuremath{\upsilon}}
\newunicodecharpdftex{ϕ}{\ensuremath{\phi}}
\newunicodecharpdftex{φ}{\ensuremath{\varphi}}
\newunicodecharpdftex{χ}{\ensuremath{\chi}}
\newunicodecharpdftex{ψ}{\ensuremath{\psi}}
\newunicodecharpdftex{ω}{\ensuremath{\omega}}
\newunicodecharpdftex{∀}{\ensuremath{\forall}}
\newunicodecharpdftex{∃}{\ensuremath{\exists}}
\newunicodecharpdftex{→}{\ensuremath{\to}}
\newunicodecharpdftex{⇒}{\ensuremath{\Rightarrow}}
\newunicodecharpdftex{×}{\ensuremath{\times}}
\newunicodecharpdftex{∧}{\ensuremath{\wedge}}
\newunicodecharpdftex{⊢}{\ensuremath{\vdash}}
\newunicodecharpdftex{𝔹}{\ensuremath{\mathbb{B}}}
\newunicodecharpdftex{ℂ}{\ensuremath{\mathbb{C}}}
\newunicodecharpdftex{ℕ}{\ensuremath{\mathbb{N}}}
\newunicodecharpdftex{ℙ}{\ensuremath{\mathbb{P}}}
\newunicodecharpdftex{ℚ}{\ensuremath{\mathbb{Q}}}
\newunicodecharpdftex{ℝ}{\ensuremath{\mathbb{R}}}
\newunicodecharpdftex{ℤ}{\ensuremath{\mathbb{Z}}}
\newunicodecharpdftex{⁰}{\ensuremath{{}^0}}
\newunicodecharpdftex{¹}{\ensuremath{{}^1}}
\newunicodecharpdftex{²}{\ensuremath{{}^2}}
\newunicodecharpdftex{³}{\ensuremath{{}^3}}
\newunicodecharpdftex{⁴}{\ensuremath{{}^4}}
\newunicodecharpdftex{⁵}{\ensuremath{{}^5}}
\newunicodecharpdftex{⁶}{\ensuremath{{}^6}}
\newunicodecharpdftex{⁷}{\ensuremath{{}^7}}
\newunicodecharpdftex{⁸}{\ensuremath{{}^8}}
\newunicodecharpdftex{⁹}{\ensuremath{{}^9}}
\newunicodecharpdftex{₀}{\ensuremath{{}_0}}
\newunicodecharpdftex{₁}{\ensuremath{{}_1}}
\newunicodecharpdftex{₂}{\ensuremath{{}_2}}
\newunicodecharpdftex{₃}{\ensuremath{{}_3}}
\newunicodecharpdftex{₄}{\ensuremath{{}_4}}
\newunicodecharpdftex{₅}{\ensuremath{{}_5}}
\newunicodecharpdftex{₆}{\ensuremath{{}_6}}
\newunicodecharpdftex{₇}{\ensuremath{{}_7}}
\newunicodecharpdftex{₈}{\ensuremath{{}_8}}
\newunicodecharpdftex{₉}{\ensuremath{{}_9}}
\newunicodecharpdftex{⁺}{\ensuremath{{}^+}}
\newunicodecharpdftex{⁻}{\ensuremath{{}^-}}
\newunicodecharpdftex{⁼}{\ensuremath{{}^=}}
\newunicodecharpdftex{⁽}{\ensuremath{{}^(}}
\newunicodecharpdftex{⁾}{\ensuremath{{}^)}}
\newunicodecharpdftex{₊}{\ensuremath{{}_+}}
\newunicodecharpdftex{₋}{\ensuremath{{}_-}}
\newunicodecharpdftex{₌}{\ensuremath{{}_=}}
\newunicodecharpdftex{₍}{\ensuremath{{}_(}}
\newunicodecharpdftex{₎}{\ensuremath{{}_)}}

\newunicodecharpdftex{ᵃ}{\ensuremath{{}^{\text{a}}}}
\newunicodecharpdftex{ᵇ}{\ensuremath{{}^{\text{b}}}}
\newunicodecharpdftex{ᶜ}{\ensuremath{{}^{\text{c}}}}
\newunicodecharpdftex{ᵈ}{\ensuremath{{}^{\text{d}}}}
\newunicodecharpdftex{ᵉ}{\ensuremath{{}^{\text{e}}}}
\newunicodecharpdftex{ᶠ}{\ensuremath{{}^{\text{f}}}}
\newunicodecharpdftex{ᵍ}{\ensuremath{{}^{\text{g}}}}
\newunicodecharpdftex{ʰ}{\ensuremath{{}^{\text{h}}}}
\newunicodecharpdftex{ⁱ}{\ensuremath{{}^{\text{i}}}}
\newunicodecharpdftex{ʲ}{\ensuremath{{}^{\text{j}}}}
\newunicodecharpdftex{ᵏ}{\ensuremath{{}^{\text{k}}}}
\newunicodecharpdftex{ˡ}{\ensuremath{{}^{\text{l}}}}
\newunicodecharpdftex{ᵐ}{\ensuremath{{}^{\text{m}}}}
\newunicodecharpdftex{ⁿ}{\ensuremath{{}^{\text{n}}}}
\newunicodecharpdftex{ᵒ}{\ensuremath{{}^{\text{o}}}}
\newunicodecharpdftex{ᵖ}{\ensuremath{{}^{\text{p}}}}
\newunicodecharpdftex{ʳ}{\ensuremath{{}^{\text{r}}}}
\newunicodecharpdftex{ˢ}{\ensuremath{{}^{\text{s}}}}
\newunicodecharpdftex{ᵗ}{\ensuremath{{}^{\text{t}}}}
\newunicodecharpdftex{ᵘ}{\ensuremath{{}^{\text{u}}}}
\newunicodecharpdftex{ᵛ}{\ensuremath{{}^{\text{v}}}}
\newunicodecharpdftex{ʷ}{\ensuremath{{}^{\text{w}}}}
\newunicodecharpdftex{ˣ}{\ensuremath{{}^{\text{x}}}}
\newunicodecharpdftex{ʸ}{\ensuremath{{}^{\text{y}}}}
\newunicodecharpdftex{ᶻ}{\ensuremath{{}^{\text{z}}}}
\newunicodecharpdftex{ₐ}{\ensuremath{{}_{\text{a}}}}
\newunicodecharpdftex{ₑ}{\ensuremath{{}_{\text{e}}}}
\newunicodecharpdftex{ₕ}{\ensuremath{{}_{\text{h}}}}
\newunicodecharpdftex{ᵢ}{\ensuremath{{}_{\text{i}}}}
\newunicodecharpdftex{ⱼ}{\ensuremath{{}_{\text{j}}}}
\newunicodecharpdftex{ₖ}{\ensuremath{{}_{\text{k}}}}
\newunicodecharpdftex{ₗ}{\ensuremath{{}_{\text{l}}}}
\newunicodecharpdftex{ₘ}{\ensuremath{{}_{\text{m}}}}
\newunicodecharpdftex{ₙ}{\ensuremath{{}_{\text{n}}}}
\newunicodecharpdftex{ₒ}{\ensuremath{{}_{\text{o}}}}
\newunicodecharpdftex{ₚ}{\ensuremath{{}_{\text{p}}}}
\newunicodecharpdftex{ᵣ}{\ensuremath{{}_{\text{r}}}}
\newunicodecharpdftex{ₛ}{\ensuremath{{}_{\text{s}}}}
\newunicodecharpdftex{ₜ}{\ensuremath{{}_{\text{t}}}}
\newunicodecharpdftex{ᵤ}{\ensuremath{{}_{\text{u}}}}
\newunicodecharpdftex{ᵥ}{\ensuremath{{}_{\text{v}}}}
\newunicodecharpdftex{ₓ}{\ensuremath{{}_{\text{x}}}}

\newunicodecharpdftex{≅}{\ensuremath{\cong}}
\newunicodecharpdftex{∘}{\ensuremath{\circ}}

\usepgfplotslibrary{units}
\usepgfplotslibrary{external}
\tikzexternalize
\tikzsetfigurename{\tikzexternalrealjob-\currfilebase-figure}

\hypersetup{raiselinks=false,colorlinks=false,linkcolor=black} % undo coqdoc stuff

% Minted red box around greek characters - https://tex.stackexchange.com/a/343506/2066
\usemintedstyle{bw}
\makeatletter
\AtBeginEnvironment{minted}{\dontdofcolorbox}
\def\dontdofcolorbox{\renewcommand\fcolorbox[4][]{##4}}
\xpatchcmd{\inputminted}{\minted@fvset}{\minted@fvset\dontdofcolorbox}{}{}
\xpatchcmd{\mintinline}{\minted@fvset}{\minted@fvset\dontdofcolorbox}{}{}
\makeatother

\renewcommand{\chapterautorefname}{Chapter}
\renewcommand{\sectionautorefname}{Section}
\renewcommand{\subsectionautorefname}{Subsection}
\renewcommand{\subsubsectionautorefname}{Subsubsection}
\newcommand{\appendixautoref}[1]{\bgroup
\def\chapterautorefname{Appendix}%
\def\sectionautorefname{Appendix}%
\def\subsectionautorefname{Appendix}%
\def\subsubsectionautorefname{Appendix}%
\newcommand{\subfigureautorefname}{\figureautorefname}
\autoref{#1}%
\egroup}

%{\catcode`\_=12 \gdef\textunderscore{_}}
%\def\_{\textunderscore}

% from http://tex.stackexchange.com/a/218441/2066
\makeatletter
\newcommand{\dashover}[2][\mathop]{#1{\mathpalette\df@over{{\dashfill}{#2}}}}
\newcommand{\fillover}[2][\mathop]{#1{\mathpalette\df@over{{\solidfill}{#2}}}}
\newcommand{\df@over}[2]{\df@@over#1#2}
\newcommand\df@@over[3]{%
  \vbox{
    \offinterlineskip
    \ialign{##\cr
      #2{#1}\cr
      \noalign{\kern1pt}
      $\m@th#1#3$\cr
    }
  }%
}
\newcommand{\dashfill}[1]{%
  \kern-.5pt
  \xleaders\hbox{\kern.5pt\vrule height.4pt width \dash@width{#1}\kern.5pt}\hfill
  \kern-.5pt
}
\newcommand{\dash@width}[1]{%
  \ifx#1\displaystyle
    2pt
  \else
    \ifx#1\textstyle
      1.5pt
    \else
      \ifx#1\scriptstyle
        1.25pt
      \else
        \ifx#1\scriptscriptstyle
          1pt
        \fi
      \fi
    \fi
  \fi
}
\newcommand{\solidfill}[1]{\leaders\hrule\hfill}
\makeatother

\newcommand{\aswidthof}[2]{\rlap{#1}\hphantom{#2}}

\newcommand{\coqbug}[1]{\href{https://github.com/coq/coq/issues/#1}{\##1}}
\newcommand{\coqpr}[1]{\href{https://github.com/coq/coq/issues/#1}{\##1}}
\newcommand{\ocamlbug}[1]{\href{https://github.com/ocaml/ocmal/issues/#1}{\##1}}
\newcommand{\kw}[1]{{\fontfamily{pcr}\selectfont\textbf{#1}}}
\newcommand{\tactic}[1]{\texttt{#1}}
\newcommand{\finishingtactic}[1]{\tactic{#1}}
\newcommand{\tacREFLEXIVITY}{\finishingtactic{reflexivity}}
\newcommand{\strcolored}[1]{\texttt{\textcolor{DarkGreen}{#1}}}
\newcommand{\stropen}{\texttt{"}}
\newcommand{\strclose}{\texttt{"}}
\newcommand{\str}[1]{\stropen\strcolored{#1}\strclose}
\newcommand{\regex}[1]{\texttt{#1}}
\newcommand{\nt}[1]{\texttt{#1}}
\newcommand{\terminal}[1]{\texttt{\textquotesingle\textcolor{DarkGreen}{#1}\textquotesingle}}
\newcommand{\production}[1]{[#1]}
\newcommand{\productions}[1]{#1}
\newcommand{\coqtype}[1]{\texttt{#1}}
\newcommand{\False}{\ensuremath{\bot}}
\newcommand{\True}{\ensuremath{\top}}
\newcommand{\Unit}{\ensuremath{\top}}
\newcommand{\true}{\texttt{true}}
\newcommand{\false}{\texttt{false}}
\newcommand{\unittt}{\texttt{()}}
\newcommand{\String}{\texttt{String}}
\newcommand{\Bool}{\texttt{Bool}}
\newcommand{\nat}{\ensuremath{\mathbb{N}}}
\newcommand{\textnbsp}{\ifmmode\else~\fi}
\newcommand{\typeprodsep}{\ensuremath{\times}}
\newcommand{\typeprod}[2]{#1\textnbsp\typeprodsep\textnbsp#2}
\newcommand{\typesumsep}{\ensuremath{+}}
\newcommand{\typesum}[2]{#1\textnbsp\typesumsep\textnbsp#2}
\newcommand{\fname}[1]{\texttt{#1}}
\newcommand{\farg}[1]{\textcolor{violet}{\texttt{#1}}}
\newcommand{\oftypesep}{:}
\newcommand{\oftype}[2]{#1\textnbsp\oftypesep\textnbsp#2}
\newcommand{\nil}{\texttt{[]}}
\newcommand{\cons}[2]{#1::#2}
%\newcommand{\hole}{\texttt{\_}}
\newcommand{\termhole}{\texttt{\_}}
\newcommand{\defeq}{\coloneqq}
\newcommand{\testeq}{=}
\newcommand{\booland}{\mathrel{\texttt{\&\&}}}
\newcommand{\boolor}{\mathrel{\texttt{||}}}
\newcommand{\letinlet}{\kw{let}~}
\newcommand{\letinkw}[1]{\letinlet#1~\kw{in}}
\newcommand{\strcat}[2]{#1#2}
\newcommand{\llstrcat}[2]{#1 \ensuremath{\cdot} #2}
\newcommand{\afun}[2]{\ensuremath{\lambda~#1.~#2}}
\newcommand{\typeto}{\ensuremath{\to}}
\newcommand{\indname}[1]{\texttt{#1}}
\newcommand{\Cat}{\kw{Cat}}
\newcommand{\Type}{\kw{Type}}
\newcommand{\Set}{\kw{Set}}
\newcommand{\Prop}{\kw{Prop}}
\newcommand{\constructorname}[1]{\texttt{#1}}
\newcommand{\Nonterminal}{\indname{Nonterminal}}
\newcommand{\Terminal}{\texttt{Char}}
\newcommand{\parsetreetype}[2]{\ensuremath{\dashover[]{#2 \in #1}}}
\newcommand{\minparsetreeannot}[2]{\ensuremath{<(#1,#2)}}
\newcommand{\minparsetreetype}[4]{\ensuremath{\dashover[]{#4 \in #3}^{~\minparsetreeannot{#1}{#2}}}}
\newcommand{\typelist}[1]{\texttt{[}#1\texttt{]}}
\newcommand{\typelistp}[1]{\typelist{#1}} % parenthesized \typelist, if need be (so, `list (foo)` instead of `list foo`)
\newcommand{\valuelist}[1]{\texttt{[}#1\texttt{]}}
\newcommand{\valuelistm}[1]{\ensuremath{\big[#1\big]}} % math mode
\newcommand{\typeoption}[2][~~]{\indname{option}#1#2}
\newcommand{\typeoptionp}[1]{\typeoption[~]{(#1)}} % parenthesized \typeoption, if need be
\newcommand{\None}{\constructorname{None}}
\newcommand{\Some}[1]{\constructorname{Some}~#1}
\newcommand{\Somep}[1]{\constructorname{Some}~(#1)}
\newcommand{\ParseQuery}{\indname{ParseQuery}}
\newcommand{\inl}[1]{\constructorname{inl}~#1}
\newcommand{\inlp}[1]{\inl{(#1)}}
\newcommand{\inr}[1]{\constructorname{inr}~#1}
\newcommand{\inrp}[1]{\inr{(#1)}}
\newcommand{\fst}[1]{\texttt{fst}~#1}
\newcommand{\snd}[1]{\texttt{snd}~#1}
\newcommand{\proj}[2]{#2\ensuremath{_{\kw{#1}}}}
\newcommand{\precaseof}{\kw{case}~~~}
\newcommand{\caseof}[1]{\precaseof#1~~~\kw{of}}
\newcommand{\acase}[3][\big]{~#1|{~{#2}~~\ensuremath{\to}~~{#3}}}
\newcommand{\cif}{\kw{if}~~~}
\newcommand{\cifindent}[1][]{\aswidthof{#1}{\kw{then}}~~~}
\newcommand{\cthen}{\cifindent[\kw{then}]}
\newcommand{\celse}{\cifindent[\kw{else}]}
\newcommand{\cancomputeto}[2]{#1~~\ensuremath{\rightsquigarrow}~~#2}
\newcommand{\Rtac}{\texorpdfstring{\ensuremath{\mathcal{R}_{\text{\textit{tac}}}}}{Rtac}}
\newcommand{\Ltac}{\texorpdfstring{\ensuremath{\mathcal{L}_{\text{\textit{tac}}}}}{Ltac}}
\newcommand{\LtacTwo}{Ltac2}
\newcommand{\coqgroupb}[1]{\texttt{\ensuremath{\left(\text{#1}\right)}}}
\newcommand{\coqgroup}[1]{\texttt{(#1)}}
\newcommand{\functiondefeq}{~\ensuremath{\operatorname{\texttt{\ensuremath{\defeq}}}}~}

% highlight overfull hboxes
%\overfullrule=5pt


% http://tex.stackexchange.com/a/183682/2066
\makeatletter

% define a macro \Autoref to allow multiple references to be passed to \autoref
\newcommand\Autoref[1]{\@first@ref#1,@}
\def\@throw@dot#1.#2@{#1}% discard everything after the dot
\def\@set@refname#1{%    % set \@refname to autoefname+s using \getrefbykeydefault
    \edef\@tmp{\getrefbykeydefault{#1}{anchor}{}}%
    \def\@refname{\@nameuse{\expandafter\@throw@dot\@tmp.@autorefname}s}%
}
\def\@first@ref#1,#2{%
  \ifx#2@\autoref{#1}\let\@nextref\@gobble% only one ref, revert to normal \autoref
  \else%
    \@set@refname{#1}%  set \@refname to autoref name
    \@refname~\ref{#1}% add autoefname and first reference
    \let\@nextref\@next@ref% push processing to \@next@ref
  \fi%
  \@nextref#2%
}
\def\@next@ref#1,#2{%
   \ifx#2@ and~\ref{#1}\let\@nextref\@gobble% at end: print and+\ref and stop
   \else, \ref{#1}% print  ,+\ref and continue
   \fi%
   \@nextref#2%
}

\makeatother
\allowdisplaybreaks

% from reification-by-parametricity


% begin appendix autoref patch [\autoref subsections in appendix](https://tex.stackexchange.com/questions/149807/autoref-subsections-in-appendix)
\makeatletter
\patchcmd{\hyper@makecurrent}{%
    \ifx\Hy@param\Hy@chapterstring
    \let\Hy@param\Hy@chapapp
    \fi
}{%
\iftoggle{inappendix}{%true-branch
    % list the names of all sectioning counters here
    \@checkappendixparam{chapter}%
    \@checkappendixparam{section}%
    \@checkappendixparam{subsection}%
    \@checkappendixparam{subsubsection}%
    \@checkappendixparam{paragraph}%
    \@checkappendixparam{subparagraph}%
}{}%
}{}{\errmessage{failed to patch}}

\newcommand*{\@checkappendixparam}[1]{%
    \def\@checkappendixparamtmp{#1}%
    \ifx\Hy@param\@checkappendixparamtmp
    \let\Hy@param\Hy@appendixstring
    \fi
}
\makeatletter

\newtoggle{inappendix}
\togglefalse{inappendix}

\apptocmd{\appendix}{\toggletrue{inappendix}}{}{\errmessage{failed to patch}}
%\apptocmd{\subappendices}{\toggletrue{inappendix}}{}{\errmessage{failed to patch}}
% end appendix autoref patch


\newcommand{\letindots}{\texttt{let \ldots\space in \ldots}}
\newcommand{\letin}[1][{\ensuremath{\cdots}}{\ensuremath{\cdots}}]{%
    \texttt{let }\@firstoftwo#1\texttt{ in }\@secondoftwo#1
}


\newcommand{\ttifycoqdoc}[1]{%
\expandafter\let\csname old\string#1\endcsname#1%
\renewcommand{#1}[1]{{%
\let\textrm\texttt
\let\textsf\texttt
\let\textit\texttt
\let\textsl\texttt
\csname old\string#1\endcsname{##1}%
}}%
}

\ttifycoqdoc\coqdocvar
\ttifycoqdoc\coqdoccst
\ttifycoqdoc\coqdocind
\ttifycoqdoc\coqdocconstr
\ttifycoqdoc\coqdocmod
\ttifycoqdoc\coqdocax

\makeatletter
\renewenvironment{coqdoccode}{\tt\@vobeyspaces}{}
\makeatother

\newcommand{\coqinput}[1]{{%
\let\oldsubsection\subsection
\renewcommand{\section}[1]{\oldsubsection{##1 (\texttt{#1})} \label{sec:#1}}%
\renewcommand{\subsection}[1]{\subsubsection{##1} \hfill$\left.\right.$ \linebreak}%
\input{#1.tex}%
}}
\newcommand{\subcoqinput}[1]{{%
\renewcommand{\section}[1]{\subsubsection{##1 (\texttt{#1})} \label{sec:#1}\hfill$\left.\right.$ \linebreak}%
\renewcommand{\subsection}[1]{\paragraph{##1} \hfill$\left.\right.$ \linebreak}%
\input{#1.tex}%
}}
\newcommand{\ocamlinput}[2]{{%
\subsection{#1 (\texttt{\detokenize{#2}})} \label{sec:#2}
\verbatiminput{benchmark/#2}%
}}
\newcommand{\subocamlinput}[2]{{%
\subsubsection{#1 (\texttt{\detokenize{#2}})} \label{sec:#2}\hfill$\left.\right.$ \linebreak
\verbatiminput{benchmark/#2}%
}}

% https://tex.stackexchange.com/a/121871/2066
\newcommand*{\fullref}[1]{\hyperref[{#1}]{\autoref*{#1} (\nameref*{#1})}} % One single link

% https://groups.google.com/d/msg/latexusersgroup/07qZP63GY3k/cu53DqbAhkYJ
\makeatletter
\newcommand{\Chapter}[1]{\split@chapter#1:}
\def\split@chapter#1:#2:{\chapter[#1]{#1\\[1ex]\huge#2}}
\makeatother



% from rewriting

%\usepackage{usebib}
%\newbibfield{author}
%\newbibfield{title}
%\bibinput{rewriting-lower}
%\newcommand{\citetitle}[1]{\emph{\usebibentry{#1}{title}}\nocite{#1}}
\makeatletter
\newcommand{\einput}[1]{\@@input #1 \space}
\newcommand{\beginTikzpictureStamped}[2][]{%
    {%
        \everyeof{\noexpand}% IDK why \noexpand is the magic one, but I got it from http://mirrors.ibiblio.org/CTAN/macros/latex/contrib/oberdiek/catchfile.pdf
        \long\xdef\@tikzstamp{#2}%
    }%
    \def\@dobegintikzpicture{\begin{tikzpicture}[#1]}%
    \expandafter\@dobegintikzpicture\expandafter\def\expandafter\tikzstamp\expandafter{\@tikzstamp}%
}
\newcommand{\TikzpictureStamped}[3][]{\beginTikzpictureStamped[#1]{#2}#3\end{tikzpicture}}%
\makeatother
\newcommand{\asserteq}[2]{\ifthenelse{\equal{#1}{#2}}{}{\GenericError{}{Not equal: \detokenize{#1} != \detokenize{#2}}{}{}}}

%% from pandoc
\providecommand{\tightlist}{%
    \setlength{\itemsep}{0pt}\setlength{\parskip}{0pt}}

%% exponential regression fit
\makeatletter
% https://tex.stackexchange.com/a/50113/2066
\newcommand*{\IsInteger}[3]{%
    \IfStrEq{#1}{ }{%
        #3% is a blank string
    }{%
    \IfInteger{#1}{#2}{#3}%
}%
}%
\newcommand{\pgftognucolumnset}[2]{%
    \IsInteger{\pgfkeysvalueof{#1}}{%
        % pgf 0-indexes columns, while gnuplot 1-indexes columns, so we add 1 to adjust
        \edef#2{\the\numexpr\pgfkeysvalueof{#1}+1\relax}%
    }{%
    \edef#2{(column("\pgfkeysvalueof{#1}"))}%
}%
}
\makeatletter
% \addplotexponentialregression[params for \addplot][default settings for a and b, also for x and y columns]{table file}
\NewDocumentCommand{\addplotexponentialregression}{ O{no markers} o m}{%
    \pgfkeyssetvalue{/addplotexponentialregression/x}{0}
    \pgfkeyssetvalue{/addplotexponentialregression/y}{1}
    \pgfkeyssetvalue{/addplotexponentialregression/a}{1}
    \pgfkeyssetvalue{/addplotexponentialregression/b}{1}
    \pgfkeys{/addplotexponentialregression/.cd,#2}
    \pgftognucolumnset{/addplotexponentialregression/x}{\@addplotexponentialregression@colx}%
    \pgftognucolumnset{/addplotexponentialregression/y}{\@addplotexponentialregression@coly}%
    \edef\@addplotexponentialregression@inita{\pgfkeysvalueof{/addplotexponentialregression/a}}%
    \edef\@addplotexponentialregression@initb{\pgfkeysvalueof{/addplotexponentialregression/b}}%
    \addplot [#1] gnuplot [raw gnuplot] { % allows arbitrary gnuplot commands
        f(x) = a*exp(b*x);     % Define the function to fit
        % Set reasonable starting values here
        a=\@addplotexponentialregression@inita;
        b=\@addplotexponentialregression@initb;
        fit f(x) '#3' u \@addplotexponentialregression@colx:\@addplotexponentialregression@coly\space via a,b; % Select the file
        stats '#3' u \@addplotexponentialregression@colx;
        plot [x=STATS_min:STATS_max] f(x);    % Specify the range to plot
        set print "#3-parameters.dat";  % Open a file to save the parameters
        print a, b;                  % Write the parameters to file
    };
    \pgfplotstableread{#3-parameters.dat}\parameters % Open the file Gnuplot wrote
    \pgfplotstablegetelem{0}{0}\of\parameters \xdef\pgfplotstableregressiona{\pgfplotsretval} % Get first element, save into \pgfplotstableregressiona
    \pgfplotstablegetelem{0}{1}\of\parameters \xdef\pgfplotstableregressionb{\pgfplotsretval}
}
% \addplotquadraticregression[params for \addplot][default settings for a and b and c, also for x and y columns]{table file}
\NewDocumentCommand{\addplotquadraticregression}{ O{no markers} o m}{%
    \pgfkeyssetvalue{/addplotquadraticregression/x}{0}
    \pgfkeyssetvalue{/addplotquadraticregression/y}{1}
    \pgfkeyssetvalue{/addplotquadraticregression/a}{1}
    \pgfkeyssetvalue{/addplotquadraticregression/b}{1}
    \pgfkeyssetvalue{/addplotquadraticregression/c}{1}
    \pgfkeys{/addplotquadraticregression/.cd,#2}
    \pgftognucolumnset{/addplotquadraticregression/x}{\@addplotquadraticregression@colx}%
    \pgftognucolumnset{/addplotquadraticregression/y}{\@addplotquadraticregression@coly}%
    \edef\@addplotquadraticregression@inita{\pgfkeysvalueof{/addplotquadraticregression/a}}%
    \edef\@addplotquadraticregression@initb{\pgfkeysvalueof{/addplotquadraticregression/b}}%
    \edef\@addplotquadraticregression@initc{\pgfkeysvalueof{/addplotquadraticregression/c}}%
    \addplot [#1] gnuplot [raw gnuplot] { % allows arbitrary gnuplot commands
        f(x) = a*x**2+b*x+c;     % Define the function to fit
        % Set reasonable starting values here
        a=\@addplotquadraticregression@inita;
        b=\@addplotquadraticregression@initb;
        c=\@addplotquadraticregression@initc;
        fit f(x) '#3' u \@addplotquadraticregression@colx:\@addplotquadraticregression@coly\space via a,b,c; % Select the file
        stats '#3' u \@addplotquadraticregression@colx;
        plot [x=STATS_min:STATS_max] f(x);    % Specify the range to plot
        set print "#3-parameters.dat";  % Open a file to save the parameters
        print a, b, c;                  % Write the parameters to file
    };
    \pgfplotstableread{#3-parameters.dat}\parameters % Open the file Gnuplot wrote
    \pgfplotstablegetelem{0}{0}\of\parameters \xdef\pgfplotstableregressiona{\pgfplotsretval} % Get first element, save into \pgfplotstableregressiona
    \pgfplotstablegetelem{0}{1}\of\parameters \xdef\pgfplotstableregressionb{\pgfplotsretval}
    \pgfplotstablegetelem{0}{2}\of\parameters \xdef\pgfplotstableregressionc{\pgfplotsretval}
}
% \addplotcubicregression[params for \addplot][default settings for a and b and c and d, also for x and y columns]{table file}
\NewDocumentCommand{\addplotcubicregression}{ O{no markers} o m}{%
    \pgfkeyssetvalue{/addplotcubicregression/x}{0}
    \pgfkeyssetvalue{/addplotcubicregression/y}{1}
    \pgfkeyssetvalue{/addplotcubicregression/a}{1}
    \pgfkeyssetvalue{/addplotcubicregression/b}{1}
    \pgfkeyssetvalue{/addplotcubicregression/c}{1}
    \pgfkeyssetvalue{/addplotcubicregression/d}{1}
    \pgfkeys{/addplotcubicregression/.cd,#2}
    \pgftognucolumnset{/addplotcubicregression/x}{\@addplotcubicregression@colx}%
    \pgftognucolumnset{/addplotcubicregression/y}{\@addplotcubicregression@coly}%
    \edef\@addplotcubicregression@inita{\pgfkeysvalueof{/addplotcubicregression/a}}%
    \edef\@addplotcubicregression@initb{\pgfkeysvalueof{/addplotcubicregression/b}}%
    \edef\@addplotcubicregression@initc{\pgfkeysvalueof{/addplotcubicregression/c}}%
    \edef\@addplotcubicregression@initd{\pgfkeysvalueof{/addplotcubicregression/d}}%
    \addplot [#1] gnuplot [raw gnuplot] { % allows arbitrary gnuplot commands
        f(x) = a*x**3+b*x**2+c*x+d;     % Define the function to fit
        % Set reasonable starting values here
        a=\@addplotcubicregression@inita;
        b=\@addplotcubicregression@initb;
        c=\@addplotcubicregression@initc;
        d=\@addplotcubicregression@initd;
        fit f(x) '#3' u \@addplotcubicregression@colx:\@addplotcubicregression@coly\space via a,b,c,d; % Select the file
        stats '#3' u \@addplotcubicregression@colx;
        plot [x=STATS_min:STATS_max] f(x);    % Specify the range to plot
        set print "#3-parameters.dat";  % Open a file to save the parameters
        print a, b, c, d;                  % Write the parameters to file
    };
    \pgfplotstableread{#3-parameters.dat}\parameters % Open the file Gnuplot wrote
    \pgfplotstablegetelem{0}{0}\of\parameters \xdef\pgfplotstableregressiona{\pgfplotsretval} % Get first element, save into \pgfplotstableregressiona
    \pgfplotstablegetelem{0}{1}\of\parameters \xdef\pgfplotstableregressionb{\pgfplotsretval}
    \pgfplotstablegetelem{0}{2}\of\parameters \xdef\pgfplotstableregressionc{\pgfplotsretval}
    \pgfplotstablegetelem{0}{3}\of\parameters \xdef\pgfplotstableregressiond{\pgfplotsretval}
}
% regress on y = a (x!)
% \addplotfactorialregression[params for \addplot][default settings for a, also for x and y columns]{table file}
\NewDocumentCommand{\addplotfactorialregression}{ O{no markers} o m}{%
    \pgfkeyssetvalue{/addplotquadraticregression/x}{0}
    \pgfkeyssetvalue{/addplotquadraticregression/y}{1}
    \pgfkeyssetvalue{/addplotquadraticregression/a}{1}
    \pgfkeys{/addplotquadraticregression/.cd,#2}
    \pgftognucolumnset{/addplotquadraticregression/x}{\@addplotquadraticregression@colx}%
    \pgftognucolumnset{/addplotquadraticregression/y}{\@addplotquadraticregression@coly}%
    \edef\@addplotquadraticregression@inita{\pgfkeysvalueof{/addplotquadraticregression/a}}%
    \addplot [#1] gnuplot [raw gnuplot] { % allows arbitrary gnuplot commands
        f(x) = a*gamma(x+1);     % Define the function to fit
        % Set reasonable starting values here
        a=\@addplotquadraticregression@inita;
        fit f(x) '#3' u \@addplotquadraticregression@colx:\@addplotquadraticregression@coly\space via a; % Select the file
        stats '#3' u \@addplotquadraticregression@colx;
        plot [x=STATS_min:STATS_max] f(x);    % Specify the range to plot
        set print "#3-parameters.dat";  % Open a file to save the parameters
        print a;                  % Write the parameters to file
    };
    \pgfplotstableread{#3-parameters.dat}\parameters % Open the file Gnuplot wrote
    \pgfplotstablegetelem{0}{0}\of\parameters \xdef\pgfplotstableregressiona{\pgfplotsretval} % Get first element, save into \pgfplotstableregressiona
}
\makeatother

%%%%%%%%%%%%%%%%%%%%%%%%%%%%%%%%%%%%%%%%%%%%%%%%%%%%%%%%%%%%%%%%%%%%%%
% from category-coq-experience
%%%%%%%%%%%%%%%%%%%%%%%%%%%%%%%%%%%%%%%%%%%%%%%%%%%%%%%%%%%%%%%%%%%%%%
\makeatletter
\newlength{\assizeof@height}
\newlength{\assizeof@depth}
\newlength{\assizeof@width}
\newcommand{\asheightof}[2]{%
    \settoheight{\assizeof@height}{#2}%
    \settowidth{\assizeof@width}{#2}%
    \settodepth{\assizeof@depth}{#2}%
    \raisebox{0pt}[\assizeof@height][\assizeof@depth]{#1}%
}
\newsavebox{\checkmarkdashed@box}
\sbox{\checkmarkdashed@box}{%
    \tikzexternaldisable
    \begin{tikzpicture}[scale=0.0018, x=\baselineskip, y=-\baselineskip]
    %[y=0.80pt,x=0.80pt,yscale=-1, inner sep=0pt, outer sep=0pt]
    \begin{scope}[shift={(-76.14285,-468.51261)}]
    \path[fill=gray] (346.8401,509.3677) .. controls (340.5769,505.0388) and
    (340.7618,500.5122) .. (356.8188,488.4276) .. controls (372.8758,476.3431) and
    (385.5965,469.4556) .. (393.7791,468.6693) .. controls (402.4251,467.8385) and
    (404.1912,473.2846) .. (397.5467,480.2873) .. controls (395.7973,482.1311) and
    (389.1909,487.6189) .. (382.8659,492.4825) .. controls (374.6709,499.4135) and
    (356.1486,515.8013) .. (346.8401,509.3677) -- cycle;
    \path[fill=gray] (260.1089,583.4074) .. controls (266.3140,575.7764) and
    (272.7509,568.2670) .. (279.3743,560.8342) .. controls (324.4398,518.7462) and
    (319.3756,546.8324) .. (289.7561,585.4171) .. controls (282.4707,594.9346) and
    (276.2845,605.1740) .. (270.0874,615.4138) .. controls (215.7257,693.3061) and
    (215.7106,638.2351) .. (260.1089,583.4074) -- cycle;
    \path[fill=gray] (184.4815,709.4074) .. controls (188.7556,700.1304) and
    (193.0479,690.9313) .. (196.6062,682.0703) .. controls (214.1074,660.9549) and
    (227.6958,691.0301) .. (219.6701,711.9379) .. controls (210.8162,735.0030) and
    (203.2310,759.0839) .. (197.0533,784.3044) .. controls (172.5104,824.2648) and
    (170.7368,739.2609) .. (184.4815,709.4074) -- cycle;
    \path[fill=gray] (121.6449,767.9646) .. controls (118.8624,763.3407) and
    (98.3906,735.2020) .. (96.1182,731.3920) .. controls (91.5735,723.7721) and
    (88.0116,717.7146) .. (85.2823,712.9577) .. controls (79.8236,703.4439) and
    (77.6952,699.1324) .. (77.6952,697.9285) .. controls (77.6952,691.5044) and
    (92.7170,679.4441) .. (106.4137,674.8717) .. controls (114.2814,672.2452) and
    (115.9904,672.2686) .. (119.1161,675.0458) .. controls (120.5090,676.2833) and
    (141.7234,710.5494) .. (153.0933,729.2494) .. controls (155.8390,738.5209) and
    (141.8256,804.9784) .. (121.6449,767.9646) -- cycle;
    \end{scope}
    \end{tikzpicture}%
    \tikzexternalenable
}

\newcommand{\checkmarkdashed}{\asheightof{\usebox{\checkmarkdashed@box}}{\checkmark}}
\makeatother
% \cat{C} is the style for a category C
\newcommand{\cat}[1]{\ensuremath{\mathcal{#1}}}
\newcommand{\maybesub}[1][]{\ifthenelse{\equal{#1}{}}{}{\ensuremath{_{#1}}}}
\newcommand{\maybecat}[1][]{\ifthenelse{\equal{#1}{}}{}{\ensuremath{_{\cat{#1}}}}}
\newcommand{\Ob}[1][]{\ensuretext{Ob}\maybecat[#1]}
\newcommand{\Hom}[1][]{\ensuretext{Hom}\maybecat[#1]}
\newcommand{\Id}[1][]{\ensuremath{1}\maybesub[#1]}
\newcommand{\term}[1]{\ensuretexttt{#1}}
\newcommand{\mailto}[1]{\href{mailto:#1}{#1}}

\newcommand{\pullbacksymbol}{{\ensuremath{\lrcorner}}}
\newcommand{\pullbackarrow}[1][dr]{\ar@{}[#1]|<<\pullbacksymbol}
%%%%%%%%%%%%%%%%%%%%%%%%%%%%%%%%%%%%%%%%%%%%%%%%%%%%%%%%%%%%%%%%%%%%%%


\usepackage{soul}
\usepackage{ifthen}
\usepackage{comment}

\provideboolean{showminortodos}

%%%%%%%%%%%%%%%%%%%%%%%
% https://tex.stackexchange.com/a/312583/2066
\newcommand{\ctext}[3][RGB]{%
  \begingroup
  \definecolor{hlcolor}{#1}{#2}\sethlcolor{hlcolor}%
  \hl{#3}%
  \endgroup
}

\provideboolean{final}

\makeatletter
\iffinal
\let\finalwarning=\errmessage
\else
\let\finalwarning=\@latex@warning
\fi
\ifx\@onlypreamble\@notprerr% (document)
\newcommand{\NowOrAtBeginDocument}[1]{#1}%
\else% (preamble)
\newcommand{\NowOrAtBeginDocument}[1]{\AtBeginDocument{#1}}%
\AtBeginDocument{\renewcommand{\NowOrAtBeginDocument}[1]{#1}}%
\fi
\newboolean{readingreadymarks}
\setboolean{readingreadymarks}{true}
\ifthenelse{\boolean{readingreadymarks} \and \not \boolean{final}}{
  \newcommand{\sectionreadingmark}[1]{\texorpdfstring{\textcolor{olive}{#1}}{#1}}
  \newcommand{\readyforreading}[1]{#1 \sectionreadingmark{(Ready for Final Reading)}}
  \newcommand{\readyforreadingmod}[2]{#1 \sectionreadingmark{(Ready for Final Reading modulo #2)}}
}{
  \newcommand{\readyforreading}[1]{#1}
  \newcommand{\readyforreadingmod}[2]{#1}
}
\newcommand{\todo}[1]{%
  \finalwarning{TODO: \detokenize{#1} in file \currfilename}%
  \NowOrAtBeginDocument{\textcolor{red}{[\textbf{TODO:} #1]}}}%
\newcommand{\todofrom}[2]{%
  \finalwarning{TODO: (FROM #1) \detokenize{#2} in file \currfilename}%
  \NowOrAtBeginDocument{\textcolor{OrangeRed}{[\textbf{TODO FROM #1:} #2]}}}%
\newcommand{\todonz}[1]{\todofrom{Nickolai}{#1}}
\newcommand{\minortodo}[1]{%
  \finalwarning{MINOR TODO: \detokenize{#1} in file \currfilename}%
  \ifthenelse{\boolean{showminortodos}}{%
    \NowOrAtBeginDocument{\textcolor{red}{[\textbf{MINOR TODO:} #1]}}%
  }{}}%
\newcommand{\etodo}[1]{\edef\@todotext{#1}\expandafter\todo\expandafter{\@todotext}}%
\newcommand{\eminortodo}[1]{\edef\@todotext{#1}\expandafter\minortodo\expandafter{\@todotext}}%
\newcommand{\todoask}[1]{%
  \finalwarning{QUESTION FOR ADAM: \detokenize{#1} in file \currfilename}%
  \NowOrAtBeginDocument{\textcolor{purple}{[\textbf{QUESTION FOR ADAM:} #1]}}}%
\newcommand{\todoaskreaders}[1]{%
  \finalwarning{QUESTION FOR READERS: \detokenize{#1} in file \currfilename}%
  \NowOrAtBeginDocument{\textcolor{magenta}{[\textbf{QUESTION FOR READERS:} #1]}}}%
\newcommand{\minortodoask}[1]{%
  \finalwarning{MINOR QUESTION FOR ADAM: \detokenize{#1} in file \currfilename}%
  \ifthenelse{\boolean{showminortodos}}{%
    \NowOrAtBeginDocument{\textcolor{purple}{[\textbf{QUESTION FOR ADAM:} #1]}}%
  }{}}%
\ifthenelse{\boolean{showminortodos}}{
  \includecomment{minorcomment}
}{
  \excludecomment{minorcomment}
}
\newcommand{\colorifzero}[3]{%
\ifnum#1=0\relax
\expandafter\@firstoftwo
\else
\expandafter\@secondoftwo
\fi
{\textcolor{#2}{#3}}%
{#3}%
}%
\makeatother
\newcommand{\editedtext}[2][0]{\colorifzero{#1}{blue}{#2}}%
\newcommand{\newtext}[2][0]{\colorifzero{#1}{blue}{#2}}%

\newcommand{\wantsomething}[4][\ctext]{%
  \finalwarning{TODO: #2: \detokenize{#4} in file \currfilename}%
  #1{#3}{#4}%
}

\newcommand{\wantinfo}[2][\ctext]{\wantsomething[#1]{Double Click, Want Info}{yellow}{#2}}
\newcommand{\wantflair}[2][\ctext]{\wantsomething[#1]{Flair}{blue}{#2}}
\newcommand{\wantclear}[2][\ctext]{\wantsomething[#1]{Unclear/Grammar}{green}{#2}}

\iffinal
\else
\AtEndDocument{%
\clearpage
\pagenumbering{roman}
\cleardoublepage

\todo{Run `make update-thesis' before submission to update the date on the cover page}
\todo{Update resume submodule before submission of forms}

\todo{change \string\finalfalse to \string\finaltrue}

\clearpage
\pagenumbering{arabic}
\cleardoublepage
}%
\fi

