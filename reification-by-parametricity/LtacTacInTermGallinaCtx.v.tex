\begin{coqdoccode}
\end{coqdoccode}
\section{Recursing under binders with tactics in terms, tracking variables with Gallina contexts}

\begin{coqdoccode}
\coqdocnoindent
\coqdockw{Require} \coqdockw{Import} \coqdocvar{Reify.ReifyCommon}.\coqdoceol
\coqdocemptyline
\end{coqdoccode}
Points of note:



\begin{itemize}
\item  We make sure to fill in all implicit arguments explicitly, to
      minimize the number of evars generated; evars are one of the
      main bottlenecks.



\item  We must bind open terms to fresh variable names to work around
      the fact that tactics in terms do not correctly support open
      terms (see COQBUG(https://github.com/coq/coq/issues/3248)).



\item  We make use of a trick from ``[coqdev] beta1 and zeta1
      reduction''\footnote{\url{https://sympa.inria.fr/sympa/arc/coqdev/2016-01/msg00060.html}}
      to bind names with a single-branch \coqdockw{match} statement without
      incurring extra β or ζ reductions.



\item  We must unfold aliases bound with this \coqdockw{match} statement trick
      (substitution does not happen until after typechecking), and if
      we are not careful with how we use \coqdoctac{fresh}, Coq will stack
      overflow on \coqdoctac{cbv} \coqdockw{delta} or otherwise misbehave.\footnote{See
      \url{https://github.com/coq/coq/issues/5448},
      \url{https://github.com/coq/coq/issues/6315},
      \url{https://github.com/coq/coq/issues/6559}.}



\item  We give the \coqdockw{return} clause on the \coqdockw{match} statement explicitly.
      Without the explicit return clause, Coq would backtrack on
      failure and attempt a second way of elaborating the \coqdockw{match}
      branches, resulting in a blowup on failure that is exponential
      in the recursive depth of the
      failure.\footnote{\url{https://github.com/coq/coq/issues/6252\#issuecomment-347041995}}
      If we used \coqdockw{return} \coqdocvar{\_}, rather than specifying the type
      explicitly, we incur the cost of allocating an additional evar,
      which is linear in the size of the context.  (This performance
      statistic courtesy of conversations with Pierre-Marie Pédrot on
      Coq's gitter.)



\item  We explicitly \coqdoctac{clear} variable bindings from the context before
      invoking the recursive call, because the cost of evars is
      proportional to the size of the context.  
\end{itemize}


 Much like typeclass-based reification, we introduce a definition
    to track which \coqdocvar{nat} binders reify to which \coqdocvar{var} binders,
    searching the context for such hypotheses. \begin{coqdoccode}
\coqdocemptyline
\coqdocnoindent
\coqdockw{Definition} \coqdocvar{var\_for} \{\coqdocvar{var} : \coqdockw{Type}\} (\coqdocvar{n} : \coqdocvar{nat}) (\coqdocvar{v} : \coqdocvar{var}) := \coqdocvar{False}.\coqdoceol
\coqdocemptyline
\coqdocnoindent
\coqdockw{Ltac} \coqdocvar{reify} \coqdocvar{var} \coqdocvar{term} :=\coqdoceol
\coqdocindent{1.00em}
\coqdockw{let} \coqdocvar{reify\_rec} \coqdocvar{term} := \coqdocvar{reify} \coqdocvar{var} \coqdocvar{term} \coqdoctac{in}\coqdoceol
\coqdocindent{1.00em}
\coqdockw{lazymatch} \coqdockw{goal} \coqdockw{with}\coqdoceol
\coqdocindent{1.00em}
\ensuremath{|} [ \coqdocvar{H} : \coqdocvar{var\_for} \coqdocvar{term} ?\coqdocvar{v} \ensuremath{\vdash} \coqdocvar{\_} ]\coqdoceol
\coqdocindent{2.00em}
\ensuremath{\Rightarrow} \coqdockw{constr}:(@\coqdocvar{Var} \coqdocvar{var} \coqdocvar{v})\coqdoceol
\coqdocindent{1.00em}
\ensuremath{|} \coqdocvar{\_}\coqdoceol
\coqdocindent{2.00em}
\ensuremath{\Rightarrow}\coqdoceol
\coqdocindent{2.00em}
\coqdockw{lazymatch} \coqdocvar{term} \coqdockw{with}\coqdoceol
\coqdocindent{2.00em}
\ensuremath{|} \coqdocvar{O} \ensuremath{\Rightarrow} \coqdockw{constr}:(@\coqdocvar{NatO} \coqdocvar{var})\coqdoceol
\coqdocindent{2.00em}
\ensuremath{|} \coqdocvar{S} ?\coqdocvar{x}\coqdoceol
\coqdocindent{3.00em}
\ensuremath{\Rightarrow} \coqdockw{let} \coqdocvar{rx} := \coqdocvar{reify\_rec} \coqdocvar{x} \coqdoctac{in}\coqdoceol
\coqdocindent{4.50em}
\coqdockw{constr}:(@\coqdocvar{NatS} \coqdocvar{var} \coqdocvar{rx})\coqdoceol
\coqdocindent{2.00em}
\ensuremath{|} ?\coqdocvar{x} \ensuremath{\times} ?\coqdocvar{y}\coqdoceol
\coqdocindent{3.00em}
\ensuremath{\Rightarrow} \coqdockw{let} \coqdocvar{rx} := \coqdocvar{reify\_rec} \coqdocvar{x} \coqdoctac{in}\coqdoceol
\coqdocindent{4.50em}
\coqdockw{let} \coqdocvar{ry} := \coqdocvar{reify\_rec} \coqdocvar{y} \coqdoctac{in}\coqdoceol
\coqdocindent{4.50em}
\coqdockw{constr}:(@\coqdocvar{NatMul} \coqdocvar{var} \coqdocvar{rx} \coqdocvar{ry})\coqdoceol
\coqdocindent{2.00em}
\ensuremath{|} (\coqdocvar{dlet} \coqdocvar{x} := ?\coqdocvar{v} \coqdoctac{in} ?\coqdocvar{f})\coqdoceol
\coqdocindent{3.00em}
\ensuremath{\Rightarrow} \coqdockw{let} \coqdocvar{rv} := \coqdocvar{reify\_rec} \coqdocvar{v} \coqdoctac{in}\coqdoceol
\coqdocindent{4.50em}
\coqdockw{let} \coqdocvar{not\_x} := \coqdoctac{fresh}  \coqdoctac{in} \coqdoceol
\coqdocindent{4.50em}
\coqdockw{let} \coqdocvar{not\_x2} := \coqdoctac{fresh}  \coqdoctac{in} \coqdoceol
\coqdocindent{4.50em}
\coqdockw{let} \coqdocvar{rf}\coqdoceol
\coqdocindent{6.50em}
:=\coqdoceol
\coqdocindent{6.50em}
\coqdockw{lazymatch}\coqdoceol
\coqdocindent{7.50em}
\coqdockw{constr}:(\coqdoceol
\coqdocindent{8.50em}
\coqdockw{fun} (\coqdocvar{x} : \coqdocvar{nat}) (\coqdocvar{not\_x} : \coqdocvar{var}) (\coqdocvar{\_} : @\coqdocvar{var\_for} \coqdocvar{var} \coqdocvar{x} \coqdocvar{not\_x})\coqdoceol
\coqdocindent{8.50em}
\ensuremath{\Rightarrow} \coqdockw{match} \coqdocvar{f} \coqdockw{return} @\coqdocvar{expr} \coqdocvar{var} \coqdockw{with} \coqdoceol
\coqdocindent{10.00em}
\ensuremath{|} \coqdocvar{not\_x2}\coqdoceol
\coqdocindent{11.00em}
\ensuremath{\Rightarrow} \coqdockw{ltac}:(\coqdockw{let} \coqdocvar{fx} := (\coqdoctac{eval} \coqdoctac{cbv} \coqdockw{delta} [\coqdocvar{not\_x2}] \coqdoctac{in} \coqdocvar{not\_x2}) \coqdoctac{in}\coqdoceol
\coqdocindent{15.50em}
\coqdoctac{clear} \coqdocvar{not\_x2};\coqdoceol
\coqdocindent{15.50em}
\coqdockw{let} \coqdocvar{rf} := \coqdocvar{reify\_rec} \coqdocvar{fx} \coqdoctac{in}\coqdoceol
\coqdocindent{15.50em}
\coqdoctac{exact} \coqdocvar{rf})\coqdoceol
\coqdocindent{10.00em}
\coqdockw{end})\coqdoceol
\coqdocindent{6.50em}
\coqdockw{with}\coqdoceol
\coqdocindent{6.50em}
\ensuremath{|} \coqdockw{fun} \coqdocvar{\_} \coqdocvar{v'} \coqdocvar{\_} \ensuremath{\Rightarrow} @?\coqdocvar{f} \coqdocvar{v'} \ensuremath{\Rightarrow} \coqdocvar{f}\coqdoceol
\coqdocindent{6.50em}
\ensuremath{|} ?\coqdocvar{f} \ensuremath{\Rightarrow} \coqdocvar{error\_cant\_elim\_deps} \coqdocvar{f}\coqdoceol
\coqdocindent{6.50em}
\coqdockw{end} \coqdoctac{in}\coqdoceol
\coqdocindent{4.50em}
\coqdockw{constr}:(@\coqdocvar{LetIn} \coqdocvar{var} \coqdocvar{rv} \coqdocvar{rf})\coqdoceol
\coqdocindent{2.00em}
\ensuremath{|} ?\coqdocvar{v}\coqdoceol
\coqdocindent{3.00em}
\ensuremath{\Rightarrow} \coqdocvar{error\_bad\_term} \coqdocvar{v}\coqdoceol
\coqdocindent{2.00em}
\coqdockw{end}\coqdoceol
\coqdocindent{1.00em}
\coqdockw{end}.\coqdoceol
\coqdocemptyline
\coqdocnoindent
\coqdockw{Ltac} \coqdocvar{Reify} \coqdocvar{x} := \coqdocvar{Reify\_of} \coqdocvar{reify} \coqdocvar{x}.\coqdoceol
\coqdocnoindent
\coqdockw{Ltac} \coqdocvar{do\_Reify\_rhs} \coqdocvar{\_} := \coqdocvar{do\_Reify\_rhs\_of} \coqdocvar{Reify} ().\coqdoceol
\coqdocnoindent
\coqdockw{Ltac} \coqdocvar{post\_Reify\_rhs} \coqdocvar{\_} := \coqdocvar{ReifyCommon.post\_Reify\_rhs} ().\coqdoceol
\coqdocnoindent
\coqdockw{Ltac} \coqdocvar{Reify\_rhs} \coqdocvar{\_} := \coqdocvar{Reify\_rhs\_of} \coqdocvar{Reify} ().\coqdoceol
\end{coqdoccode}
